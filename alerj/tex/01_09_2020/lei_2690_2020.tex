\documentclass[10pt]{article}
\usepackage[portuguese]{babel}
\usepackage[utf8]{inputenc}
\usepackage[pdftex]{graphicx}
\usepackage{venndiagram}
\usepackage{subcaption}
\usepackage{caption}
\usepackage[backend=biber,style=authoryear-ibid]{biblatex}
\usepackage[normalem]{ulem}
\usepackage[margin=0.8in]{geometry}
\addbibresource{psychoanalysis.bib}
\graphicspath{{Pictures/}}
\usepackage{tikz}
\usepackage{setspace}
\usepackage{enumitem}
\usepackage{textcomp}
\usepackage{hyperref}


\date{}

\newcommand{\quotebox}[3]{
  \begin{center}
\noindent\fbox{ 
  \parbox{#3\textwidth}{%
  {\itshape#1\itshape}

  \raggedleft {\textbf{#2}} 
    }%  
}
\end{center}
}

\newcommand{\spawnfig}[3]
{
  \begin{figure}[h]
  \centering
  \includegraphics[scale={#3}]{#1}
  \caption{#2}
  \end{figure}
}


\begin{document}
\maketitle
\begin{center}
  % Deputado RODRIGO AMORIM
  \huge
  \vspace{-3cm}\href{http://alerjln1.alerj.rj.gov.br/scpro1923.nsf/f4b46b3cdbba990083256cc900746cf6/eb57848f7270915903258574005f00fb?OpenDocument}{PROJETO DE LEI Nº 2690/2020}
\bigskip
\bigskip
\bigskip
  
\end{center}

\textbf{EMENTA:} 







  Requerimento de Urgência =&gt; 20200302690 =&gt; RODRIGO AMORIM =&gt; A imprimir. Deferido automaticamente nos termos do§ 4º do Art. 127 do Regimento Interno.05/28/2020

  Distribuição =&gt; 20200302690 =&gt; Comissão de Constituição e Justiça =&gt; Relator: ROSENVERG REIS =&gt; Proposição 20200302690 =&gt; Parecer: Pela Constitucionalidade com Emenda07/28/2020






















\bigskip

\noindent
A ASSEMBLEIA LEGISLATIVA DO ESTADO DO RIO DE JANEIRO RESOLVE:

\begin{enumerate}[label=Art. \arabic*\textdegree]
\item - Fica determinado que as locações de casas de festas e buffets no âmbito do Estado do Rio de Janeiro poderão ser remarcados, a pedido do consumidor, em razão da doença COVID-19, causada pelo novo coronavírus (Sars-Cov-2).

§ 1º A casa de festa e/ou buffet deverá remarcar a data do evento, a pedido do consumidor, nas mesmas condições previstas contratualmente, para qualquer data disponível, conforme agendamento prévio a ser realizado pelo fornecedor do serviço, até o final do ano de 2021, com isenção de pagamento de qualquer taxa extra, multa ou reajuste anual para a referida alteração;

§ 2° Fica o consumidor obrigado a notificar, por escrito, ao fornecedor do serviço sobre a opção de remarcação de data.

§ 3° O fornecedor de serviço de que trata esta Lei terá o prazo máximo de 7 (sete) dias corridos para efetuar a remarcação solicitada pelo consumidor, sob pena de incorrer nas sanções previstas no artigo 4º desta Lei.
\item - O consumidor poderá ainda, caso não opte pela remarcação da data do evento, optar pela concessão de crédito, no valor do preço pago à época da contratação, com prazo de utilização de 24 (vinte e quatro) meses.

§1° Fica o consumidor obrigado a notificar, por escrito, ao fornecedor do serviço sobre a opção de adquirir o crédito previsto no caput deste artigo. 

§2° A data da notificação prevista no parágrafo 1° será considerada para o início da contagem do prazo previsto no caput deste artigo.

\item - Nos casos em que o consumidor optar pelo cancelamento, o prazo para o reembolso do valor relativo à locação da casa de festa e/ou buffet será até 12 (doze) meses, a partir de 01 de janeiro de 2021, observadas as regras do contrato de serviço contratado.

\item - O descumprimento ao que dispõe a presente lei acarretará ao infrator multa no valor de 3.000 (três mil) UFIR-RJ por cada autuação, a ser revertida para o Fundo Especial para Programas de Proteção e Defesa do Consumidor - FEPROCON.

\item - Esta Lei se destina a vigência temporária pelo período de 06 (seis) meses, podendo ser renovada por igual período enquanto perdurar a proliferação da doença COVID-19, doença causada pelo novo coronavírus (Sars-Cov-2) pela Organização Mundial da Saúde.

\item - Esta Lei entra em vigor na data de sua publicação.

\end{enumerate}




\begin{center}
  Plenário Barbosa Lima Sobrinho, 25 de maio de 2020.

   \bigskip

  \textbf{ RODRIGO AMORIM}

  \bigskip

  \textbf{JUSTIFICATIVA}
  \bigskip

\end{center}

  Atualmente vivemos uma situação nunca experimentada, em razão da pandemia da COVID-19, decretada pela OMS (Organização Mundial de Saúde). Desde o início, o Estado do Rio de Janeiro vem adotando protocolos a fim de obstar a proliferação do vírus, sendo o isolamento social decretado em 16/03/2020 e quase 03 (três) meses depois não há qualquer sinalização de retorno no funcionamento, ainda que gradual e restrito, dos comércios e serviços.  

O presente projeto de Lei é fruto da reivindicação da Associação de Casas de Festa Infantil do Rio de Janeiro (ACAFIRJ), uma vez que o referido setor foi afetado frontalmente, uma vez que congregação de alto número de pessoas é inerente ao seu objeto social, sendo obrigadas a suspender suas atividades.

Tal segmento, possui grande relevância para a economia do Estado do Rio de Janeiro, notadamente pela geração de inúmeros empregos, circulação de bens e serviços e recolhimento de impostos. Ressalte-se que, com a determinação de isolamento social, houve prejuízo imenso ao segmento.

Assim, o presente projeto de Lei visa regulamentar tal situação de forma a não prejudicar tanto os consumidores quanto as Casas de Festas e Buffets, como forma de ponderação de interesses, motivo pelo qual pugna-se a sua aprovação por esta Casa de Leis.

Rio de Janeiro, dia 25 de maio de 2020.

DEPUTADO RODRIGO AMORIM



\iffalse
\begin{center}
  \textbf{REFERÊNCIAS}
\end{center}


\fi



\end{document}

