\documentclass[10pt]{article}
\usepackage[portuguese]{babel}
\usepackage[utf8]{inputenc}
\usepackage[pdftex]{graphicx}
\usepackage{venndiagram}
\usepackage{subcaption}
\usepackage{caption}
\usepackage[backend=biber,style=authoryear-ibid]{biblatex}
\usepackage[normalem]{ulem}
\usepackage[margin=0.8in]{geometry}
\addbibresource{psychoanalysis.bib}
\graphicspath{{Pictures/}}
\usepackage{tikz}
\usepackage{setspace}
\usepackage{enumitem}
\usepackage{textcomp}
\usepackage{hyperref}


\date{}

\newcommand{\quotebox}[3]{
  \begin{center}
\noindent\fbox{ 
  \parbox{#3\textwidth}{%
  {\itshape#1\itshape}

  \raggedleft {\textbf{#2}} 
    }%  
}
\end{center}
}

\newcommand{\spawnfig}[3]
{
  \begin{figure}[h]
  \centering
  \includegraphics[scale={#3}]{#1}
  \caption{#2}
  \end{figure}
}


\begin{document}
\maketitle
\begin{center}
  % Deputada MARTHA ROCHA
  \huge
  \vspace{-3cm}\href{http://alerjln1.alerj.rj.gov.br/scpro1519.nsf/f4b46b3cdbba990083256cc900746cf6/064b8a46fa033c5b8325828900628aea?OpenDocument}{PROJETO DE LEI Nº 4075/2018}
\bigskip
\bigskip
\bigskip
  
\end{center}

\textbf{EMENTA:} 
AUTORIZA O PODER EXECUTIVO A REALIZAR CONVÊNIO COM INSTITUIÇÕES PÚBLICAS E PRIVADAS DE ENSINO SUPERIOR PARA PRESTAÇÃO DE ESTÁGIO SUPERVISIONADO DE PSICOLOGIA, DE SERVIÇO SOCIAL E DE ÁREAS AFINS EM UNIDADES PRISIONAIS E SOCIOEDUCATIVAS, NO ÂMBITO DO ESTADO DO RIO DE JANEIRO.








\bigskip

\noindent
A ASSEMBLEIA LEGISLATIVA DO ESTADO DO RIO DE JANEIRO RESOLVE:

\begin{enumerate}[label=Art. \arabic*\textdegree]

\item - Fica o Poder Executivo autorizado a realizar convênio com instituições públicas e privadas de ensino superior para prestação de estágio supervisionado de psicologia, de serviço social e de áreas afins em unidades prisionais da Secretaria de Estado de Administração Penitenciária (SEAP) e unidades socioeducativas do Departamento Geral de Ações Socioeducativas (DEGASE-SEE), no âmbito do Estado do Rio de Janeiro.

\item - O Poder Executivo poderá disponibilizar ajuda de custo aos estagiários, prestadores do serviço disposto no caput do Art. 1º.
§ 1º - O Poder Executivo poderá isentar o estagiário de pagamento de taxa de inscrição em concurso público específico de sua área, pelo tempo correspondente ao de serviço voluntário, em caso de impossibilidade de provisão da ajuda de custo.
§ 2º - O período de estágio supervisionado nas Unidades contará como título e/ou prática no concurso específico de sua área. 

\item - As despesas decorrentes da aplicação desta lei correrão por conta de dotação orçamentária própria.

\item - O Poder Executivo regulamentará esta Lei.

\item - Esta Lei entrará em vigor na data de sua publicação.




















\end{enumerate}




\begin{center}
  Plenário Barbosa Lima Sobrinho, 10 de maio de 2018

   \bigskip

  \textbf{ MARTHA ROCHA}

  \bigskip

  \textbf{JUSTIFICATIVA}
  \bigskip

\end{center}

  	A SEAP e o DEGASE vivem em uma crise de infraestrutura, de pessoal e, evidentemente, de recursos financeiros. Se por um lado as unidades estão superlotadas, por outro lado o serviço prestado se encontra deficitário dada a supressão de demanda do sistema. Estimam-se mais de 9 unidades do DEGASE para resolver o atual problema de superlotação. Já a superlotação da SEAP está calculada em 200\%.
	Em termos de pessoal, o Estado tem um déficit de mais de 3 mil agentes penitenciários e de mil agentes socioeducativos. Além do mais, não há reabilitação psicossocial e projetos de reinserção de adolescentes e detentos no Rio de Janeiro. Portanto, é mais do que primordial a execução desse projeto com as nossas instituições educativas e o aproveitamento da qualificação de psicólogos e servidores sociais a benefício da sociedade e dos próprios estudantes.
	Foram estabelecidos estímulos aos estagiários que quiserem prestar serviços nos termos aqui propostos.
	Por essas razões supracitadas, convido meus nobres pares a aprovarem este Projeto de Lei e a permanecerem em vigília para execução deste na prática.



\iffalse
\begin{center}
  \textbf{REFERÊNCIAS}
\end{center}


\fi



\end{document}

