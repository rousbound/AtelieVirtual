\documentclass[10pt]{article}
\usepackage[portuguese]{babel}
\usepackage[utf8]{inputenc}
\usepackage[pdftex]{graphicx}
\usepackage{venndiagram}
\usepackage{subcaption}
\usepackage{caption}
\usepackage[backend=biber,style=authoryear-ibid]{biblatex}
\usepackage[normalem]{ulem}
\usepackage[margin=0.8in]{geometry}
\addbibresource{psychoanalysis.bib}
\graphicspath{{Pictures/}}
\usepackage{tikz}
\usepackage{setspace}
\usepackage{enumitem}
\usepackage{textcomp}
\usepackage{hyperref}


\date{}

\newcommand{\quotebox}[3]{
  \begin{center}
\noindent\fbox{ 
  \parbox{#3\textwidth}{%
  {\itshape#1\itshape}

  \raggedleft {\textbf{#2}} 
    }%  
}
\end{center}
}

\newcommand{\spawnfig}[3]
{
  \begin{figure}[h]
  \centering
  \includegraphics[scale={#3}]{#1}
  \caption{#2}
  \end{figure}
}


\begin{document}
\maketitle
\begin{center}
  %  DEFENSORIA PÚBLICA
  \huge
  \vspace{-3cm}\href{http://alerjln1.alerj.rj.gov.br/scpro1923.nsf/f4b46b3cdbba990083256cc900746cf6/eb9d757e8b554e6f032585d1006eac39?OpenDocument}{PROJETO DE LEI Nº 3081/2020}
\bigskip
\bigskip
\bigskip
  
\end{center}

\textbf{EMENTA:} 
ALTERA A LEI Nº. 1146, DE 26 DE FEVEREIRO DE 1987, E DÁ OUTRAS PROVIDÊNCIAS 








\bigskip

\noindent
A ASSEMBLEIA LEGISLATIVA DO ESTADO DO RIO DE JANEIRO RESOLVE:

\begin{enumerate}[label=Art. \arabic*\textdegree]
\item - Os arts. 1º, 2º, 3º e 5º da Lei nº 1146, de 26 de fevereiro de 1987, que cria o Centro de Estudos Jurídicos da Assistência Judiciária do Estado do Rio de Janeiro, passam a vigorar com a seguinte redação:


\item - Centro de Estudos Jurídicos da Defensoria Pública do Estado do Rio de Janeiro, diretamente subordinado ao(à) Defensor(a) Público(a) Geral do Estado, com as seguintes atribuições:
\begin{enumerate}[label=\Roman*]
\item - promover a capacitação, especialização e atualização técnico-profissional das pessoas que integram a Defensoria Pública do Estado do Rio de Janeiro;
\end{enumerate}
......................................................................
\item - participar da organização do curso de preparação à carreira destinado a defensores(as) públicos(as) e servidores(as) em estágio probatório;
\item - divulgar toda matéria de natureza jurídica de interesse da Defensoria Pública;
\item - promover concursos e premiações voltados ao aperfeiçoamento da atuação institucional e à difusão dos direitos humanos;
\item - promover atividades de ensino, capacitação, pesquisa e extensão, proporcionando a acadêmicos de cursos de nível superior e bacharéis em direito o conhecimento teórico e prático nas áreas de atuação da Defensoria Pública, sob supervisão da Coordenação Geral de Estágio Forense e Residência Jurídica e orientação acadêmica do Centro de Estudos Jurídicos;
..........................................................................
\end{enumerate}
Parágrafo único - O Centro de Estudos Jurídicos poderá desenvolver outras atividades que lhe forem conferidas, desde que conexas com as atribuições mencionadas nos incisos deste artigo.

\item - Fica instituído Fundo Orçamentário Especial destinado a atender às despesas efetuadas pelo Centro de Estudos Jurídicos no desempenho das atribuições previstas no art. 1º, podendo ser utilizado também em prol do aparelhamento material da Defensoria Pública do Estado do Rio de Janeiro.

\item - 
\begin{enumerate}[label=\Roman*]
\item - os honorários que caibam à Defensoria Pública em qualquer processo judicial, bem como em atuações extrajudiciais;
.....................................................................................
Parágrafo único. A Defensoria Pública do Estado do Rio de Janeiro, pelos seus órgãos de atuação, postulará e executará as verbas mencionadas no inciso I deste artigo, observadas as isenções previstas no art. 91, caput, e no § 1º do art. 1.007 da Lei Federal nº 13.105, de 16 de março de 2015 - Código de Processo Civil.

\item - Os recursos do Fundo serão movimentados em contas específicas.
Parágrafo único - As verbas mencionadas no art. 3º, inciso I, serão depositadas diretamente nas contas a que se refere o presente artigo.


\item - Ficam revogados os incisos II, V, VI e VII do art. 3º da Lei nº 1146, de 26 de fevereiro de 1987, que cria o Centro de Estudos Jurídicos da Assistência Judiciária do Estado do Rio de Janeiro.

\item - As despesas resultantes da aplicação desta Lei correrão à conta das dotações orçamentárias próprias da Defensoria Pública, previstas na Lei 1146, de 26 de fevereiro de 1987, bem como na Lei nº. 4664, de 14 de dezembro de 2005, observadas as disponibilidades orçamentária e financeira e os limites estabelecidos pela Lei Complementar Federal nº 101, de 04 de maio de 2000.

\item - Esta Lei entra em vigor na data de sua publicação. 



\end{enumerate}




\begin{center}
  

   \bigskip

  \textbf{  DEFENSORIA PÚBLICA}

  \bigskip

  \textbf{JUSTIFICATIVA}
  \bigskip

\end{center}

  MENSAGEM Nº 01/2020 &#- Ofício DPGERJ/SEGAB Nº.418/2020



Excelentíssimo Senhor Presidente da Assembleia Legislativa,


Tenho a honra de cumprimentá-lo e encaminhar a Vossa Excelência o presente projeto de lei, que dispõe sobre modificações à Lei nº 1146, de 26 de fevereiro de 1987, que cria o Centro de Estudos Jurídicos da Defensoria Pública do Estado do Rio de Janeiro.

O principal objetivo desta proposição legislativa é promover uma atualização na Lei do Centro de Estudos Jurídicos, especialmente para consolidar os projetos exitosos envolvendo acadêmicos de cursos de nível superior e bacharéis em direito, que recebem conhecimentos práticos e teóricos relacionados às atividades fins da Defensoria Pública, enquanto colaboram com a prestação de serviços jurídicos, sob a orientação e supervisão de defensores públicos.

Aproveito a oportunidade para solicitar tramitação em regime de urgência, na forma do art. 127 do Regimento Interno da Assembleia Legislativa, tendo em vista que a participação desses acadêmicos e bacharéis tem se mostrado ainda mais relevante nesse contexto de pandemia, razão pela qual a regulamentação proposta nesse projeto de lei colaborará significativamente para a melhoria do atendimento à população em meio à pandemia de Covid-19.

Ao ensejo, renovo os votos e elevada estima e consideração.
Rio de Janeiro, 27 de agosto de 2020.


Rodrigo Baptista Pacheco
Defensor Pública Geral do Estado

	A Lei estadual nº 1.146, que criou o Centro de Estudos Jurídicos da Defensoria Pública e o Fundo Orçamentário Especial destinado a custear as despesas efetuadas pelo órgão, data de 1987. Tem, portanto, mais de 30 anos. É anterior à Constituição de 1988 e ao vigoroso crescimento institucional da Defensoria Pública. É certo que algumas modificações pontuais foram feitas desde o longínquo ano de 1987, mas a Lei carece de uma atualização mais abrangente, tanto assim que ainda contém dispositivo &#- que há de ser alterado &#- fazendo referência ao antigo nome da instituição, &#``Assistência Judiciária&#" (inciso IX do art. 1º).
	Pretende-se, portanto, atualizar a Lei nº 1.146/1987, que continua sendo muito importante para a Defensoria Pública do Estado do Rio de Janeiro.
Uma das principais alterações propostas é a consolidação em lei da atuação do Centro de Estudos Jurídicos da Defensoria Pública no fomento ao ensino, capacitação, pesquisa e extensão voltados para acadêmicos de cursos de nível superior e bacharéis em direito com o especial objetivo de difundir conhecimentos práticos e teóricos relacionados às atividades fins da Defensoria Pública. 
	Dessa forma, acadêmicos e bacharéis vêm se capacitando e especializando nas áreas de atuação da Defensoria Pública, colocando em prática os ensinamentos teóricos e familiarizando-se com as peculiaridades da prestação do serviço de assistência jurídica às pessoas em situação de vulnerabilidade.
	Esses programas têm se mostrado muito eficientes na formação de novos profissionais mais atentos e empáticos às dificuldades vividas pelas pessoas e famílias que buscam diuturnamente o atendimento jurídico gratuito da Defensoria Pública, colaborando também para a prestação desse atendimento.
	Além disso, busca-se alguns ajustes para a compatibilização plena da Lei com o atual regime constitucional da Defensoria Pública. Nesse sentido, à vista da autonomia constitucional da instituição, encarece-se a revogação expressa do dispositivo segundo o qual determinadas aplicações do Centro de Estudos Jurídicos ficariam condicionadas à prévia autorização do Governador (inciso X do art. 1º, na redação atual). Ainda que a revogação tácita do dispositivo se afigure evidente, a supressão expressa mostra-se imperiosa, a fim de que o texto legal não permaneça tão divorciado da ordem constitucional.
	Também é importante harmonizar a Lei nº 1.146/1987 com a lei orgânica nacional da Defensoria Pública (Lei Complementar nº 80/1994, modificada amplamente pela Lei Complementar nº 132/2009). Indica a lei orgânica nacional, no inciso XXI do seu art. 4º, duas destinações para as verbas de honorários, quais sejam: a capacitação profissional dos(as) integrantes da instituição e o aparelhamento da Defensoria Pública. Convém que as mesmas destinações sejam reproduzidas no plano estadual, o que se consegue por meio da alteração, ora sugerida, do art. 2º da Lei nº 1.146.
	Mais uma atualização inevitável é a supressão dos comandos da Lei nº 1.146 que aludem à Escola Superior da Defensoria Pública Geral do Estado do Rio de Janeiro &#- ESU/DP, entidade que não existe mais.
	Outro ponto importante da proposta diz respeito aos honorários auferidos pela Defensoria Pública. Nos termos da Lei Complementar federal nº 80/1994, a execução e recebimento de honorários consiste em função institucional expressa da Defensoria Pública (art. 4º, XXI).
	Cuida-se de função institucional que não pode ser negligenciada, representando indispensável fonte de custeio para atividades relevantes da Defensoria Pública em matéria de capacitação do seu pessoal, ou mesmo extrapolando o âmbito interno da instituição, como se vê pelo inciso XII do art. 1º do texto vigente (que confere ao Centro de Estudos Jurídicos a incumbência de &#``apoiar atividades desenvolvidas pela Defensoria Pública que promovam a difusão e a conscientização dos direitos humanos, da cidadania e do ordenamento jurídico&#").
	Saliente-se que a receita de honorários tem o mérito de não envolver dotações orçamentárias do Tesouro, mas sim recursos próprios, derivando dos ingentes esforços da instituição na sua lida diária em prol de pessoas e grupos carentes.
	Conforme bem observado pelo Ministro Roberto Barroso, do Supremo Tribunal Federal, em voto proferido no julgamento da Repercussão Geral no Recurso Extraordinário 1.140.005, em 03/08/18, a receita de honorários pode contribuir para ao menos amenizar o desnivelamento orçamentário ainda existente da Defensoria em relação a outras instituições do mundo jurídico:




&#``Além disso, é fato notório que a maior parte das Defensorias Públicas enfrenta problemas de estruturação de seus órgãos, situação que, em muitos Estados, não corresponde ao grau de aparelhamento do Poder Judiciário e do Ministério Público, a indicar a existência de um desfavorecimento da instituição na escolha das prioridades orçamentárias. Essa situação, inegavelmente, compromete a atuação constitucional da Defensoria Pública, e poderia ser atenuada pelo recebimento de honorários.&#"




	Atenta à relevância dessa receita, a proposta deixa claro, mediante a adição de parágrafo único ao art. 3º, que a postulação e execução dos honorários compete a todos os órgãos de atuação da Defensoria Pública, o que vem referendar prática institucional implementada há várias décadas e inegavelmente bem-sucedida. Além disso, prevê-se a possibilidade de os honorários resultarem de atuações extrajudiciais, o que também já ocorre na prática, notadamente em acordos coletivos firmados pelo Núcleo de Defesa do Consumidor da Defensoria Pública.



\iffalse
\begin{center}
  \textbf{REFERÊNCIAS}
\end{center}


\fi



\end{document}

