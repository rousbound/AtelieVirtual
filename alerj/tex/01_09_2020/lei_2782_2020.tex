\documentclass[10pt]{article}
\usepackage[portuguese]{babel}
\usepackage[utf8]{inputenc}
\usepackage[pdftex]{graphicx}
\usepackage{venndiagram}
\usepackage{subcaption}
\usepackage{caption}
\usepackage[backend=biber,style=authoryear-ibid]{biblatex}
\usepackage[normalem]{ulem}
\usepackage[margin=0.8in]{geometry}
\addbibresource{psychoanalysis.bib}
\graphicspath{{Pictures/}}
\usepackage{tikz}
\usepackage{setspace}
\usepackage{enumitem}
\usepackage{textcomp}
\usepackage{hyperref}


\date{}

\newcommand{\quotebox}[3]{
  \begin{center}
\noindent\fbox{ 
  \parbox{#3\textwidth}{%
  {\itshape#1\itshape}

  \raggedleft {\textbf{#2}} 
    }%  
}
\end{center}
}

\newcommand{\spawnfig}[3]
{
  \begin{figure}[h]
  \centering
  \includegraphics[scale={#3}]{#1}
  \caption{#2}
  \end{figure}
}


\begin{document}
\maketitle
\begin{center}
  % Deputado CARLOS MINC
  \huge
  \vspace{-3cm}\href{http://alerjln1.alerj.rj.gov.br/scpro1923.nsf/f4b46b3cdbba990083256cc900746cf6/a578ac35ff03555d03258590000027b2?OpenDocument}{PROJETO DE LEI Nº 2782/2020}
\bigskip
\bigskip
\bigskip
  
\end{center}

\textbf{EMENTA:} 
CRIA O PROGRAMA &#``PORTAL DO CONHECIMENTO&#" PARA A PUBLICAÇÃO DE CONTEÚDOS CURRICULARES ELABORADOS POR PROFESSORES E PROFESSORAS DA REDE PÚBLICA ESTADUAL DE ENSINO.








\bigskip

\noindent
A ASSEMBLEIA LEGISLATIVA DO ESTADO DO RIO DE JANEIRO RESOLVE:

\begin{enumerate}[label=Art. \arabic*\textdegree]
\item - Fica instituído o programa Portal do Conhecimento a ser desenvolvido pelas  secretarias de Educação e de Ciência e Tecnologia.

Parágrafo único - As secretarias mencionadas no caput, ou órgãos que vierem a substituí-las, providenciarão uma plataforma virtual que será utilizada para a inserção de aulas a serem disponibilizadas para os estudantes regularmente matriculados no segundo segmento do ensino fundamental e no ensino médio.

\item - As aulas do Portal do Conhecimento serão elaboradas e disponibilizadas por professores e professoras das redes estaduais de ensino, individualmente ou por equipes.

§ 1º - A estes profissionais serão garantidos os direitos autorais sobre suas aulas através da sua devida identificação no Portal.

§ 2º - As aulas serão agrupadas por disciplinas e poderão abarcar mais de um componente curricular desde que sejam correlatos.  

§ 3º - A bibliografia utilizada e as fontes, se houver, serão necessariamente citadas.

\item - O Portal do Conhecimento ficará permanentemente aberto a consultas de estudantes regularmente matriculados nas redes públicas estaduais e seu acesso remoto não substitui a freqüência às aulas presenciais.

\item - O Portal do Conhecimento contará com uma equipe de especialistas por disciplina que monitorará as publicações.

\item - Anualmente serão premiados os autores das 10 melhores aulas publicadas e os 10 professores ou equipes de professores que mais publicaram em cada disciplina, sendo as aulas premiadas reunidas em uma edição a ser enviada como material didático a todas as escolas da rede pública estadual.
\item - As secretarias de Educação e de Ciência e Tecnologia firmarão parceria com as universidades estaduais para o cumprimento da presente lei.

\item - Os recursos que custearão as despesas decorrentes da presente lei farão parte dos orçamentos anuais em rubricas próprias da Função Educação.

\item - As secretarias de Educação e de Ciência e Tecnologia regulamentarão de forma conjunta a presente lei.

\item - Esta lei entra em vigor na data de sua publicação.   

\end{enumerate}




\begin{center}
  Plenário Barbosa Lima Sobrinho, 22 de Junho de 2020.

   \bigskip

  \textbf{ CARLOS MINC}

  \bigskip

  \textbf{JUSTIFICATIVA}
  \bigskip

\end{center}

  A pandemia que estamos vivendo parece não ter data para terminar e, segundo especialistas, enquanto não houver uma vacina, estaremos sujeitos a reinfestações sucessivas com a necessidade de suspensão de aulas entre outras atividades. 

Os estudantes precisarão, tanto para o caso de novas suspensões de aulas presenciais como para o necessário reforço escolar pós pandemia, que lhes sejam fornecidos meios de acesso a plataformas virtuais de ensino, além de livros e apostilas. Os chips para celulares, enquanto não houver em todas as cidades e comunidades redes públicas de acesso à internet , passam a ser  material didático imprescindível.

O ensino remoto não substitui a necessária interação professor alunos. Aprender é trocar ideias, experiências e saberes. No entanto é preciso que a rede pública de ensino conte com ferramentas de apoio para aulas remotas e nada melhor que contar com a expertise dos nossos professores e professoras. E este trabalho, pela sua importância e relevância precisa ser reconhecido, divulgado e premiado pelo Poder Público.
  



\iffalse
\begin{center}
  \textbf{REFERÊNCIAS}
\end{center}


\fi



\end{document}

