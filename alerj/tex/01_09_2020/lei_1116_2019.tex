\documentclass[10pt]{article}
\usepackage[portuguese]{babel}
\usepackage[utf8]{inputenc}
\usepackage[pdftex]{graphicx}
\usepackage{venndiagram}
\usepackage{subcaption}
\usepackage{caption}
\usepackage[backend=biber,style=authoryear-ibid]{biblatex}
\usepackage[normalem]{ulem}
\usepackage[margin=0.8in]{geometry}
\addbibresource{psychoanalysis.bib}
\graphicspath{{Pictures/}}
\usepackage{tikz}
\usepackage{setspace}
\usepackage{enumitem}
\usepackage{textcomp}
\usepackage{hyperref}


\date{}

\newcommand{\quotebox}[3]{
  \begin{center}
\noindent\fbox{ 
  \parbox{#3\textwidth}{%
  {\itshape#1\itshape}

  \raggedleft {\textbf{#2}} 
    }%  
}
\end{center}
}

\newcommand{\spawnfig}[3]
{
  \begin{figure}[h]
  \centering
  \includegraphics[scale={#3}]{#1}
  \caption{#2}
  \end{figure}
}


\begin{document}
\maketitle
\begin{center}
  % Deputado MÁRCIO CANELLA
  \huge
  \vspace{-3cm}\href{http://alerjln1.alerj.rj.gov.br/scpro1923.nsf/f4b46b3cdbba990083256cc900746cf6/48d90fd585ae2a638325843b00637364?OpenDocument}{PROJETO DE LEI Nº 1116/2019}
\bigskip
\bigskip
\bigskip
  
\end{center}

\textbf{EMENTA:} 
ALTERA A LEI Nº 5.645, DE 06 DE JANEIRO DE 2010, PARA INSTITUIR NO CALENDÁRIO OFICIAL DO ESTADO DO RIO DE JANEIRO  O DIA ESTADUAL E A CAMPANHA DE PREVENÇÃO E COMBATE À INFECÇÃO GENERALIZADA (SEPSE) NOS HOSPITAIS E DEMAIS UNIDADES DE SAÚDE DO ESTADO DO RIO DE JANEIRO.








\bigskip

\noindent
A ASSEMBLEIA LEGISLATIVA DO ESTADO DO RIO DE JANEIRO RESOLVE:

\begin{enumerate}[label=Art. \arabic*\textdegree]

\item - Fica instituída no Estado do Rio de Janeiro a Campanha de Prevenção  e Combate à Infecção Generalizada (SEPSE) nos hospitais e demais unidades de saúde do Estado do Rio de Janeiro, a ser promovida anualmente durante todo o mês de setembro, com o objetivo de conscientizar e esclarecer a população e os profissionais da saúde sobre os riscos da infecção generalizada e as formas de sua identificação precoce e devido tratamento, inclusive fora do ambiente hospitalar.

§ 1º - Fica instituído o dia 13 de setembro como o Dia Estadual de Prevenção e Combate à Infecção Generalizada (SEPSE), somando forças na divulgação e propagação das campanhas desenvolvidas ao redor do planeta neste Dia Mundial da SEPSE, com o objetivo de reduzir a taxa de mortalidade pela infecção generalizada.  

§ 2º - No decorrer do mês serão desenvolvidas ações educativas tais como palestras e seminários nos diversos segmentos da sociedade, bem como panfletagem, Mutirões da Saúde  e outras estratégias junto às diversas unidades de saúde do Estado, podendo o Poder Público firmar convênios com os municípios e associações sem fins lucrativos para realização destes atos.

\item - A campanha deverá ser desenvolvida em todas as esferas do poder executivo, em ações unificadas do Poder Executivo Estadual e respectivos municípios, com participação dos profissionais da saúde e enfermagem necessários para a intensificação das ações preventivas na rede de saúde pública e privada do Estado.

\item - A campanha ora instituída passa a integrar o Calendário Oficial de Datas e Eventos do Estado do Rio de Janeiro, passando o Anexo da Lei nº 5.645, de 06 de janeiro de 2010, a vigorar com a seguinte redação:



CALENDÁRIO DATAS COMEMORATIVAS DO ESTADO DO RIO DE JANEIRO:

SETEMBRO

(&#8230;)

MÊS DE SETEMBRO - Mês da Campanha de Prevenção e Combate à Infecção Generalizada - SEPSE.

DIA 13 - Dia Estadual de Prevenção e Combate à Infecção Generalizada - SEPSE.

(...)



\item - As despesas decorrentes da execução desta lei correrão por conta das dotações orçamentárias próprias, suplementadas, se necessário.

\item - Esta lei entra em vigor na data de sua publicação.


\end{enumerate}




\begin{center}
  Plenário Barbosa Lima Sobrinho, 01 de julho de 2019.

   \bigskip

  \textbf{ MÁRCIO CANELLA}

  \bigskip

  \textbf{JUSTIFICATIVA}
  \bigskip

\end{center}

  
Uma pesquisa do Instituto Latinoamericano de Sepse (ILAS) mostra que a maioria da população nunca ouviu falar de SEPSE. A doença já foi chamada de septicemia e o que muita gente não sabe também que o paciente desenvolve mais essa doença fora do ambiente hospitalar. Os dados do Instituto Latinoamericano de Sepse indicam que só 30 a 40\% dos casos vêm do hospital, enquanto que 60 a 70\% das pessoas com Sepse desenvolveram a doença a partir de bactérias, vírus e fungos contraídos fora do ambiente hospitalar.  O ILAS fez uma pesquisa em 134 municípios brasileiros e apurou que 86\% das pessoas nunca ouviram falar da doença, o que explica porque tanta gente morre disso por aqui.  A Sepse nada mais é do que uma resposta inflamatória generalizada, mesmo que a infecção esteja localizada no trato urinário ou no trato respiratório, onde o organismo reage de uma forma exagerada a uma infecção, desestabilizando os sinais vitais desestabilizados e impondo um risco de morte em torno de 50\% em seis horas.

Não é possível que uma enfermidade tão letal possa continuar desconhecida da população, fator que potencializa sua letalidade e amplia os índices de vítimas fatais, sendo vital o conhecimento dos procedimentos preventivos pela equipe de enfermagem e acompanhantes do paciente, inclusive em casa, após a alta hospitalar.   Em razão da importância da matéria, tratando-se de questão de Saúde Pública, conto com o apoio dos meus nobres pares para a aprovação da presente proposição.



\iffalse
\begin{center}
  \textbf{REFERÊNCIAS}
\end{center}


\fi



\end{document}

