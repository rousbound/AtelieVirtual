\documentclass[10pt]{article}
\usepackage[portuguese]{babel}
\usepackage[utf8]{inputenc}
\usepackage[pdftex]{graphicx}
\usepackage{venndiagram}
\usepackage{subcaption}
\usepackage{caption}
\usepackage[backend=biber,style=authoryear-ibid]{biblatex}
\usepackage[normalem]{ulem}
\usepackage[margin=0.8in]{geometry}
\addbibresource{psychoanalysis.bib}
\graphicspath{{Pictures/}}
\usepackage{tikz}
\usepackage{setspace}
\usepackage{enumitem}
\usepackage{textcomp}
\usepackage{hyperref}


\date{}

\newcommand{\quotebox}[3]{
  \begin{center}
\noindent\fbox{ 
  \parbox{#3\textwidth}{%
  {\itshape#1\itshape}

  \raggedleft {\textbf{#2}} 
    }%  
}
\end{center}
}

\newcommand{\spawnfig}[3]
{
  \begin{figure}[h]
  \centering
  \includegraphics[scale={#3}]{#1}
  \caption{#2}
  \end{figure}
}


\begin{document}
\maketitle
\begin{center}
  % Deputados CARLO CAIADO, GUSTAVO SCHMIDT, GIOVANI RATINHO, SUBTENENTE BERNARDO
  \huge
  \vspace{-3cm}\href{http://alerjln1.alerj.rj.gov.br/scpro1923.nsf/f4b46b3cdbba990083256cc900746cf6/2073d5bb816dad4e0325853c004feede?OpenDocument}{PROJETO DE LEI Nº 2222/2020}
\bigskip
\bigskip
\bigskip
  
\end{center}

\textbf{EMENTA:} 
DISPÕE SOBRE AÇÕES EMERGENCIAIS DE APOIO AO SETOR CULTURAL DO ESTADO DO RIO DE JANEIRO DURANTE A PANDEMIA DO NOVO CORONAVÍRUS - COVID-19 E DÁ OUTRAS PROVIDÊNCIAS.








\bigskip

\noindent
A ASSEMBLEIA LEGISLATIVA DO ESTADO DO RIO DE JANEIRO RESOLVE:

\begin{enumerate}[label=Art. \arabic*\textdegree]
\item - Esta Lei dispõe sobre ações emergenciais de apoio ao setor cultural do Estado do Rio de Janeiro durante a pandemia do novo coronarívus - COVID-19.

\item - Ficam suspensas as cobranças de contas dos estabelecimentos culturais referentes à prestação de serviços essenciais por empresas públicas ou privadas concessionárias do Estado enquanto perdurar a pandemia.

§ 1º - Entende-se como estabelecimentos culturais museus, teatros, cinemas, casas de espetáculos, shows, exposições, circos, casas de festas, ou qualquer outro estabelecimento que promova eventos com venda de ingresso ou entrada, excluídos restaurantes e bares que ofereçam aos clientes serviço de música ao vivo.

§ 2º - Entende-se como serviços essenciais água, luz e esgoto.

§ 3º - As cobranças suspensas pelo período desta Lei deverão ser quitadas com as respectivas concessionárias em 12 meses após o fim da pandemia.

\item - O Poder Executivo fica autorizado a postergar a cobrança impostos estaduais, sobretudo o ICMS, das empresas que promovam atividades culturais, podendo parcelar os débitos nos meses subsequentes ao fim da pandemia. 

\item - Fica autorizado o Poder Executivo a realizar os atos complementares necessários à execução da presente lei.

\item - Só serão beneficiados pela presente Lei os estabelecimentos culturais e empresas que promovam atividades culturais que comprovadamente não demitam funcionários enquanto as determinações do Poder Executivo a respeito do enfrentamento ao COVID-19 estiverem em vigor. 

\item - Esta Lei entra em vigor na data de sua publicação.

\end{enumerate}




\begin{center}
  Plenário Barbosa Lima Sobrinho, 31 de março de 2020.

   \bigskip

  \textbf{ CARLO CAIADO, GUSTAVO SCHMIDT, GIOVANI RATINHO, SUBTENENTE BERNARDO}

  \bigskip

  \textbf{JUSTIFICATIVA}
  \bigskip

\end{center}

  	O setor cultural do Rio de Janeiro está sendo um dos mais afetados pela epidemia do novo coronavírus - COVID-19, com o fechamento de museus, cinemas, teatros e afins desde o dia 15 de março deste ano.
	A presente proposta que apresentamos à esta Casa de Leis traz algumas medidas importantes para minimizar os prejuízos ao setor, como a suspensão da cobrança de água, luz e esgoto e de impostos estaduais enquanto perdurar a epidemia.
	Importante frisar que o benefícios só serão acessados por quem comprovadamente não demitir funcionários durante o período, de forma a se evitar o aumento do desemprego no nosso Estado.
	Face ao exposto é que solicitamos o apoio dos nobres colegas para que a presente proposta seja aprovada e vire Lei. 



\iffalse
\begin{center}
  \textbf{REFERÊNCIAS}
\end{center}


\fi



\end{document}

