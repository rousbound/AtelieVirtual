\documentclass[10pt]{article}
\usepackage[portuguese]{babel}
\usepackage[utf8]{inputenc}
\usepackage[pdftex]{graphicx}
\usepackage{venndiagram}
\usepackage{subcaption}
\usepackage{caption}
\usepackage[backend=biber,style=authoryear-ibid]{biblatex}
\usepackage[normalem]{ulem}
\usepackage[margin=0.8in]{geometry}
\addbibresource{psychoanalysis.bib}
\graphicspath{{Pictures/}}
\usepackage{tikz}
\usepackage{setspace}
\usepackage{enumitem}
\usepackage{textcomp}
\usepackage{hyperref}


\date{}

\newcommand{\quotebox}[3]{
  \begin{center}
\noindent\fbox{ 
  \parbox{#3\textwidth}{%
  {\itshape#1\itshape}

  \raggedleft {\textbf{#2}} 
    }%  
}
\end{center}
}

\newcommand{\spawnfig}[3]
{
  \begin{figure}[h]
  \centering
  \includegraphics[scale={#3}]{#1}
  \caption{#2}
  \end{figure}
}


\begin{document}
\maketitle
\begin{center}
  % Deputado ANDRÉ L. CECILIANO
  \huge
  \vspace{-3cm}\href{http://alerjln1.alerj.rj.gov.br/scpro1923.nsf/f4b46b3cdbba990083256cc900746cf6/cdfbf95a3c8cfd40832584030069ae4b?OpenDocument}{PROJETO DE LEI Nº 620/2019}
\bigskip
\bigskip
\bigskip
  
\end{center}

\textbf{EMENTA:} 
AUTORIZA O PODER EXECUTIVO A EFETUAR O PAGAMENTO DA OPERAÇÃO DE CRÉDITO DE QUE TRATA A LEI ESTADUAL N 7529, DE 07 DE MARÇO DE 2017, COM OS RECURSOS QUE ESPECIFICA.









\bigskip

\noindent
A ASSEMBLEIA LEGISLATIVA DO ESTADO DO RIO DE JANEIRO RESOLVE:

\begin{enumerate}[label=Art. \arabic*\textdegree]
\item - Fica autorizado o Poder Executivo a utilizar créditos a título de royalties, royalties excedentes e participação especial, decorrentes da atividade de exploração e produção de petróleo e gás natural, a que o estado faz jus a receber no exercício de 2019 e/ou 2020, por força do art. 20, § 1º da Constituição Federal e da Lei nº 7.990/89, com suas alterações, para pagamento do empréstimo de que trata a Lei Estadual nº 7529, de 07 de março de 2017.

Parágrafo Único: A utilização de que trata o caput deste artigo deverá se limitar a parcela excedente ao valor estimado de arrecadação no exercício financeiro de 2019 e/ou 2020.


\item - Fica, ainda, autorizada a antecipação de recursos de que trata o artigo 1º desta Lei em quantia exata para o pagamento do empréstimo de que trata a Lei Estadual nº 7529, de 07 de março de 2017, eventuais juros e correção.

\item - Esta Lei entra em vigor na data de sua publicação.

\end{enumerate}




\begin{center}
  Plenário Barbosa Lima Sobrinho, 23 de maio de 2019

   \bigskip

  \textbf{ ANDRÉ L. CECILIANO}

  \bigskip

  \textbf{JUSTIFICATIVA}
  \bigskip

\end{center}

  O projeto pretende autorizar o pagamento do empréstimo de trata a Lei nº 7529 de 07 de março de 2017, que autorizou a alienação da Companhia Estadual de Águas e Esgotos, a CEDAE, com o valor excedente dos royalties referentes ao exercício de 2019 e 2020.
 
A arrecadação do estado com royalties de petróleo e participações especiais teve um expressivo aumento em comparação ao mesmo quadrimestre do ano anterior. A estimativa é de uma elevação de mais de 100\%, saindo de menos de três bilhões de reais em abril de 2018 para mais de seis bilhões em 2019, o que justifica o emprego desses recursos no pagamento da parcela do empréstimo. E de acordo com a Agencia Nacional do Petróleo, a ANP, a arrecadação será crescente nos próximos cinco anos saindo de cerca de cinco bilhões de reais para oito bilhões.
 
A Alerj fez um grande esforço para reverter a alienação das ações da Cedae por sua importância para população fluminense. E com essa proposta busca garantir os recursos necessários para solucionar a questão.



\iffalse
\begin{center}
  \textbf{REFERÊNCIAS}
\end{center}


\fi



\end{document}

