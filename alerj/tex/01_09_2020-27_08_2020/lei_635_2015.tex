\documentclass[10pt]{article}
\usepackage[portuguese]{babel}
\usepackage[utf8]{inputenc}
\usepackage[pdftex]{graphicx}
\usepackage{venndiagram}
\usepackage{subcaption}
\usepackage{caption}
\usepackage[backend=biber,style=authoryear-ibid]{biblatex}
\usepackage[normalem]{ulem}
\usepackage[margin=0.8in]{geometry}
\addbibresource{psychoanalysis.bib}
\graphicspath{{Pictures/}}
\usepackage{tikz}
\usepackage{setspace}
\usepackage{enumitem}
\usepackage{textcomp}
\usepackage{hyperref}


\date{}

\newcommand{\quotebox}[3]{
  \begin{center}
\noindent\fbox{ 
  \parbox{#3\textwidth}{%
  {\itshape#1\itshape}

  \raggedleft {\textbf{#2}} 
    }%  
}
\end{center}
}

\newcommand{\spawnfig}[3]
{
  \begin{figure}[h]
  \centering
  \includegraphics[scale={#3}]{#1}
  \caption{#2}
  \end{figure}
}


\begin{document}
\maketitle
\begin{center}
  % Deputado ÁTILA NUNES
  \huge
  \vspace{-3cm}\href{http://alerjln1.alerj.rj.gov.br/scpro1519.nsf/f4b46b3cdbba990083256cc900746cf6/2610e42fb729daa583257e7d00675a7c?OpenDocument}{PROJETO DE LEI Nº 635/2015}
\bigskip
\bigskip
\bigskip
  
\end{center}

\textbf{EMENTA:} 
ALTERA A LEI Nº 5.645, DE 06 DE JANEIRO DE 2010, INSTITUINDO, NO ÂMBITO DO ESTADO DO RIO DE JANEIRO, A &#``SEMANA ESTADUAL DE CONSCIENTIZAÇÃO AO USO DO TRANSPORTE COLETIVO E MEIOS DE TRANSPORTES ALTERNATIVOS".








\bigskip

\noindent
A ASSEMBLEIA LEGISLATIVA DO ESTADO DO RIO DE JANEIRO RESOLVE:

\begin{enumerate}[label=Art. \arabic*\textdegree]

\item - Fica instituída no âmbito do Estado do Rio de Janeiro a &#``Semana Estadual de Conscientização ao Uso do Transporte Coletivo e Meios de transporte Alternativos", a ser desenvolvida anualmente na semana do mês de setembro em que constar o dia 22 (vinte e dois), data em que fica instituído o "Dia Respire Melhor sem Carro".
 § 1º - A Semana de que trata o caput deste artigo destina-se à realização de campanhas para incentivar o uso ao transporte coletivo e meio de transportes alternativos em promoção ao "Dia Mundial sem Carro" (22). 
§ 2º - No decorrer da semana serão desenvolvidas ações educativas tais como palestras, debates e seminários nos diversos segmentos da sociedade, bem como ampla divulgação na mídia, cartazes e distribuição de folhetos aos motoristas nas vias públicas, para incentivar o uso de bicicletas, transporte público e outros meios alternativos de transporte, objetivando a preservação do meio ambiente.
§ 3º - As Escolas incluirão através de um programa específico, o tema sobre o transporte individual e seus impactos a ser ministrado na Semana ora instituída, mostrando as consequências do agravamento da poluição do ar, as doenças causadas por esta poluição, estatísticas com os índices de acidentes de trânsito e a falta de democratização do espaço público em decorrência da abertura ininterrupta de vias para o carro. 

\item - O Poder Público poderá firmar convênios ou parcerias com instituições de ensino, associações e entidades da sociedade civil e órgãos do poder público  para realização destes atos.

\item - O Anexo da Lei nº 5.645, de 06 de janeiro de 2010, passa a vigorar com a seguinte redação:

              CALENDÁRIO DATAS COMEMORATIVAS DO ESTADO DO RIO DE JANEIRO:

(...)

            SETEMBRO

             (...)

DIA 22 - Dia "Respire Melhor sem Carro".

         SEMANA DO DIA 22 - Semana Estadual de Conscientização ao Uso do Transporte Coletivo e Meios de transporte Alternativos.

(...)


               Art. 4º - As despesas decorrentes da aplicação desta lei correrão por conta de dotações orçamentárias próprias, suplementadas se necessário.


\end{enumerate}




\begin{center}
  Plenário Barbosa Lima Sobrinho, 04 de agosto de 2015.

   \bigskip

  \textbf{ ÁTILA NUNES}

  \bigskip

  \textbf{JUSTIFICATIVA}
  \bigskip

\end{center}

  Países europeus, da América Latina e mesmo os Estados Unidos, têm ampliado o transporte público e o transporte alternativo, desenvolvendo políticas claras de retirar os subsídios públicos e implantar medidas de restrição explicita ao uso dos automóveis em áreas congestionadas e centrais. Dessa forma, tem-se ampliado a participação do transporte público e do não motorizado em relação ao automóvel.    Os grandes centros urbanos brasileiros têm problemas gravíssimos de trânsito, devido a opção do uso irrestrito de automóveis, situação que fica a cada dia mais grave, com consequências iminentes para o meio ambiente e para a vida nas cidades.
É chegado o momento de se refletir sobre alternativas ao uso do transporte individual motorizado, como a bicicleta e o transporte coletivo, principalmente nos centros das cidades, pois essas modalidades podem contribuir muito para a redução de carros nas ruas e para o meio ambiente.  Diante do exposto, peço o apoio dos meus ilustres pares para aprovação do presente projeto de lei, que tem por finalidade a criação do Dia "Respire melhor sem carro", unificando tal comemoração e campanha com o "Dia Mundial sem Carro" a ser comemorado, anualmente, no dia 22 de setembro, proporcionando reflexão e permitindo a toda sociedade e estudantes do Estado do Rio de Janeiro perceber a urgência de mudança na política de transporte individual, disciplinando o uso dos automóveis.

Assim justifica-se este projeto visando conscientizar a população sobre os danos ambientais e a saúde causados pelo uso intenso de automóveis e incentivar o uso de meios de transportes sustentáveis, entre os quais se destaca a bicicleta, ou ainda o uso de transporte coletivo que reduz esse impacto negativo no meio ambiente e proporciona uma melhoria na fluência do nosso caótico trânsito.   Diante disto, conto com o apoio dos meus nobres pares para a aprovação da presente proposição.




\iffalse
\begin{center}
  \textbf{REFERÊNCIAS}
\end{center}


\fi



\end{document}

