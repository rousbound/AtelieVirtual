\documentclass[10pt]{article}
\usepackage[portuguese]{babel}
\usepackage[utf8]{inputenc}
\usepackage[pdftex]{graphicx}
\usepackage{venndiagram}
\usepackage{subcaption}
\usepackage{caption}
\usepackage[backend=biber,style=authoryear-ibid]{biblatex}
\usepackage[normalem]{ulem}
\usepackage[margin=0.8in]{geometry}
\addbibresource{psychoanalysis.bib}
\graphicspath{{Pictures/}}
\usepackage{tikz}
\usepackage{setspace}
\usepackage{enumitem}
\usepackage{textcomp}
\usepackage{hyperref}


\date{}

\newcommand{\quotebox}[3]{
  \begin{center}
\noindent\fbox{ 
  \parbox{#3\textwidth}{%
  {\itshape#1\itshape}

  \raggedleft {\textbf{#2}} 
    }%  
}
\end{center}
}

\newcommand{\spawnfig}[3]
{
  \begin{figure}[h]
  \centering
  \includegraphics[scale={#3}]{#1}
  \caption{#2}
  \end{figure}
}


\begin{document}
\maketitle
\begin{center}
  % Deputados FLÁVIO SERAFINI, ANDRÉ CECILIANO
  \huge
  \vspace{-3cm}\href{http://alerjln1.alerj.rj.gov.br/scpro1923.nsf/f4b46b3cdbba990083256cc900746cf6/1d293292b140e998832584a3005c1ac0?OpenDocument}{PROJETO DE LEI Nº 1537/2019}
\bigskip
\bigskip
\bigskip
  
\end{center}

\textbf{EMENTA:} 
ESTABELECE A OBRIGATORIEDADE DA CHAMADA PÚBLICA E BUSCA ATIVA DE CRIANÇAS, ADOLESCENTES, JOVENS E ADULTOS NA REDE ESTADUAL DE ENSINO DO RIO DE JANEIRO E DÁ OUTRAS PROVIDÊNCIAS.









\bigskip

\noindent
A ASSEMBLEIA LEGISLATIVA DO ESTADO DO RIO DE JANEIRO RESOLVE:

\begin{enumerate}[label=Art. \arabic*\textdegree]
\item - Torna obrigatório o processo de chamada pública de crianças, adolescentes, jovens e adultos nas escolas da rede estadual de ensino do Rio de Janeiro. 

§1º - Para os fins desta lei e em consonância com o Artigo 5º da Lei Federal nº 9.394/1996, entende-se por chamada pública a ampla divulgação, em todos os grandes veículos de comunicação, meios de comunicação oficial, nas páginas da internet e nas redes sociais do governo do Estado do Rio de Janeiro e da Secretaria de Estado de Educação e em cada unidade escolar da rede estadual de ensino do Rio de Janeiro, de informações referentes à oferta do ensino fundamental, médio e educação de jovens e adultos, bem como o período de matrícula de cada etapa e modalidade de ensino.

\item - O processo de chamada pública deverá começar 30 dias antes do início do período de matrículas nas unidades escolares, estendendo-se até o seu término. 

§ 1º A chamada pública deverá orientar a população a procurar as escolas públicas e os canais de informação da Secretaria de Estado de Educação para obtenção de informações sobre os procedimentos de matrícula na rede estadual de ensino.

\item - A Secretaria de Estado de Educação promoverá parcerias com outras instituições e/ou órgãos públicos para realizar a busca ativa por crianças e adolescentes em idade escolar obrigatória que se encontrarem fora da escola e proceder o imediato ingresso na rede estadual de ensino.

\item - As unidades da rede estadual de ensino, devidamente apoiadas com suporte técnico e profissional da SEEDUC, deverão monitorar permanentemente a freqüência  dos estudantes, buscando contatar as famílias e, se necessário o Conselho Tutelar e o Juizado da Infância e Adolescência.

Parágrafo Único - Anualmente, as unidades escolares deverão fazer uma análise de seus casos de infrequência e evasão, sinalizando as principais causas diagnosticadas e construindo sugestões de enfrentamento ao problema para apresentar à SEEDUC.

\item - O Poder Executivo Estadual regulamentará esta Lei naquilo que lhe couber.

\item - As eventuais despesas decorrentes da presente Lei correrão por conta de dotações orçamentárias próprias. 

\item - Esta lei entra em vigor na data de sua publicação.


\end{enumerate}




\begin{center}
  Plenário Barbosa Lima Sobrinho, 30 de outubro de 2019.

   \bigskip

  \textbf{ FLÁVIO SERAFINI, ANDRÉ CECILIANO}

  \bigskip

  \textbf{JUSTIFICATIVA}
  \bigskip

\end{center}

  A universalização do acesso à educação básica, especialmente ao Ensino Médio, ainda é uma tarefa posta à Coisa Pública em suas diferentes esferas. É perceptível, ao longo da escolaridade, a existência de &#``funis&#" entre cada etapa de ensino com o ingresso nos níveis mais elevados de um número menor de estudantes do que os concluintes na etapa imediatamente anterior.
Em paralelo, aqueles que não estão inseridos em nenhuma rede de ensino, aliada à ausência de procedimentos administrativos eficientes e transparentes que permitam à população demandar e acessar vagas no ensino público, enfrentam dificuldades adicionais que podem impedi-los de conhecer seus direitos, suas responsabilidades, bem como os instrumentos para acessá-los.
A Lei de Diretrizes e Bases da Educação, Lei 9.394 de 1996, em seu artigo 5º, inciso II, coloca como dever do Poder Público, na esfera de sua competência, promover dentre outros mecanismos, a chamada pública. 
 O artigo 5º define:

		Art. 5º - O acesso ao ensino fundamental é direito público subjetivo, podendo qualquer cidadão, grupo de cidadãos, associação comunitária, organização sindical, entidade de classe ou outra legalmente constituída, e, ainda, o Ministério Público, acionar o Poder 			Público para exigi-lo.
		§ 1º. Compete aos Estados e aos Municípios, em regime de colaboração, e com a assistência da União:
		I - recensear a população em idade escolar para o ensino fundamental, e os jovens e adultos que a ele não tiveram acesso;
		II - fazer-lhes a chamada pública;
		III - zelar, junto aos pais ou responsáveis, pela frequência à escola.

Nessa perspectiva, a presente proposta tem por objetivo contribuir com o acesso e permanência de todas as pessoas, independentemente da idade, ao ensino fundamental, médio e educação de jovens e adultos na rede estadual de ensino do Rio de Janeiro, entendendo a Educação, como um direito de síntese por potencializar o exercício dos demais direitos, pela totalidade da população fluminense.
Por todo o exposto, coloco o presente projeto de lei à apreciação dos nobres pares, contando com a sua aprovação.



\iffalse
\begin{center}
  \textbf{REFERÊNCIAS}
\end{center}


\fi



\end{document}

