\documentclass[10pt]{article}
\usepackage[portuguese]{babel}
\usepackage[utf8]{inputenc}
\usepackage[pdftex]{graphicx}
\usepackage{venndiagram}
\usepackage{subcaption}
\usepackage{caption}
\usepackage[backend=biber,style=authoryear-ibid]{biblatex}
\usepackage[normalem]{ulem}
\usepackage[margin=0.8in]{geometry}
\addbibresource{psychoanalysis.bib}
\graphicspath{{Pictures/}}
\usepackage{tikz}
\usepackage{setspace}
\usepackage{enumitem}
\usepackage{textcomp}
\usepackage{hyperref}


\date{}

\newcommand{\quotebox}[3]{
  \begin{center}
\noindent\fbox{ 
  \parbox{#3\textwidth}{%
  {\itshape#1\itshape}

  \raggedleft {\textbf{#2}} 
    }%  
}
\end{center}
}

\newcommand{\spawnfig}[3]
{
  \begin{figure}[h]
  \centering
  \includegraphics[scale={#3}]{#1}
  \caption{#2}
  \end{figure}
}


\begin{document}
\maketitle
\begin{center}
  % Deputado BAGUEIRA
  \huge
  \vspace{-3cm}\href{http://alerjln1.alerj.rj.gov.br/scpro1923.nsf/f4b46b3cdbba990083256cc900746cf6/145acc86e866ec24832584a800691e73?OpenDocument}{PROJETO DE LEI Nº 1566/2019}
\bigskip
\bigskip
\bigskip
  
\end{center}

\textbf{EMENTA:} 

&#`` ALTERA A LEI Nº 5.645 , DE 06 DE JANEIRO  DE 2010, PARA INCLUIR  NO  CALENDÁRIO OFICIAL DO ESTADO DO RIO DE JANEIRO, O DIA  DA  &#``CONSTITUIÇÃO FEDERAL NO ESTADO DO RIO DE JANEIRO,&#",  A SER COMEMORADO  NO DIA 05 DE OUTUBRO &#".








\bigskip

\noindent
A ASSEMBLEIA LEGISLATIVA DO ESTADO DO RIO DE JANEIRO RESOLVE:

\begin{enumerate}[label=Art. \arabic*\textdegree]
\item - Fica  incluído  no Anexo da Lei nº 5.645, de 06 de Janeiro  de 2010, que consolida a legislação relativa às  datas comemorativas  do   Estado do Rio de Janeiro,  o  &#``  Dia da Constituição Federal no Estado do Rio de Janeiro &#`` , a ser comemorado o dia 05 de Outubro.  .
\item - O  Anexo da Lei nº  5.645, de 06 de janeiro de 2010, passa a vigorar com as seguinte redação : 
CALENDÁRIO DE DATAS COMEMORATIVAS DO ESTADO DO RIO DE JANEIRO
                                                   ( ... )
                                                OUTUBRO
                                                   ( ... )
Dia 05  DE OUTUBRO - &#`` Dia da Constituição Federal no  Estado do Rio de Janeiro &#`` .

\item - Esta lei entrará em vigor na  data da sua publicação .


\end{enumerate}




\begin{center}
  Plenário Barbosa Lima Sobrinho, 05  de Novembro de 2019

   \bigskip

  \textbf{ BAGUEIRA}

  \bigskip

  \textbf{JUSTIFICATIVA}
  \bigskip

\end{center}

  No dia  05 de Outubro a  nossa Constituição Federal faz aniversário. Ela, é carinhosamente chamada de &#`` Constituição Cidadã &#``. 
Recebeu este título do então Presidente da Câmara, Deputado Ulysses Guimarães , porque ampliou as garantias e liberdades dos cidadãos, restabelecendo as eleições livres e diretas, dando um fim a censura e permitiu o voto do analfabeto .
O Brasil já teve sete Constituições, desde sua independência,   no dia 07 de Setembro de 1822.
 Entre outras medidas, ampliou  os  poderes do Congresso Nacional, tornando o Brasil mais democrático. E, por isso, o mesmo foi registrado na nossa Constituição Estadual, promulgada pelo então Deputado Gilberto Rodrigues, que editou vários exemplares em cor azul, com os nomes do  Constituinte na capa.
 Além dos parlamentares, os jornalistas credenciados no Comitê de Imprensa receberam a Constituição com seu nome   também na capa. Do antigo Estado do Rio de Janeiro, apenas dois jornalistas foram homenageados : Continentino Porto e Rogério Coelho Neto.
Diante dessas argumentações, solicitamos aos nobres parlamentares a aprovação desta matéria .



\iffalse
\begin{center}
  \textbf{REFERÊNCIAS}
\end{center}


\fi



\end{document}

