\documentclass[10pt]{article}
\usepackage[portuguese]{babel}
\usepackage[utf8]{inputenc}
\usepackage[pdftex]{graphicx}
\usepackage{venndiagram}
\usepackage{subcaption}
\usepackage{caption}
\usepackage[backend=biber,style=authoryear-ibid]{biblatex}
\usepackage[normalem]{ulem}
\usepackage[margin=0.8in]{geometry}
\addbibresource{psychoanalysis.bib}
\graphicspath{{Pictures/}}
\usepackage{tikz}
\usepackage{setspace}
\usepackage{enumitem}
\usepackage{textcomp}
\usepackage{hyperref}


\date{}

\newcommand{\quotebox}[3]{
  \begin{center}
\noindent\fbox{ 
  \parbox{#3\textwidth}{%
  {\itshape#1\itshape}

  \raggedleft {\textbf{#2}} 
    }%  
}
\end{center}
}

\newcommand{\spawnfig}[3]
{
  \begin{figure}[h]
  \centering
  \includegraphics[scale={#3}]{#1}
  \caption{#2}
  \end{figure}
}


\begin{document}
\maketitle
\begin{center}
  % Deputado ROSENVERG REIS
  \huge
  \vspace{-3cm}\href{http://alerjln1.alerj.rj.gov.br/scpro1923.nsf/f4b46b3cdbba990083256cc900746cf6/5beb39e2577b54738325844e004db19d?OpenDocument}{PROJETO DE LEI Nº 990/2019}
\bigskip
\bigskip
\bigskip
  
\end{center}

\textbf{EMENTA:} 
INCLUI NO ANEXO DA CONSOLIDAÇÃO DE DATAS COMEMORATIVAS DO ESTADO DO RIO DE JANEIRO, O DIA ESTADUAL  DA OPERAÇÃO SALOMÃO COMEMORADO, ANUALMENTE, NO DIA 24 DE MAIO. 








\bigskip

\noindent
A ASSEMBLEIA LEGISLATIVA DO ESTADO DO RIO DE JANEIRO RESOLVE:

\begin{enumerate}[label=Art. \arabic*\textdegree]
\item - Fica incluído no anexo da Lei Estadual nº 5645, de 06 de janeiro de 2010, que consolida a legislação das datas comemorativas do Calendário Oficial do Estado do Rio de Janeiro, o &#``Dia Estadual da Operação Salomão&#", a ser comemorado, anualmente, no dia 24 de Maio.

\item - O anexo da Lei nº 5645/2010, passa a ter a seguinte redação:

&#``ANEXO

CALENDÁRIO DATAS COMEMORATIVAS DO ESTADO DO RIO DE JANEIRO:

(...)
MAIO
(...)

24 DE MAIO - Dia do Metodismo Wesleyano. Lei nº 6147/2012. 
24 DE MAIO - DIA ESTADUAL DA OPERAÇÃO SALOMÃO.

(...) NR&#"



\item - Esta Lei entrará em vigor na data de sua publicação.



\end{enumerate}




\begin{center}
  Plenário Barbosa Lima Sobrinho, 06 de agosto de 2019.

   \bigskip

  \textbf{ ROSENVERG REIS}

  \bigskip

  \textbf{JUSTIFICATIVA}
  \bigskip

\end{center}

  A proposta objetiva criar no calendário do Estado do Rio de Janeiro, o Dia Estadual da Operação Salomão, a ser comemorado, anualmente, no dia 24 de Maio.

Em 1991, entre os dias 23 e 24 de maio, foi concluída a Operação Salomão, em 36 horas,  os Falashas (uma comunidade judaica) que viviam na Etiópia e que estavam sendo perseguidos, foram resgatados por Israel.

No fim dos anos 1980, essa comunidade judaica estava em sério risco, em razão da guerra civil que ocorria na Etiópia. E, após negociação, conduzida principalmente pelo Embaixador de Israel na Etiópia, o ditador  Mengistu Haile Mariam, concordou com a saída dos Falashas para Israel.


A operação de resgate foi bem sucedida. Nada menos que 14325 Falashas foram resgatados em segurança.




\iffalse
\begin{center}
  \textbf{REFERÊNCIAS}
\end{center}


\fi



\end{document}

