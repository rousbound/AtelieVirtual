\documentclass[10pt]{article}
\usepackage[portuguese]{babel}
\usepackage[utf8]{inputenc}
\usepackage[pdftex]{graphicx}
\usepackage{venndiagram}
\usepackage{subcaption}
\usepackage{caption}
\usepackage[backend=biber,style=authoryear-ibid]{biblatex}
\usepackage[normalem]{ulem}
\usepackage[margin=0.8in]{geometry}
\addbibresource{psychoanalysis.bib}
\graphicspath{{Pictures/}}
\usepackage{tikz}
\usepackage{setspace}
\usepackage{enumitem}
\usepackage{textcomp}
\usepackage{hyperref}


\date{}

\newcommand{\quotebox}[3]{
  \begin{center}
\noindent\fbox{ 
  \parbox{#3\textwidth}{%
  {\itshape#1\itshape}

  \raggedleft {\textbf{#2}} 
    }%  
}
\end{center}
}

\newcommand{\spawnfig}[3]
{
  \begin{figure}[h]
  \centering
  \includegraphics[scale={#3}]{#1}
  \caption{#2}
  \end{figure}
}


\begin{document}
\maketitle
\begin{center}
  % Deputado ÁTILA NUNES
  \huge
  \vspace{-3cm}\href{http://alerjln1.alerj.rj.gov.br/scpro1519.nsf/f4b46b3cdbba990083256cc900746cf6/a09a888519ff3d1c8325830700521f83?OpenDocument}{PROJETO DE LEI Nº 4404/2018}
\bigskip
\bigskip
\bigskip
  
\end{center}

\textbf{EMENTA:} 
ALTERA A LEI Nº 5.645, DE 06 DE JANEIRO DE 2010, PARA INSTITUIR NO CALENDÁRIO OFICIAL DO ESTADO DO RIO DE JANEIRO O DIA ESTADUAL DO TERAPEUTA HOLÍSTICO, NO ÂMBITO DO ESTADO DO RIO DE JANEIRO.








\bigskip

\noindent
A ASSEMBLEIA LEGISLATIVA DO ESTADO DO RIO DE JANEIRO RESOLVE:

\begin{enumerate}[label=Art. \arabic*\textdegree]

\item - Fica instituída no Estado do Rio de Janeiro o &#``Dia Estadual do Terapeuta Holístico&#", que se realizará anualmente, no dia 31 de março, fazendo menção ao Dia Nacional da Terapia Holística.  

\item - O Dia Estadual do Terapeuta Holístico deverá ser comemorado anualmente durante todo o mês de março, com o objetivo de mostrar à importância deste profissional.

\item - O Dia Estadual do Terapeuta Holístico deverá ser marcado com caminhadas, palestras, simpósios, distribuição de informativos e campanhas na mídia.

\item - Fica o Poder Executivo autorizado a firmar convênios não onerosos com instituições públicas e particulares, para que sejam elaboradas campanhas publicitárias de divulgação, esclarecimentos e difusão sobre o Dia Estadual do Terapeuta Holístico, bem como a utilização de iluminação e decorações em monumentos e logradouros públicos, em especial os de relevante importância e grande fluxo de pessoas em todo o Estado do Rio de Janeiro.

\item - As despesas decorrentes da aplicação desta lei correrão por conta de dotações orçamentárias próprias para este fim, suplementadas se necessárias.

\item - O Anexo da Lei nº 5645, de 06 de Janeiro de 2010 passa a vigorar com a seguinte redação:


CALENDÁRIO DE DATAS COMEMORATIVAS DO ESTADO DO RIO DE JANEIRO:

(&#8230;)

MARÇO

(&#8230;)

DIA 31 - DIA ESTADUAL DO TERAPEUTA HOLÍSTICO.

(...)


\item - Esta Lei entrará em vigor na data de sua publicação. 


\end{enumerate}




\begin{center}
  Plenário Barbosa Lima Sobrinho, 11 de Setembro de 2018

   \bigskip

  \textbf{ ÁTILA NUNES}

  \bigskip

  \textbf{JUSTIFICATIVA}
  \bigskip

\end{center}

  Este Projeto de Lei tem como objetivo instituir o dia 31 de março como o Dia Estadual da Terapia Holística.  Através da Terapia Holística a vida das pessoas pode tornar-se mais saudável pois utiliza-se uma somatória de técnicas milenares e modernas, sempre suaves e naturais, proporcionando harmonia, autoconhecimento e incrementando a capacidade da pessoa tratada. 

Dentre estas técnicas podemos citar Yoga, Reiki, Tai Chi Chuan, Acupuntura, Aromaterapia, Homeopatia, Fitoterapia, Cromoterapia, Cristaloterapia, Xamamismo, e outras terapias alternativas que ajudam a combater doenças de maneira eficaz e barata. As popularmente chamadas de "terapias alternativas" são aplicadas pelo Terapeuta Holístico, que procede ao estudo e à análise do cliente, realizados sempre sob o paradigma holístico, cuja abordagem leva em consideração os aspectos sócio-somato-psíquicos.

Cada caso é considerado único e deve-se dispor dos mais variados métodos, para possibilitar a opção por aqueles com os quais o cliente tenha maior afinidade, promovendo a otimização da qualidade de vida, estabelecendo um processo interativo com seu cliente, levando este ao autoconhecimento e a mudanças em várias áreas, sendo as mais comuns: comportamento, elaboração da realidade e/ou preocupações com a mesma, incremento na capacidade de ser bem-sucedido nas situações da vida (aumento máximo das oportunidades e minimização das condições adversas), além de conhecimento e habilidade para tomada de decisão. 

Avalia os desequilíbrios energéticos, suas predisposições e possíveis consequências, além de promover a catalisação da tendência natural ao auto equilíbrio, facilitando-a pela aplicação de uma somatória de terapêuticas de abordagem holística, com o objetivo de transmutar a desarmonia em autoconhecimento.

A Organização Mundial da Saúde reconhece a importância da fé e da religiosidade no desenvolvimento do processo de cura. Considerando que o homem é corpo físico e espiritual, nenhum processo de cura pode hoje se dissociar de contemplar o homem como um ser múltiplo, devendo ser respeitados diversos aspectos ao proceder-se um tratamento de saúde.

Sabe-se também que a manifestação de uma desarmonia em todo esse complexo, consubstanciada no que chamamos doença, é, na maioria das vezes, a somatização, no físico, de um processo desarmônico em alguma parte do todo que é o homem.

A escolha do dia 31 de março como o Dia do Terapeuta Holístico é em homenagem a São Benedito. São Benedito nasceu em 1526, em São Filadelfo, nos arredores de Messina, era filho de pais descendentes de escravos levados para a Sicília. Manifestou desde os 10 anos uma pronunciada tolerância para a penitência e para a solidão. Foi chamado de &#8216;Santo Mouro&#8217; por causa de sua cor preta e aos 18 anos, com o fruto de seu trabalho, provia o seu sustento e dos pobres e operou diversos milagres.

Além de tido, a oficialização do Dia Estadual do Terapeuta Holístico será o reconhecimento e a homenagem merecida aos abnegados profissionais que emprestam seus dons e conhecimentos pessoais em prol da comunidade, buscando melhorar o ser humano e o universo.

Dessa forma, solicito o apoio dos meus pares para que o presente Projeto de Lei seja aprovado.



\iffalse
\begin{center}
  \textbf{REFERÊNCIAS}
\end{center}


\fi



\end{document}

