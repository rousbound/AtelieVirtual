\documentclass[10pt]{article}
\usepackage[portuguese]{babel}
\usepackage[utf8]{inputenc}
\usepackage[pdftex]{graphicx}
\usepackage{venndiagram}
\usepackage{subcaption}
\usepackage{caption}
\usepackage[backend=biber,style=authoryear-ibid]{biblatex}
\usepackage[normalem]{ulem}
\usepackage[margin=0.8in]{geometry}
\addbibresource{psychoanalysis.bib}
\graphicspath{{Pictures/}}
\usepackage{tikz}
\usepackage{setspace}
\usepackage{enumitem}
\usepackage{textcomp}
\usepackage{hyperref}


\date{}

\newcommand{\quotebox}[3]{
  \begin{center}
\noindent\fbox{ 
  \parbox{#3\textwidth}{%
  {\itshape#1\itshape}

  \raggedleft {\textbf{#2}} 
    }%  
}
\end{center}
}

\newcommand{\spawnfig}[3]
{
  \begin{figure}[h]
  \centering
  \includegraphics[scale={#3}]{#1}
  \caption{#2}
  \end{figure}
}


\begin{document}
\maketitle
\begin{center}
  % Deputado PAULO RAMOS
  \huge
  \vspace{-3cm}\href{http://alerjln1.alerj.rj.gov.br/scpro1519.nsf/f4b46b3cdbba990083256cc900746cf6/b3a9666913b9d547832582a300644571?OpenDocument}{PROJETO DE LEI Nº 4166/2018}
\bigskip
\bigskip
\bigskip
  
\end{center}

\textbf{EMENTA:} 
ALTERA A LEI Nº 5645, DE 6 DE JANEIRO DE 2010, INCLUINDO, NO CALENDÁRIO OFICIAL DO ESTADO DO RIO DE JANEIRO, O "DIA ESTADUAL CONTRA A EXPORTAÇÃO DE GADO VIVO" NO ESTADO DO RIO DE JANEIRO.








\bigskip

\noindent
A ASSEMBLEIA LEGISLATIVA DO ESTADO DO RIO DE JANEIRO RESOLVE:

\begin{enumerate}[label=Art. \arabic*\textdegree]

\item - Fica incluído, no anexo da Lei nº 5.645, de 6 de janeiro de 2010, que consolida a legislação relativa às datas comemorativas do Calendário Oficial do Estado do Rio de Janeiro, o &#``Dia Estadual Contra a Exportação de Gado Vivo" no Estado do Rio de Janeiro, a ser comemorado, anualmente, no dia 14 de Junho.

\item - O Anexo da Lei nº 5.645, de 2010, passa a vigorar com a seguinte redação:

"ANEXO

CALENDÁRIO DATAS COMEMORATIVAS DO ESTADO DO RIO DE JANEIRO
(&#8230;)
JUNHO
(&#8230;)
14 - "DIA ESTADUAL CONTRA A EXPORTAÇÃO DE GADO VIVO"


\item - Esta Lei entra em vigor na data de sua publicação.


\end{enumerate}




\begin{center}
  Plenário Barbosa Lima Sobrinho, 05 de junho de 2018

   \bigskip

  \textbf{ PAULO RAMOS}

  \bigskip

  \textbf{JUSTIFICATIVA}
  \bigskip

\end{center}

  O Brasil exporta anualmente milhares de bovinos para abate no Oriente Médio, em viagens de semanas, condições muitíssimo precárias, alta densidade de lotação e insuficiente assistência veterinária. Se mortos durante o trajeto, são atirados ao mar, assim como toneladas de dejetos produzidos na viagem. Além de impacto ambiental, risco de acidentes é alarmante. Em 2012, ventilação de navio parou de funcionar em alto mar e 2.750 bovinos morreram; em 2015, navio com 5.000 afundou em porto do Pará. 
O movimento cresce globalmente. Recentemente, milhares de pessoas já protestaram  em Israel e Portugal. O Fórum Animal e a Animals International atuam desde 2016 no Brasil, onde esse comércio era quase desconhecido. Neste ano, Ação Civil Pública do Fórum Animal conseguiu laudo técnico veterinário (Dra. Magda Regina) e suspensão da exportação em nível federal, até intervenção da Advocacia Geral da União. Fórum Animal espera agora resultado de seu Recurso na Justiça Federal para restabelecer a proibição federal da exportação. 
Projetos de Lei contra a exportação foram apresentados no Congresso, na Assembleia do Estado de São Paulo, na do Estado do Rio de Janeiro, e na Prefeitura de Santos.
Em 2018, haverá uma integração no movimento global com 13 cidades: Rio de Janeiro, São Paulo, Santos, Belo Horizonte, Salvador, Curitiba, Paranaguá, Florianópolis, Porto Alegre, São Luiz, Belém, Brasília, Manaus, para acabar com a exportação do gado vivo. 
A proposta é criar um dia estadual, unindo ao mundial, para combate à exportação de gado vivo e contra os maus-tratos de animais.



\iffalse
\begin{center}
  \textbf{REFERÊNCIAS}
\end{center}


\fi



\end{document}

