\documentclass[10pt]{article}
\usepackage[portuguese]{babel}
\usepackage[utf8]{inputenc}
\usepackage[pdftex]{graphicx}
\usepackage{venndiagram}
\usepackage{subcaption}
\usepackage{caption}
\usepackage[backend=biber,style=authoryear-ibid]{biblatex}
\usepackage[normalem]{ulem}
\usepackage[margin=0.8in]{geometry}
\addbibresource{psychoanalysis.bib}
\graphicspath{{Pictures/}}
\usepackage{tikz}
\usepackage{setspace}
\usepackage{enumitem}
\usepackage{textcomp}
\usepackage{hyperref}


\date{}

\newcommand{\quotebox}[3]{
  \begin{center}
\noindent\fbox{ 
  \parbox{#3\textwidth}{%
  {\itshape#1\itshape}

  \raggedleft {\textbf{#2}} 
    }%  
}
\end{center}
}

\newcommand{\spawnfig}[3]
{
  \begin{figure}[h]
  \centering
  \includegraphics[scale={#3}]{#1}
  \caption{#2}
  \end{figure}
}


\begin{document}
\maketitle
\begin{center}
  % Deputado BEBETO DO TETRA
  \huge
  \vspace{-3cm}\href{http://alerjln1.alerj.rj.gov.br/scpro1115.nsf/f4b46b3cdbba990083256cc900746cf6/2ba323942b234b2f8325783200527358?OpenDocument}{PROJETO DE LEI Nº 54/2011}
\bigskip
\bigskip
\bigskip
  
\end{center}

\textbf{EMENTA:} 
ALTERA A LEI 3960/2002 DE 17 DE SETEMBRO DE 2002 NA FORMA QUE MENCIONA.








\bigskip

\noindent
A ASSEMBLEIA LEGISLATIVA DO ESTADO DO RIO DE JANEIRO RESOLVE:

\begin{enumerate}[label=Art. \arabic*\textdegree]
\item - Acrescenta Paragrafó Único ao Artigo 1º com a seguinte redação:

  Parágrafo Único - Os lugares disponibilizados aos portadores de necessidades especiais e cadeirantes deverão  também estar em área (s) considerada (s) privilegiada (s), não podendo sob qualquer hipotese ser destinado (s) espaço (s) que possa (m) traduzir constrangimento ao cliente.

\item - Esta Lei entra em vigor na data de sua publicação sendo revogada as disposições em contrário.


\end{enumerate}




\begin{center}
  Plenário Barbosa Lima Sobrinho 09 de   Fevereiro de 2011

   \bigskip

  \textbf{ BEBETO DO TETRA}

  \bigskip

  \textbf{JUSTIFICATIVA}
  \bigskip

\end{center}

  Nossos irmãos portadores de necessidades especiais merecem todos os esforços para que suas vidas sejam facilitadas. Temos uma enorme quantidade de cidadãos em nossos Estado do Rio de Janeiro que procuram levar seus cotidianos da forma mais natural possível, alguns em determinados segmentos encontram barreiras que são intransponíveis, isso, pela total falta de compreensão de pessoas que não possuem qualquer tipo de solidariedade humana, tratam esta causa de uma forma deselegante, porém se esquecem que não foram eles que escolheram esta forma de vida, que apesar de seus problemas produzem tanto quanto muitos que não são acometidos, e, muito mais que tudo são exatamente cidadãos do Estado do Rio de Janeiro, e merecem o respeito de todos. 



\iffalse
\begin{center}
  \textbf{REFERÊNCIAS}
\end{center}


\fi



\end{document}

