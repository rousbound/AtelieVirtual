\documentclass[10pt]{article}
\usepackage[portuguese]{babel}
\usepackage[utf8]{inputenc}
\usepackage[pdftex]{graphicx}
\usepackage{venndiagram}
\usepackage{subcaption}
\usepackage{caption}
\usepackage[backend=biber,style=authoryear-ibid]{biblatex}
\usepackage[normalem]{ulem}
\usepackage[margin=0.8in]{geometry}
\addbibresource{psychoanalysis.bib}
\graphicspath{{Pictures/}}
\usepackage{tikz}
\usepackage{setspace}
\usepackage{enumitem}
\usepackage{textcomp}
\usepackage{hyperref}


\date{}

\newcommand{\quotebox}[3]{
  \begin{center}
\noindent\fbox{ 
  \parbox{#3\textwidth}{%
  {\itshape#1\itshape}

  \raggedleft {\textbf{#2}} 
    }%  
}
\end{center}
}

\newcommand{\spawnfig}[3]
{
  \begin{figure}[h]
  \centering
  \includegraphics[scale={#3}]{#1}
  \caption{#2}
  \end{figure}
}


\begin{document}
\maketitle
\begin{center}
  % Deputado DANNIEL LIBRELON
  \huge
  \vspace{-3cm}\href{http://alerjln1.alerj.rj.gov.br/scpro1923.nsf/f4b46b3cdbba990083256cc900746cf6/20d9df819eb4cb938325840900603833?OpenDocument}{PROJETO DE LEI Nº 657/2019}
\bigskip
\bigskip
\bigskip
  
\end{center}

\textbf{EMENTA:} 
DISPÕE SOBRE A OBRIGATORIEDADE DE INSTALAÇÃO DE POSTOS DE ATENDIMENTO MÉDICO NAS ESTAÇÕES DO METRÔRIO E DA SUPERVIA .








\bigskip

\noindent
A ASSEMBLEIA LEGISLATIVA DO ESTADO DO RIO DE JANEIRO RESOLVE:

\begin{enumerate}[label=Art. \arabic*\textdegree]
\item - As concessionárias de serviços do MetrôRio e da Supervia que operam no âmbito do Estado do Rio de Janeiro ficam obrigadas a instalar pelo menos 1 (um) posto de atendimento médico no interior de suas estações.

\item - Estes postos de atendimento médico devem estar bem sinalizados, e em local de fácil acesso para os usuários, a fim de atender os casos de emergência.

\item - Cada posto de atendimento deve contar com pelo menos um médico, um técnico de enfermagem, equipamentos e materiais de primeiros-socorros e medicamentos.

\item - No caso de descumprimento desta Lei, aplicar-se-á multa no valor de 5.000 (cinco mil) UFIR-RJ.

\item - O Poder Executivo regulamentará a presente lei.
				
\item - Esta lei entra em vigor na data de sua publicação.


\end{enumerate}




\begin{center}
  Plenário Barbosa Lima Sobrinho, 29   de abril de 2019.

   \bigskip

  \textbf{ DANNIEL LIBRELON}

  \bigskip

  \textbf{JUSTIFICATIVA}
  \bigskip

\end{center}

  O presente projeto tem como objetivo possibilitar que os usuários dos serviços de metrô e de trens, que passam diariamente nas dependências das estações tenham um atendimento médico emergencial em caso de necessidade. 
É dever de todas as concessionárias a prestação de um serviço adequado, de qualidade e com a devida segurança.
Muitas vezes a prestação de um socorro emergencial, no momento em que o problema aconteceu pode salvar a vida de muitas pessoas.
Diante do exposto, conto com o apoio dos meus pares para aprovação deste projeto.



\iffalse
\begin{center}
  \textbf{REFERÊNCIAS}
\end{center}


\fi



\end{document}

