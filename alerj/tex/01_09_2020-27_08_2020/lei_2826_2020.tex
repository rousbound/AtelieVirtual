\documentclass[10pt]{article}
\usepackage[portuguese]{babel}
\usepackage[utf8]{inputenc}
\usepackage[pdftex]{graphicx}
\usepackage{venndiagram}
\usepackage{subcaption}
\usepackage{caption}
\usepackage[backend=biber,style=authoryear-ibid]{biblatex}
\usepackage[normalem]{ulem}
\usepackage[margin=0.8in]{geometry}
\addbibresource{psychoanalysis.bib}
\graphicspath{{Pictures/}}
\usepackage{tikz}
\usepackage{setspace}
\usepackage{enumitem}
\usepackage{textcomp}
\usepackage{hyperref}


\date{}

\newcommand{\quotebox}[3]{
  \begin{center}
\noindent\fbox{ 
  \parbox{#3\textwidth}{%
  {\itshape#1\itshape}

  \raggedleft {\textbf{#2}} 
    }%  
}
\end{center}
}

\newcommand{\spawnfig}[3]
{
  \begin{figure}[h]
  \centering
  \includegraphics[scale={#3}]{#1}
  \caption{#2}
  \end{figure}
}


\begin{document}
\maketitle
\begin{center}
  % Deputado JAIR BITTENCOURT
  \huge
  \vspace{-3cm}\href{http://alerjln1.alerj.rj.gov.br/scpro1923.nsf/f4b46b3cdbba990083256cc900746cf6/d1a35b0c79c23628032585980056d9bc?OpenDocument}{PROJETO DE LEI Nº 2826/2020}
\bigskip
\bigskip
\bigskip
  
\end{center}

\textbf{EMENTA:} 
DISPÕE SOBRE A PRORROGAÇÃO EM UM ANO DA VACINAÇÃO CONTRA O HPV DOS ADOLESCENTES QUE COMPLETARAM 15 ANOS, IDADE MÁXIMA PARA IMUNIZAÇÃO NA REDE PÚBLICA, DURANTE O ANO DE 2020 E QUE POR DETERMINAÇÃO DO ISOLAMENTO SOCIAL DEVIDO A PANDEMIA DO COVID-19 FICARAM IMPEDIDOS DE SEREM VACINADOS.








\bigskip

\noindent
A ASSEMBLEIA LEGISLATIVA DO ESTADO DO RIO DE JANEIRO RESOLVE:

\begin{enumerate}[label=Art. \arabic*\textdegree]
\item - Fica prorrogado em um ano, na rede pública de saúde dentro do Estado do Rio de Janeiro, a vacinação contra HPV dos adolescentes que completaram 15 anos durante o ano de 2020, período de pandemia da COVID-19.

\item - Esta lei entra em vigor na data de sua publicação.


\end{enumerate}




\begin{center}
  Plenário Barbosa Lima Sobrinho, 1 de julho de 2020.

   \bigskip

  \textbf{ JAIR BITTENCOURT}

  \bigskip

  \textbf{JUSTIFICATIVA}
  \bigskip

\end{center}

  A saúde de nossa população deve ser garantida mediante políticas sociais e econômicas que visem à redução do risco de doença, outros agravos e ao acesso universal e igualitário às ações e serviços para sua promoção, proteção e recuperação. Assim, considerando que a vacinação de HPV é ofertada para adolescentes até 14 anos, 11 meses e 29 dias, significando dizer, antes de completarem os 15 anos e a pandemia que estamos enfrentando do COVID-19 que tem como orientação de isolamento social que impossibilitou inúmeros adolescentes de serem imunizados devido a terem alcançado a idade máxima.
Devido a relevância do presente Projeto de Lei que resguarda o direito de vacinação dos adolescentes que completaram 15 anos durante o ano de 2020, prorrogando o período em ano, bem como o fato da medida ser preventiva de várias doenças, peço o apoio dos meus pares para aprovação da proposição.




\iffalse
\begin{center}
  \textbf{REFERÊNCIAS}
\end{center}


\fi



\end{document}

