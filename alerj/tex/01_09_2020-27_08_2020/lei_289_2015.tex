\documentclass[10pt]{article}
\usepackage[portuguese]{babel}
\usepackage[utf8]{inputenc}
\usepackage[pdftex]{graphicx}
\usepackage{venndiagram}
\usepackage{subcaption}
\usepackage{caption}
\usepackage[backend=biber,style=authoryear-ibid]{biblatex}
\usepackage[normalem]{ulem}
\usepackage[margin=0.8in]{geometry}
\addbibresource{psychoanalysis.bib}
\graphicspath{{Pictures/}}
\usepackage{tikz}
\usepackage{setspace}
\usepackage{enumitem}
\usepackage{textcomp}
\usepackage{hyperref}


\date{}

\newcommand{\quotebox}[3]{
  \begin{center}
\noindent\fbox{ 
  \parbox{#3\textwidth}{%
  {\itshape#1\itshape}

  \raggedleft {\textbf{#2}} 
    }%  
}
\end{center}
}

\newcommand{\spawnfig}[3]
{
  \begin{figure}[h]
  \centering
  \includegraphics[scale={#3}]{#1}
  \caption{#2}
  \end{figure}
}


\begin{document}
\maketitle
\begin{center}
  % Deputado CARLOS MACEDO
  \huge
  \vspace{-3cm}\href{http://alerjln1.alerj.rj.gov.br/scpro1519.nsf/f4b46b3cdbba990083256cc900746cf6/f69ef1ae7bb1309283257e200051b11e?OpenDocument}{PROJETO DE LEI Nº 289/2015}
\bigskip
\bigskip
\bigskip
  
\end{center}

\textbf{EMENTA:} 
DISPÕE SOBRE O FUNCIONAMENTO DOS POSTOS DE ATENDIMENTO PRESENCIAL DAS EMPRESAS CONCESSIONÁRIAS DE SERVIÇOS PÚBLICOS ESSENCIAIS E DÁ OUTRAS PROVIDÊNCIAS 








\bigskip

\noindent
A ASSEMBLEIA LEGISLATIVA DO ESTADO DO RIO DE JANEIRO RESOLVE:

\begin{enumerate}[label=Art. \arabic*\textdegree]
\item - As concessionárias de serviços públicos do estado do Rio de Janeiro ficam obrigadas a manterem postos de atendimento nos municípios em que prestam serviço, com o objetivo de assegurar ao consumidor o atendimento presencial nas unidades das concessionárias. não se valendo apenas do atendimento via telefonia ou através da rede mundial de computadores, aos sábados, no horário compreendido entre 8h e 14h.
Parágrafo Único: As concessionárias citadas no caput deverão prestar o serviço ao público de forma gratuita, através de distribuição de senhas por ordem de chegada, respeitando o atendimento preferencial estabelecido em lei.

\item - Os horários de atendimento devem ser estabelecidos de segunda-feira à sexta-feira:
\begin{enumerate}[label=\Roman*]
\item - oito) horas semanais em municípios com até 2.000 (duas mil) unidades consumidoras; e
\item - quatro) horas diárias em municípios com mais de 2.000 (duas mil) e até 10.000 (dez mil) unidades consumidoras ; e
\item - oito) horas diárias em municípios com mais de 10.000 (dez mil) unidades consumidoras.
\end{enumerate}
1º§ - Os horários de atendimento disponibilizados ao público devem ser regulares em cada município, previamente informados e afixados na entrada de todo posto de atendimento.
2º§ - Fica assegurado que no mínimo um sábado do mês deverá, obter atendimento presencial nas concessionarias.

\item - Nos casos previstos no artigo mencionado, a concessionária em observância aos critérios assediados, poderá substituir a loja física por unidade de atendimento presencial móvel, cuja assiduidade estará sujeita a critérios relacionados intimamente com a demanda do município. 

\item - Nos atendimentos agendados através de telefone ou internet, bem como nos efetuados em loja, a empresa concessionária deverá disponibilizar o atendimento no prazo máximo de 48h (quatro e oito horas), sendo que a previsão horária deverá ser atendida pelos períodos estabelecidos em horário comercial.

\item - O prazo para a adequação das novas medidas, pelas concessionárias de serviços públicos, será de 180 (cento e oitenta) dias a partir da publicação desta Lei.

\item - O descumprimento ao que dispõe a presente Lei acarretará a concessionária infratora multa diária no valor de 200 (duzentos) UFIR´s, devendo a referida ser revertida ao FEPROCON- Fundo Especial para Programas de Proteção e Defesa ao Consumidor.

\item - Esta Lei entrará em vigor na data de sua publicação, ficando revogada a Lei nº3878, de 24 de junho de 2002.
       

\end{enumerate}




\begin{center}
  Plenário Barbosa Lima Sobrinho, 08 de abril de 2015

   \bigskip

  \textbf{ CARLOS MACEDO}

  \bigskip

  \textbf{JUSTIFICATIVA}
  \bigskip

\end{center}

  Os usuários que trabalham durante a semana encontram dificuldades em resolver problemas relacionados à prestação de serviços públicos, devido a algumas concessionárias não disponibilizarem unidades de atendimento presenciais aos sábados.
Tais situações não apenas levam o consumidor a adiar suas reivindicações, como também causam contratempos em sua rotina de vida, uma vez que muitos têm que sair durante o horário de trabalho para resolver tais problemas. Ademais, o atendimento efetuado pelo os operadores via telefone nem sempre atendem a demanda do consumidor, eis que na sua grande maioria, aqueles treinados para exercer o atendimento não presencial carecem de informações técnicas para a solução imediata do problema ao contrário do que se propõe nesse projeto.
Por conta disso, peço a atenção aos meus pares para a relevância desta matéria que, por certo, atenderá a população com maior comodidade, facilidade e eficiência.



\iffalse
\begin{center}
  \textbf{REFERÊNCIAS}
\end{center}


\fi



\end{document}

