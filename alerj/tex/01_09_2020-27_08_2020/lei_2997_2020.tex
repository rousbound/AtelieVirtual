\documentclass[10pt]{article}
\usepackage[portuguese]{babel}
\usepackage[utf8]{inputenc}
\usepackage[pdftex]{graphicx}
\usepackage{venndiagram}
\usepackage{subcaption}
\usepackage{caption}
\usepackage[backend=biber,style=authoryear-ibid]{biblatex}
\usepackage[normalem]{ulem}
\usepackage[margin=0.8in]{geometry}
\addbibresource{psychoanalysis.bib}
\graphicspath{{Pictures/}}
\usepackage{tikz}
\usepackage{setspace}
\usepackage{enumitem}
\usepackage{textcomp}
\usepackage{hyperref}


\date{}

\newcommand{\quotebox}[3]{
  \begin{center}
\noindent\fbox{ 
  \parbox{#3\textwidth}{%
  {\itshape#1\itshape}

  \raggedleft {\textbf{#2}} 
    }%  
}
\end{center}
}

\newcommand{\spawnfig}[3]
{
  \begin{figure}[h]
  \centering
  \includegraphics[scale={#3}]{#1}
  \caption{#2}
  \end{figure}
}


\begin{document}
\maketitle
\begin{center}
  % Deputado MARCIO GUALBERTO
  \huge
  \vspace{-3cm}\href{http://alerjln1.alerj.rj.gov.br/scpro1923.nsf/f4b46b3cdbba990083256cc900746cf6/168af90f677f70fa032585c20064857d?OpenDocument}{PROJETO DE LEI Nº 2997/2020}
\bigskip
\bigskip
\bigskip
  
\end{center}

\textbf{EMENTA:} 
DETERMINA A COMPRA E TROCA PERMANENTE DE EQUIPAMENTOS DE SEGURANÇA E DE USO LABORAL DOS SERVIDORES DA PCERJ, BMERJ, PMERJ, SEAP E DEGASE, NA FORMA QUE MENCIONA E DÁ OUTRAS PROVIDÊNCIAS. 








\bigskip

\noindent
A ASSEMBLEIA LEGISLATIVA DO ESTADO DO RIO DE JANEIRO RESOLVE:

\begin{enumerate}[label=Art. \arabic*\textdegree]
\item - As compras de equipamentos de uso pessoal ou coletivo, para a prática laboral ou em razão dela, para os servidores públicos da área de Segurança Pública, Polícia Civil, Polícia Militar, DEGASE,  Bombeiro Militar e SEAP deverão ser realizadas com o prazo mínimo de até 90(noventa) dias antes do vencimento da validade de cada produto estabelecida pelo fabricante.

Parágrafo único - Para efeitos desta lei, os equipamentos de uso pessoal a que se refere o caput são entendidos, entre outros, como:

a - Coletes balísticos (à prova de balas);
b - Munições de todos os calibres utilizados pelas forças policiais;
c - capacetes;
d - viseiras;
e - armamento;
f - equipamentos de proteção individual;
g - uniformes antichamas e trajes antibombas;
h - botas e coturnos;
i - cordas;
j - coletes salva vidas;
k - botes;
l - escudos balísticos e antichoque;
m - joelheiras;
n - cotoveleiras;
o - máscaras antigases;
p - pistolas tasers;
q - munição de elastômero;
r - material de APH;

\item - O Poder Executivo deverá realizar as compras respeitando o rito e o prazo legal estabelecido pela lei 8.666, de 21 de junho de 1993, devendo porém, todo o rito legal ser finalizado em, no mínimo, até 90 (noventa) dias antes do vencimento de cada produto.

\item - Outros equipamentos que se enquadrem no disposto no parágrafo único do caput poderão ser incluídos no estabelecido por esta lei, bastando ser equipamento de segurança para o serviço dos agentes de que se trata no parágrafo único do artigo primeiro.

Parágrafo único - fica proibido o uso de munições de treinamento para serviço efetivo dos agentes, sendo sua compra permitida somente para fins de aperfeiçoamento dos servidores, com uso em locais específicos para este fim.

\item - Os equipamentos deverão ser trocados com máxima urgência de modo que se ponha em risco a saúde e a vida dos servidores elencados.

\item - Esta lei correrá por dotações orçamentárias próprias, complementadas se necessário.

\item - Esta lei entra em vigor na data de sua publicação.

\end{enumerate}




\begin{center}
  Plenário Barbosa Lima Sobrinho, 12 de agosto de 2020.

   \bigskip

  \textbf{ MARCIO GUALBERTO}

  \bigskip

  \textbf{JUSTIFICATIVA}
  \bigskip

\end{center}

  Foi veiculada em diversos meios de comunicação a informação de que dos quase 16.000 coletes balísticos da Polícia Civil do Estado do Rio de Janeiro, menos de 2.000 estão dentro do prazo de validade. Não é preciso trazer a baila a óbvia importância de tal material de segurança para estes servidores. Não é possível indicar o motivo exato que nos levou a esta situação. O que pretendo, porém, é evitar que novamente isto ocorra. Por isso, apresento a meus nobres pares o projeto de lei que pretende estabelecer a compra periódica desses e outros materiais. Conto com a colaboração de meus nobres pares para a aprovação desta propositura.



\iffalse
\begin{center}
  \textbf{REFERÊNCIAS}
\end{center}


\fi



\end{document}

