\documentclass[10pt]{article}
\usepackage[portuguese]{babel}
\usepackage[utf8]{inputenc}
\usepackage[pdftex]{graphicx}
\usepackage{venndiagram}
\usepackage{subcaption}
\usepackage{caption}
\usepackage[backend=biber,style=authoryear-ibid]{biblatex}
\usepackage[normalem]{ulem}
\usepackage[margin=0.8in]{geometry}
\addbibresource{psychoanalysis.bib}
\graphicspath{{Pictures/}}
\usepackage{tikz}
\usepackage{setspace}
\usepackage{enumitem}
\usepackage{textcomp}
\usepackage{hyperref}


\date{}

\newcommand{\quotebox}[3]{
  \begin{center}
\noindent\fbox{ 
  \parbox{#3\textwidth}{%
  {\itshape#1\itshape}

  \raggedleft {\textbf{#2}} 
    }%  
}
\end{center}
}

\newcommand{\spawnfig}[3]
{
  \begin{figure}[h]
  \centering
  \includegraphics[scale={#3}]{#1}
  \caption{#2}
  \end{figure}
}


\begin{document}
\maketitle
\begin{center}
  % Deputado SERGIO LOUBACK
  \huge
  \vspace{-3cm}\href{http://alerjln1.alerj.rj.gov.br/scpro1923.nsf/9665df2600e114f703256caa00231316/46374430d0e4a1600325850a00569138?OpenDocument}{PROJETO DE LEI Nº 378/2020}
\bigskip
\bigskip
\bigskip
  
\end{center}

\textbf{EMENTA:} 
CONCEDE MEDALHA TIRADENTES E O RESPECTIVO DIPLOMA AO TENENTE CORONEL SILVIO LUIZ DA SILVA PEKLY








\bigskip

\noindent
A ASSEMBLEIA LEGISLATIVA DO ESTADO DO RIO DE JANEIRO RESOLVE:

\begin{enumerate}[label=Art. \arabic*\textdegree]
\item - Fica concedida a MEDALHA TIRADENTES e o respectivo Diploma ao TENENTE CORONEL SILVIO LUIZ DA SILVA PEKLY.

\item - Esta Resolução entra em vigor na data de sua publicação.

\end{enumerate}




\begin{center}
  Plenário Barbosa Lima Sobrinho, 10 de fevereiro de 2020.

   \bigskip

  \textbf{ SERGIO LOUBACK}

  \bigskip

  \textbf{JUSTIFICATIVA}
  \bigskip

\end{center}

  
Tenente Coronel da Polícia Militar do Estado do Rio de Janeiro Silvio Luiz da Silva Pekly, nascido no Rio de Janeiro (RJ). Tem sua trajetória profissional marcada pelo empenho e brilhantismo notado em sua promoção de Capitão PM em 21 de agosto de 2007, posteriormente a de Major PM em 21 de agosto de 2011 e por fim, promovido a Tenente Coronel em 21 de abril de 2018.
Em sua ampla formação acadêmica constam diferenciados cursos que corroboram para sua excelente performance em suas funções, como de Formação de Oficiais &#- APM D. João VI PMERJ (1991/2001), curso de Aperfeiçoamento de Oficiais &#- ESPM/PMERJ (2010), curso de Policiamento em Praças Desportivas &#- CPPD/GEPE/PMERJ (2012) e curso de Pós Graduação em Polícia Judiciária Militar &#- Instituto Venturo (2019). Na área acadêmica também ressalta como Instrutor do Curso de Policiamento em Praças Desportivas.
Dedica-se à Polícia Militar do Estado do Rio de Janeiro e atuou em múltiplas funções tais como: Chefe da 3ª Seção do Estado Maior 20º BPM, Chefe da 3ª Seção do Estado Maior 17º BPM, Chefe da 3ª Seção do Estado Maior 7º CPA, Chefe da 1ª e 2ª Seções do Estado Maior 26º BPM, Chefe da 3ª Seção do Estado Maior do GEPE, Subcomandante do Grupamento Especial de Policiamento em Estádios (GEPE), Comandante do Grupamento Especial de Policiamento em Estádios (GEPE) e Comandante do Batalhão Especializado de Policiamento em Estádios (BEPE).
Resultados de seu profundo afinco à Polícia Militar do Estado do Rio de Janeiro sendo evidenciado em medalhas e distintivos recebidos, como: Distintivo Lealdade e Constância &#- PMERJ, Medalha 10 anos de efetivo serviço &#- PMERJ, Medalha Cruz do Mérito Empreendedor Juscelino Kubitschek &#- Ministério da Justiça, Medalha Regimento Marechal Caetano de Faria &#- Batalhão de Choque PMERJ e Medalha Ordem dos Cavaleiros Honorários &#- Regimento de Polícia Montada PMERJ.



\iffalse
\begin{center}
  \textbf{REFERÊNCIAS}
\end{center}


\fi



\end{document}

