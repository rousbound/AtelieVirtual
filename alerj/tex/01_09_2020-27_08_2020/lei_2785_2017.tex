\documentclass[10pt]{article}
\usepackage[portuguese]{babel}
\usepackage[utf8]{inputenc}
\usepackage[pdftex]{graphicx}
\usepackage{venndiagram}
\usepackage{subcaption}
\usepackage{caption}
\usepackage[backend=biber,style=authoryear-ibid]{biblatex}
\usepackage[normalem]{ulem}
\usepackage[margin=0.8in]{geometry}
\addbibresource{psychoanalysis.bib}
\graphicspath{{Pictures/}}
\usepackage{tikz}
\usepackage{setspace}
\usepackage{enumitem}
\usepackage{textcomp}
\usepackage{hyperref}


\date{}

\newcommand{\quotebox}[3]{
  \begin{center}
\noindent\fbox{ 
  \parbox{#3\textwidth}{%
  {\itshape#1\itshape}

  \raggedleft {\textbf{#2}} 
    }%  
}
\end{center}
}

\newcommand{\spawnfig}[3]
{
  \begin{figure}[h]
  \centering
  \includegraphics[scale={#3}]{#1}
  \caption{#2}
  \end{figure}
}


\begin{document}
\maketitle
\begin{center}
  % Deputado FIGUEIREDO
  \huge
  \vspace{-3cm}\href{http://alerjln1.alerj.rj.gov.br/scpro1519.nsf/f4b46b3cdbba990083256cc900746cf6/1b9d326a3d6c00f18325811b0061aa4c?OpenDocument}{PROJETO DE LEI Nº 2785/2017}
\bigskip
\bigskip
\bigskip
  
\end{center}

\textbf{EMENTA:} 
DISPÕE SOBRE A INCLUSÃO, NA CARTEIRA DE IDENTIDADE E NA CARTEIRA NACIONAL DE HABILITAÇÃO, DE INFORMAÇÕES ACERCA DE DOENÇAS DO PORTADOR.








\bigskip

\noindent
A ASSEMBLEIA LEGISLATIVA DO ESTADO DO RIO DE JANEIRO RESOLVE:

\begin{enumerate}[label=Art. \arabic*\textdegree]

\item - Fica instituído que a Secretaria Estadual de Segurança Pública e o Departamento Estadual de Trânsito - DETRAN/RJ, quando solicitados, devem incluir no documento da Carteira de Identidade (CI) e da Carteira Nacional de Habilitação (CNH) informações acerca de todo e qualquer tipo de doença que afete o portador.

\item - Esta lei entra em vigor na data de sua publicação.
	

\end{enumerate}




\begin{center}
  Plenário Barbosa Lima Sobrinho, 09 de maio de 2017.

   \bigskip

  \textbf{ FIGUEIREDO}

  \bigskip

  \textbf{JUSTIFICATIVA}
  \bigskip

\end{center}

  O projeto de lei apresentado visa incluir na Carteira de Identidade (CI) e da Carteira Nacional de Habilitação &#- CNHs emitidas no Estado do Rio de Janeiro expressões que informem a presença de doenças que acometem os portadores dos referidos documentos.
Essa medida é de grande importância, pois quando ocorre um acidente ou a pessoa é vítima de mal súbito que a deixe inconsciente, tais documentos são utilizados para sua identificação. Assim, a presença, no corpo do documento, de expressões que informem que o portador possui qualquer doença auxilia no atendimento por parte do socorrista e da equipe médica, bem como garante o tratamento adequado à vítima.
Há uma série de alergias e doenças autoimunes, se não identificadas e tratadas corretamente, podem trazer danos irreparáveis e irreversíveis a vida do paciente.



\iffalse
\begin{center}
  \textbf{REFERÊNCIAS}
\end{center}


\fi



\end{document}

