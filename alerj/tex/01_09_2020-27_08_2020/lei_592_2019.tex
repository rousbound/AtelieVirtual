\documentclass[10pt]{article}
\usepackage[portuguese]{babel}
\usepackage[utf8]{inputenc}
\usepackage[pdftex]{graphicx}
\usepackage{venndiagram}
\usepackage{subcaption}
\usepackage{caption}
\usepackage[backend=biber,style=authoryear-ibid]{biblatex}
\usepackage[normalem]{ulem}
\usepackage[margin=0.8in]{geometry}
\addbibresource{psychoanalysis.bib}
\graphicspath{{Pictures/}}
\usepackage{tikz}
\usepackage{setspace}
\usepackage{enumitem}
\usepackage{textcomp}
\usepackage{hyperref}


\date{}

\newcommand{\quotebox}[3]{
  \begin{center}
\noindent\fbox{ 
  \parbox{#3\textwidth}{%
  {\itshape#1\itshape}

  \raggedleft {\textbf{#2}} 
    }%  
}
\end{center}
}

\newcommand{\spawnfig}[3]
{
  \begin{figure}[h]
  \centering
  \includegraphics[scale={#3}]{#1}
  \caption{#2}
  \end{figure}
}


\begin{document}
\maketitle
\begin{center}
  % Deputado MÁRCIO CANELLA
  \huge
  \vspace{-3cm}\href{http://alerjln1.alerj.rj.gov.br/scpro1923.nsf/f4b46b3cdbba990083256cc900746cf6/bbc5ef2acdc927a98325840200646354?OpenDocument}{PROJETO DE LEI Nº 592/2019}
\bigskip
\bigskip
\bigskip
  
\end{center}

\textbf{EMENTA:} 
DISPÕE SOBRE A PUBLICAÇÃO NA INTERNET DE LISTA DE PESSOAS CONDENADAS POR CRIME DE VIOLÊNCIA CONTRA A MULHER, À CRIANÇA E O ADOLESCENTE, NA FORMA QUE MENCIONA.








\bigskip

\noindent
A ASSEMBLEIA LEGISLATIVA DO ESTADO DO RIO DE JANEIRO RESOLVE:

\begin{enumerate}[label=Art. \arabic*\textdegree]
\item - O Poder Executivo Estadual disponibilizará na Rede Mundial de Computadores o nome, a foto e demais dados processuais das pessoas condenadas criminalmente com trânsito em julgado por crime de violência contra o idoso, a mulher, à criança, o adolescente ou contra sua dignidade sexual.

Parágrafo único - A lista de pessoas condenadas por crime de violência contra a mulher, à criança e o adolescente será disponibilizada, observado o seguinte:

\begin{enumerate}[label=\Roman*]
\item - Qualquer cidadão poderá ter acesso ao Cadastro, relativamente à identificação e foto dos cadastrados, desde a condenação transitada em julgado até o fim do cumprimento da pena;
\end{enumerate}
 
\item - s Polícias Civil e Militar, Conselhos Tutelares, membros do Ministério Público e do Poder Judiciário, e demais autoridades, a critério da Secretaria da Segurança Pública.

\item - O Poder Executivo fica autorizado a criar aplicativo para dispositivo móvel, a ser utilizado para ampliar a disponibilização das informações a que se refere esta Lei.

\item - Os dados do condenado deverão ser eliminados do cadastro assim que ocorrer o cumprimento integral da pena ou a concessão de algum benefício que lhe garanta a liberdade condicional, permanecendo disponíveis exclusivamente para fins de consulta dos órgãos de segurança pública.

\item - As despesas decorrentes da execução desta Lei correrão por conta de dotações orçamentárias próprias, suplementadas, se necessário.

\item - Esta Lei entra em vigor na data de sua publicação. 


\end{enumerate}




\begin{center}
  Plenário Barbosa Lima Sobrinho, 22 de maio de 2019.

   \bigskip

  \textbf{ MÁRCIO CANELLA}

  \bigskip

  \textbf{JUSTIFICATIVA}
  \bigskip

\end{center}

  
O presente Projeto de Lei trás à baila tema que já havia sido alvo de estudo nesta Casa em proposição anteriormente oferecida, que tinha por inspiração na lei britânica conhecida pelo nome de Lei Clare Wood, jovem inglesa estrangulada e incendiada pelo ex-namorado George Appleton, que ela conheceu no Facebook.    A lei nº 11.340, de 7 de agosto de 2006 (Lei Maria da Penha) também estabeleceu mecanismos para coibir e prevenir a violência doméstica e familiar contra a mulher, mas o Código Penal dispõe que os processos em que se apuram crimes contra a dignidade sexual devem correr em segredo de justiça, o que acaba vedando à sociedade o conhecimento de quem são esses agentes criminosos, sendo um direito do cidadão de bem saber quem foi condenado definitivamente por este motivo, até mesmo dado o caráter preventivo desta informação.

A sociedade tem o direito de saber quem são os condenados por comportamentos que podem produzir danos à dignidade e à vida das pessoas que a integram, restando esclarecer que a proposta adotou critérios a serem observados para a disponibilização da lista de pessoas condenadas por crime de violência contra o idoso, a mulher, à criança e o adolescente a fim de se atentar aos princípios constitucionais, fazendo com que a punição dada não ultrapasse a pena judicial do condenado, o que já vem sendo adotado em outros Estados da Federação.

Desta forma, a presente proposição visa instrumentalizar o cidadão a identificar os agentes de crimes tão bárbaros, valendo-se da rede mundial de computadores para facilitar o acesso à informação do cidadão de bem e das autoridades de segurança pública, sendo certo que a implementação de tal sistema é plenamente viável em nossos dias e de baixo custo operacional. 

Diante do exposto, conto com o apoio dos meus nobres pares para a aprovação da presente proposição.



\iffalse
\begin{center}
  \textbf{REFERÊNCIAS}
\end{center}


\fi



\end{document}

