\documentclass[10pt]{article}
\usepackage[portuguese]{babel}
\usepackage[utf8]{inputenc}
\usepackage[pdftex]{graphicx}
\usepackage{venndiagram}
\usepackage{subcaption}
\usepackage{caption}
\usepackage[backend=biber,style=authoryear-ibid]{biblatex}
\usepackage[normalem]{ulem}
\usepackage[margin=0.8in]{geometry}
\addbibresource{psychoanalysis.bib}
\graphicspath{{Pictures/}}
\usepackage{tikz}
\usepackage{setspace}
\usepackage{enumitem}
\usepackage{textcomp}
\usepackage{hyperref}


\date{}

\newcommand{\quotebox}[3]{
  \begin{center}
\noindent\fbox{ 
  \parbox{#3\textwidth}{%
  {\itshape#1\itshape}

  \raggedleft {\textbf{#2}} 
    }%  
}
\end{center}
}

\newcommand{\spawnfig}[3]
{
  \begin{figure}[h]
  \centering
  \includegraphics[scale={#3}]{#1}
  \caption{#2}
  \end{figure}
}


\begin{document}
\maketitle
\begin{center}
  % Deputado MARCIO GUALBERTO
  \huge
  \vspace{-3cm}\href{http://alerjln1.alerj.rj.gov.br/scpro1923.nsf/f4b46b3cdbba990083256cc900746cf6/1a80f05e623824210325852f005acaec?OpenDocument}{PROJETO DE LEI Nº 2035/2020}
\bigskip
\bigskip
\bigskip
  
\end{center}

\textbf{EMENTA:} 
QUE ALTERA A LEI 3796 DE 01 DE ABRIL DE 2002, AMPLIANDO O ATENDIMENTO PRIORITÁRIO AOS IDOSOS E GRUPO DE RISCO EM CASOS DE ENDEMIA, EPIDEMIA E PANDEMIA, NA FORMA QUE MENCIONA.  








\bigskip

\noindent
A ASSEMBLEIA LEGISLATIVA DO ESTADO DO RIO DE JANEIRO RESOLVE:

\begin{enumerate}[label=Art. \arabic*\textdegree]
\item - Modifique-se a ementa da Lei 3796 de 01 de abril de 2002, que passa a ter a seguinte redação: 
&#``QUE ESTABELECE O ATENDIMENTO PRIORITÁRIO E A VALORIZAÇÃO DA VIDA DOS IDOSOS E DOS INTEGRANTES DO GRUPO DE RISCO EM CASOS DE ENDEMIA, EPIDEMIA E PANDEMIA, NA FORMA QUE MENCIONA.&#" 
 
\item - Modifique-se o artigo 1° da Lei 3796 de 01 de abril de 2002, que passa a ter a seguinte redação: 
&#``Art. 1° - Fica estabelecido o atendimento prioritário e a valorização da vida das pessoas idosas e das pessoas que enquadram o grupo de risco em casos de endemia, epidemia e pandemia, em toda a rede de saúde, pública ou privada, no âmbito do Estado do Rio de Janeiro. 
§1° - Fica estabelecido que a pessoa idosa é aquela com idade igual ou superior a 60 (sessenta) anos, na forma do artigo 1° da Lei n° 10.741, de 1° de outubro de 2003 - Estatuto do Idoso. 
§2° - Para efeitos desta lei, o grupo de risco a que se refere o caput deste artigo, será aquele com pessoas que figurem no seguinte quadro: 
\begin{enumerate}[label=\Roman*]
\item - Pessoas com problemas cardíacos; 
\item - Pessoas com Problemas respiratórios de qualquer natureza; 
\item - Diabéticos; 
\item - Hipertensos; 
\item - Gestantes; 
\item - Quaisquer enfermidades que agravem o quadro do paciente em caso de contaminação com a doença endêmica ou pandêmica 
\item - Lactantes 
\item - Crianças.&#" 
\end{enumerate}
 
\item - Modifique-se o artigo 2° da Lei 3796 de 01 de abril de 2002, que passa a ter a seguinte redação: 
&#``Art. 1° ... 
(...) 
\item - O não cumprimento no disposto nesta Lei acarretará em multa administrativa, de acordo com o previsto no artigo 132 do Código Penal, aplicada ao diretor, chefe ou encarregado da unidade médico-hospitalar recalcitrante. 
§1° Qualquer idoso, membro do grupo de risco, ou seu respectivo representante legal poderá denunciar a prática abusiva prevista no caput deste artigo, bastando para tal, o comparecimento  à delegacia para registrar a ocorrência. 
§2° - O valor da multa fica estabelecido em 1000 (hum mil) UFIRs, em caso de descumprimento do disposto nesta lei, sem prejuízo da legislação penal. 
§3° - Em caso de morte das pessoas elencadas no caput desta lei, em decorrência do não atendimento devido, esgotados todos os recursos cabíveis para a manutenção da vida, a multa será atribuída no valor de 10000 (dez mil) UFIRs.&#" 
 
\item - O poder executivo regulamentará a presente lei. 
 
\item - Esta lei entra em vigor na data de sua publicação. 
 

\end{enumerate}




\begin{center}
  Plenário Barbosa Lima Sobrinho, 18 de fevereiro de 2020.

   \bigskip

  \textbf{ MARCIO GUALBERTO}

  \bigskip

  \textbf{JUSTIFICATIVA}
  \bigskip

\end{center}

  O nosso Estado deve agir com celeridade na tentativa de evitar os danos causados pela Pandemia que paira sobre nós. Apresento a seguinte proposição com o intuito de evitar a perda dos mais afetados pelo COVID-19, que são os idosos e o grupo de risco. Pretendo, pois, estabelecer prioridade para o seu atendimento, visando garantir a saúde física e mental destes. Conto com o apoio dos nobres pares para a aprovação do projeto.



\iffalse
\begin{center}
  \textbf{REFERÊNCIAS}
\end{center}


\fi



\end{document}

