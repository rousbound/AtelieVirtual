\documentclass[10pt]{article}
\usepackage[portuguese]{babel}
\usepackage[utf8]{inputenc}
\usepackage[pdftex]{graphicx}
\usepackage{venndiagram}
\usepackage{subcaption}
\usepackage{caption}
\usepackage[backend=biber,style=authoryear-ibid]{biblatex}
\usepackage[normalem]{ulem}
\usepackage[margin=0.8in]{geometry}
\addbibresource{psychoanalysis.bib}
\graphicspath{{Pictures/}}
\usepackage{tikz}
\usepackage{setspace}
\usepackage{enumitem}
\usepackage{textcomp}
\usepackage{hyperref}


\date{}

\newcommand{\quotebox}[3]{
  \begin{center}
\noindent\fbox{ 
  \parbox{#3\textwidth}{%
  {\itshape#1\itshape}

  \raggedleft {\textbf{#2}} 
    }%  
}
\end{center}
}

\newcommand{\spawnfig}[3]
{
  \begin{figure}[h]
  \centering
  \includegraphics[scale={#3}]{#1}
  \caption{#2}
  \end{figure}
}


\begin{document}
\maketitle
\begin{center}
  % Deputado ANDERSON MORAES
  \huge
  \vspace{-3cm}\href{http://alerjln1.alerj.rj.gov.br/scpro1923.nsf/f4b46b3cdbba990083256cc900746cf6/30bdb3a66f639899032585a4006ece35?OpenDocument}{PROJETO DE LEI Nº 2874/2020}
\bigskip
\bigskip
\bigskip
  
\end{center}

\textbf{EMENTA:} 
DISPÕE SOBRE A DESTINAÇÃO DOS VALORES REFERENTES AO PRODUTO OU PROVEITO DECORRENTES DE CRIMES PRATICADOS POR AGENTES PÚBLICOS, NAS AÇÕES DE ENFRENTAMENTO A PANDEMIA DO CORONAVÍRUS - COVID-19.








\bigskip

\noindent
A ASSEMBLEIA LEGISLATIVA DO ESTADO DO RIO DE JANEIRO RESOLVE:

\begin{enumerate}[label=Art. \arabic*\textdegree]
\item - Os valores referentes ao produto ou proveito decorrente de crimes praticados por agentes públicos nas ações de enfrentamento à pandemia do Coronavírus - Covid-19, ficam destinados a Agência Estadual de Fomento (AGERIO).

Parágrafo Único: A Agência Estadual de Fomento (AGERIO), deverá investir os valores, prioritariamente, na recuperação econômica do setor de bares e restaurantes.


\item - Esta Lei entrará em vigor na data de sua publicação.


\end{enumerate}




\begin{center}
  Plenário Barbosa Lima Sobrinho, 13 de julho de 2020.

   \bigskip

  \textbf{ ANDERSON MORAES}

  \bigskip

  \textbf{JUSTIFICATIVA}
  \bigskip

\end{center}

  O presente Projeto de Lei tem o objetivo de destinar, os valores recuperados pelos órgãos de persecução penal e pela justiça estadual, decorrentes do produto ou proveito de crimes praticados, contra os recursos que eram destinados ao combate e prevenção da pandemia COVID-19, no Estado do Rio de Janeiro, para a Agência Estadual de Fomento (AGERIO).


É notório que as últimas notícias de fortes indícios de corrupção e desvio de dinheiro público na Secretária Estadual de Saúde, provoca uma revolta na população fluminense, que vive um cenário social caótico, provocado pela pandemia do covid-19 e pela gravíssima crise econômica que se avizinha.

A população fluminense esperava do poder público estadual, uma conduta ilibada e transparente, em um período pandêmico sem precedentes, onde várias pessoas estão perdendo seus entes queridos, assistindo seus empreendimentos entrarem em falência. Contudo o que está sendo amplamente divulgados pelos mais diversos meios de comunicação, é uma série de escândalos de corrupção, superfaturamento de equipamento hospitalares e hospitais de campanha milionários que sequer saíram do papel, um escarnio com a população.

Desta forma, é legítima e meritória essa proposta legislativa, para que os valores desviados pelos esquemas de corrupção e recuperados pelo brilhante trabalho conjunto dos órgãos de persecução penal estadual e Poder Judiciário Estadual, retornem para a população, através de investimentos, principalmente no setor de bares e restaurante, um dos mais atingidos pela pelas políticas de isolamento social, mas forte aliado na recuperação econômica e geração de empregos, no Estado do Rio de Janeiro.  









Matérias associadas ao tema:

 www.g1.globo.com/rj/rio-de-janeiro/noticia/2020/07/11/ex-secretario-edmar-santos-sabia-das-contratacoes-suspeitas-segundo-ex-servidores-da-saude-do-rj.ghtml

www.oglobo.globo.com/rio/escandalo-na-saude-do-rj-quase-1-bilhao-foi-empenhado-em-contratos-de-emergencia-1-24446031

www.noticias.r7.com/rio-de-janeiro/mp-apreende-r-5-mi-em-endereco-de-ex-secretario-de-saude-do-rj-11072020


www.jovempan.com.br/programas/jornal-da-manha/denuncias-corrupcao-afastamentos-rj.html



\iffalse
\begin{center}
  \textbf{REFERÊNCIAS}
\end{center}


\fi



\end{document}

