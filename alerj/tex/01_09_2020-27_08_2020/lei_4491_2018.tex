\documentclass[10pt]{article}
\usepackage[portuguese]{babel}
\usepackage[utf8]{inputenc}
\usepackage[pdftex]{graphicx}
\usepackage{venndiagram}
\usepackage{subcaption}
\usepackage{caption}
\usepackage[backend=biber,style=authoryear-ibid]{biblatex}
\usepackage[normalem]{ulem}
\usepackage[margin=0.8in]{geometry}
\addbibresource{psychoanalysis.bib}
\graphicspath{{Pictures/}}
\usepackage{tikz}
\usepackage{setspace}
\usepackage{enumitem}
\usepackage{textcomp}
\usepackage{hyperref}


\date{}

\newcommand{\quotebox}[3]{
  \begin{center}
\noindent\fbox{ 
  \parbox{#3\textwidth}{%
  {\itshape#1\itshape}

  \raggedleft {\textbf{#2}} 
    }%  
}
\end{center}
}

\newcommand{\spawnfig}[3]
{
  \begin{figure}[h]
  \centering
  \includegraphics[scale={#3}]{#1}
  \caption{#2}
  \end{figure}
}


\begin{document}
\maketitle
\begin{center}
  % Deputado WALDECK CARNEIRO
  \huge
  \vspace{-3cm}\href{http://alerjln1.alerj.rj.gov.br/scpro1519.nsf/f4b46b3cdbba990083256cc900746cf6/eb451c5dac7839350325834d0058b41e?OpenDocument}{PROJETO DE LEI Nº 4491/2018}
\bigskip
\bigskip
\bigskip
  
\end{center}

\textbf{EMENTA:} 
INSTITUI A OBRIGATORIEDADE DE INSTALAÇÃO DE PONTOS DE RECARGA PARA VEÍCULOS ELÉTRICOS, NA FORMA QUE MENCIONA.








\bigskip

\noindent
A ASSEMBLEIA LEGISLATIVA DO ESTADO DO RIO DE JANEIRO RESOLVE:

\begin{enumerate}[label=Art. \arabic*\textdegree]

\item - As concessionárias de serviços de distribuição de energia elétrica serão obrigadas a instalar pontos de recarga de baterias de veículos elétricos em estacionamentos públicos.

\item - O Poder Executivo desenvolverá mecanismos que promovam a instalação, em prédios residenciais e comerciais, de tomadas para recarga de veículos elétricos em suas garagens.

\item - Para os efeitos desta Lei, define-se como veículo elétrico aquele que, independentemente do número de eixos, é acionado por pelo menos um motor elétrico.

Parágrafo Único: Para a aplicação desta Lei, enquadra-se nessa definição, além de veículos a bateria, os veículos híbridos cujas baterias também podem ser recarregadas eletricamente.

\item - O Poder Executivo regulamentará a presente Lei para determinar, entre outros aspectos, que as concessionárias citadas no caput do Art. 1º terão prazo de até 12 (doze) meses para se adaptar ao disposto nesta Lei.

\item - Esta Lei entrará em vigor na data de sua publicação.


\end{enumerate}




\begin{center}
  Plenário Barbosa Lima Sobrinho, 22 de novembro de 2018

   \bigskip

  \textbf{ WALDECK CARNEIRO}

  \bigskip

  \textbf{JUSTIFICATIVA}
  \bigskip

\end{center}

   O intuito da proposição ora apresentada é evitar que as concessionárias de energia elétrica fiquem à margem das mudanças no setor de transporte urbano, notadamente quanto às inovações tecnológicas da indústria automobilísticas destinadas à redução das emissões de carbono. No Brasil, o Instituto de Eletrotécnica e Energia da Universidade de São Paulo (USP), em parceria com a concessionária de energia EDP, instalou, no estacionamento daquela Universidade, ponto de abastecimento para a frota de veículos elétricos de São Paulo. Na cidade e Curitiba (PR), o Hibridus, ônibus que tem dois motores que funcionem em paralelo, é mais um exemplo de viabilidade da utilização de energia limpa no transporte urbano.



\iffalse
\begin{center}
  \textbf{REFERÊNCIAS}
\end{center}


\fi



\end{document}

