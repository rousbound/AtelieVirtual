\documentclass[10pt]{article}
\usepackage[portuguese]{babel}
\usepackage[utf8]{inputenc}
\usepackage[pdftex]{graphicx}
\usepackage{venndiagram}
\usepackage{subcaption}
\usepackage{caption}
\usepackage[backend=biber,style=authoryear-ibid]{biblatex}
\usepackage[normalem]{ulem}
\usepackage[margin=0.8in]{geometry}
\addbibresource{psychoanalysis.bib}
\graphicspath{{Pictures/}}
\usepackage{tikz}
\usepackage{setspace}
\usepackage{enumitem}
\usepackage{textcomp}
\usepackage{hyperref}


\date{}

\newcommand{\quotebox}[3]{
  \begin{center}
\noindent\fbox{ 
  \parbox{#3\textwidth}{%
  {\itshape#1\itshape}

  \raggedleft {\textbf{#2}} 
    }%  
}
\end{center}
}

\newcommand{\spawnfig}[3]
{
  \begin{figure}[h]
  \centering
  \includegraphics[scale={#3}]{#1}
  \caption{#2}
  \end{figure}
}


\begin{document}
\maketitle
\begin{center}
  % Deputado ROSENVERG REIS
  \huge
  \vspace{-3cm}\href{http://alerjln1.alerj.rj.gov.br/scpro1923.nsf/f4b46b3cdbba990083256cc900746cf6/d5e437fdb820d951832584b00043bed3?OpenDocument}{PROJETO DE LEI Nº 1603/2019}
\bigskip
\bigskip
\bigskip
  
\end{center}

\textbf{EMENTA:} 
ALTERA A LEI Nº 5.645, DE 06 DE JANEIRO DE 2010, PARA INSTITUIR NO CALENDÁRIO OFICIAL DO ESTADO DO RIO DE JANEIRO O DIA ESTADUAL DO PERITO JUDICIAL, NO ÂMBITO DO ESTADO DO RIO DE JANEIRO.  








\bigskip

\noindent
A ASSEMBLEIA LEGISLATIVA DO ESTADO DO RIO DE JANEIRO RESOLVE:

\begin{enumerate}[label=Art. \arabic*\textdegree]
\item - Fica instituído no âmbito do Estado do Rio de Janeiro o Dia Estadual do Perito Judicial, que se realizará anualmente, no dia 05 de julho.

\item - O Anexo da Lei nº 5645, de 06 de Janeiro de 2010 passa a vigorar com a seguinte redação:


CALENDÁRIO DE DATAS COMEMORATIVAS DO ESTADO DO RIO DE JANEIRO


(&#8230;)

JULHO

(&#8230;)

DIA 05 - DIA ESTADUAL DO PERITO JUDICIAL.

(...)

\item - Esta Lei entra em vigor na data de sua publicação.


\end{enumerate}




\begin{center}
  Plenário Barbosa Lima Sobrinho, 12 de novembro de 2019.

   \bigskip

  \textbf{ ROSENVERG REIS}

  \bigskip

  \textbf{JUSTIFICATIVA}
  \bigskip

\end{center}

  O perito judicial é o expert do juízo, sendo o mesmo nomeado por decisão judicial do magistrado, sendo respaldado o mister na Lei Processual Civil. O trabalho de perito judicial é desenvolvido para auxiliar o Poder Judiciário em suas decisões sendo, portanto, de relevância ímpar a atuação da categoria em destaque para os jurisdicionados que procuram o Poder Judiciário.

O Tribunal de Justiça do Estado do Rio de Janeiro possui em seu sistema mais de 3.000 peritos judiciais cadastrados no SEJUD, sendo diversas atividades laborativas, tais como engenharia, advocacia, grafotecnia, medicina, odontologia, dentre outras atividades que são desempenhadas para ser realizada a Justiça. Estes profissionais têm a função de zelar pelo cumprimento das normas legais e realizar perícias judiciais, elaborando laudos e pareceres técnicos para respaldar decisões judiciais.

Diante do importante trabalho realizado para o nosso estado, especialmente para o Poder Judiciário e os jurisdicionados, este projeto tem o objetivo de homenagear estes peritos judiciais que, muitas vezes, são esquecidos, mas são extremamente relevantes para o desfecho de diversas demandas judiciais e, por conseguinte, para o povo do Estado do Rio de Janeiro na mais distante comarca que, ao procurar o judiciário, terá um perito judicial para atuar em casos que o magistrado o designar, reconhecendo, assim, sua importância no sistema jurídico estadual.



\iffalse
\begin{center}
  \textbf{REFERÊNCIAS}
\end{center}


\fi



\end{document}

