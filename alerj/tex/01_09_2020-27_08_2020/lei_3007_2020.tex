\documentclass[10pt]{article}
\usepackage[portuguese]{babel}
\usepackage[utf8]{inputenc}
\usepackage[pdftex]{graphicx}
\usepackage{venndiagram}
\usepackage{subcaption}
\usepackage{caption}
\usepackage[backend=biber,style=authoryear-ibid]{biblatex}
\usepackage[normalem]{ulem}
\usepackage[margin=0.8in]{geometry}
\addbibresource{psychoanalysis.bib}
\graphicspath{{Pictures/}}
\usepackage{tikz}
\usepackage{setspace}
\usepackage{enumitem}
\usepackage{textcomp}
\usepackage{hyperref}


\date{}

\newcommand{\quotebox}[3]{
  \begin{center}
\noindent\fbox{ 
  \parbox{#3\textwidth}{%
  {\itshape#1\itshape}

  \raggedleft {\textbf{#2}} 
    }%  
}
\end{center}
}

\newcommand{\spawnfig}[3]
{
  \begin{figure}[h]
  \centering
  \includegraphics[scale={#3}]{#1}
  \caption{#2}
  \end{figure}
}


\begin{document}
\maketitle
\begin{center}
  % Deputado BRUNO DAUAIRE
  \huge
  \vspace{-3cm}\href{http://alerjln1.alerj.rj.gov.br/scpro1923.nsf/f4b46b3cdbba990083256cc900746cf6/14c84b54c826700a032585c8005ca096?OpenDocument}{PROJETO DE LEI Nº 3007/2020}
\bigskip
\bigskip
\bigskip
  
\end{center}

\textbf{EMENTA:} 
ALTERA À LEI Nº 8.484, DE 26 DE JULHO DE 2019, QUE INSTITUI REGIME DIFERENCIADO DE TRIBUTAÇÃO PARA O SETOR DE JOALHERIA, OURIVESARIA E BIJUTERIA








\bigskip

\noindent
A ASSEMBLEIA LEGISLATIVA DO ESTADO DO RIO DE JANEIRO RESOLVE:

\begin{enumerate}[label=Art. \arabic*\textdegree]
\item - Ficam alterados os arts. 1º, caput e inciso II, 2º e 6º da Lei nº 8.484, de 26 de julho de 2019, que passam a vigorar com a seguinte redação:

&#``Art. 1º Fica estabelecido, nos termos do § 8º do artigo 3º da Lei Complementar nº 160/2017, de 07 de agosto de 2017, tratamento tributário especial para os  estabelecimentos localizados no Estado do Rio de Janeiro que realizem operações internas com artefatos de joalheria e ourivesaria, a fim de que possam optar, em substituição ao regime normal de apuração e recolhimento do imposto, pela tributação nos seguintes termos:
\begin{enumerate}[label=\Roman*]
\item - 
\item - alíquota de 12% (doze por cento), nas saídas realizadas por estabelecimentos comerciais.
\end{enumerate}
§ 1º (...)
§ 2º (...)
\item - O disposto no inc. II do caput e nos §§ 1º e 2º, todos do art. 1º, aplica-se, também, às operações com artefatos de bijuterias e com relógios e suas peças.
(...)
\item - O incentivo previsto no inc. I do art. 1° decorre de adesão ao disposto no art. 75, inciso XXVIII, da Parte Geral, do Regulamento do ICMS aprovado pelo Decreto do Estado de Minas Gerais nº 47.604/2018, de 28 de dezembro de 2018, e produzirá efeitos até a data de 31 de dezembro de 2032.&#"


\item - Esta lei entra em vigor na data da sua publicação.


\end{enumerate}




\begin{center}
  Plenário Barbosa Lima Sobrinho, 13 de agosto de 2020.

   \bigskip

  \textbf{ BRUNO DAUAIRE}

  \bigskip

  \textbf{JUSTIFICATIVA}
  \bigskip

\end{center}

  O presente projeto de lei busca corrigir alguns aspectos da Lei nº 8.484, de 26 de julho de 2019, para fins de adequação às regras estabelecidas na Lei Complementar nº 160/2017 e no Convênio ICMS nº 190/2017.

Altera-se o caput de art. 1º da Lei nº 8.484, de 26 de julho de 2019, para fins de limitar o incentivo fiscal às operações internas com artefatos de joalheria e ourivesaria, já que o incentivo fiscal que serviu de paradigma não contempla bijuterias. Essa alteração é importante para não configurar uma ampliação do incentivo fiscal que serviu de paradigma.

Modifica-se, também, a redação do inc. II do art. 1º para que a alíquota seja fixada em 12\% nas operações realizadas por estabelecimentos comerciais, aplicando
essa mesma alíquota às operações com bijuterias e com relógios e suas peças.

Embora se mantenha a mesma tributação que está prevista na redação original, a alteração visa evitar uma discussão jurídica no sentido de se a redução da base de
cálculo, de modo que a tributação efetiva seja equivalente a 12\%, caracteriza ou não um incentivo fiscal.

Definindo-se a alíquota a 12\% (doze por cento), evita-se a discussão já que é pacífico que os Estados têm a discricionariedade política para fixar alíquotas internas
do ICMS no patamar mínimo de 12\% (doze por cento).

Por fim, propõe-se a alteração do art. 6º para fins de deixar claro que o incentivo fiscal a que se adere produzirá efeitos até a data de 31 de dezembro de 2032.

Tais alterações são importantes para dar segurança jurídica aos contribuintes e, por outro lado, evitar problemas com o regime de recuperação fiscal.



\iffalse
\begin{center}
  \textbf{REFERÊNCIAS}
\end{center}


\fi



\end{document}

