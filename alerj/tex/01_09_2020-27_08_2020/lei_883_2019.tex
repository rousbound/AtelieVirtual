\documentclass[10pt]{article}
\usepackage[portuguese]{babel}
\usepackage[utf8]{inputenc}
\usepackage[pdftex]{graphicx}
\usepackage{venndiagram}
\usepackage{subcaption}
\usepackage{caption}
\usepackage[backend=biber,style=authoryear-ibid]{biblatex}
\usepackage[normalem]{ulem}
\usepackage[margin=0.8in]{geometry}
\addbibresource{psychoanalysis.bib}
\graphicspath{{Pictures/}}
\usepackage{tikz}
\usepackage{setspace}
\usepackage{enumitem}
\usepackage{textcomp}
\usepackage{hyperref}


\date{}

\newcommand{\quotebox}[3]{
  \begin{center}
\noindent\fbox{ 
  \parbox{#3\textwidth}{%
  {\itshape#1\itshape}

  \raggedleft {\textbf{#2}} 
    }%  
}
\end{center}
}

\newcommand{\spawnfig}[3]
{
  \begin{figure}[h]
  \centering
  \includegraphics[scale={#3}]{#1}
  \caption{#2}
  \end{figure}
}


\begin{document}
\maketitle
\begin{center}
  % Deputado CORONEL SALEMA
  \huge
  \vspace{-3cm}\href{http://alerjln1.alerj.rj.gov.br/scpro1923.nsf/f4b46b3cdbba990083256cc900746cf6/9c1d2619dc13c83e8325842c005bd7b4?OpenDocument}{PROJETO DE LEI Nº 883/2019}
\bigskip
\bigskip
\bigskip
  
\end{center}

\textbf{EMENTA:} 
ALTERA A LEI Nº 3527, DE 09 DE JANEIRO DE 2001, QUE INSTITUI AUXÍLIO-INVALIDEZ POR LESÃO À INTEGRIDADE FÍSICA TENDO POR DESTINATÁRIO POLICIAL CIVIL, POLICIAL MILITAR, BOMBEIRO MILITAR E AGENTE DO DESIPE.








\bigskip

\noindent
A ASSEMBLEIA LEGISLATIVA DO ESTADO DO RIO DE JANEIRO RESOLVE:

\begin{enumerate}[label=Art. \arabic*\textdegree]
	Art. 1º - Altera a Ementa da Lei nº 3527, de 09 de janeiro de 2001, que passa a vigorar com a seguinte redação:

	&#``INSTITUI AUXÍLIO-INVALIDEZ POR LESÃO À INTEGRIDADE FÍSICA TENDO POR DESTINATÁRIO POLICIAL CIVIL, POLICIAL MILITAR, BOMBEIRO MILITAR E INSPETOR DE SEGURANÇA E ADMINISTRAÇÃO PENITENCIÁRIA&#".

	Art. 1º - Altera o Art. 1º da Lei nº 3527, de 09 de janeiro de 2001, que passa a vigorar com a seguinte redação:

	Art. 1º - O policial, civil e militar, o bombeiro militar e o inspetor de segurança e administração penitenciária que foi ou que venha a ser aposentado ou reformado por incapacidade definitiva e considerado inválido, em razão de paraplegia, tetraplegia, lesão motora total ou parcial,  sequelas em decorrência de traumatismo crânio-encefálico, cegueira total ou monocular, bem como da amputação de membro (s) superior (es) e/ou inferior (es), decorrente de acidente de serviço, impossibilitado total e permanente para qualquer atividade laboral, não podendo prover os meios de sua subsistência, fará jus a auxílio-invalidez, a ser pago, mensalmente no valor de R\$ 3.000,00 (três mil reais).

	Art. 2º - Esta Lei entra em vigor na data de sua publicação.


\end{enumerate}




\begin{center}
  Plenário Barbosa Lima Sobrinho, 03 de julho de 2019.

   \bigskip

  \textbf{ CORONEL SALEMA}

  \bigskip

  \textbf{JUSTIFICATIVA}
  \bigskip

\end{center}

  O presente Projeto de Lei que submeto a apreciação desta Casa Legislativa, tem por objetivo ALTERAR A LEI 3527, DE 09 DE JANEIRO DE 2001, QUE INSTITUI AUXÍLIO-INVALIDEZ POR LESÃO À INTEGRIDADE FÍSICA TENDO POR DESTINATÁRIO POLICIAL CIVIL, POLICIAL MILITAR E AGENTE DO DESIPE, com vistas a corrigir terminologia da Ementa, bem como acrescentar &#``lesão motora total ou parcial&#" e o &#``traumatismo crânio-encefálico&#" por ferimentos advindos de  instrumentos perfuro contundentes (arma de fogo).

Muitos servidores aposentados ou reformados por incapacidade definitiva e inválidos para exercer a atividade laboral não recebem tal adicional por não estarem inseridos no rol das lesões descritas na Lei 3527/2001.

	Vale esclarecer que muitas lesões, principalmente, por arma de fogo, podem levar o servidor a perda de movimentos de um braço ou perna, sem deixa-lo totalmente paraplégico ou tetraplégico, porém incapaz para exercer qualquer função.
	Também, aqueles servidores que sofrem lesões no cérebro que podem afetar a fala e o uso da linguagem.
	Assim, a proposta que ora apresento, configura medida de inteira justiça  ao estender o recebimento do auxílio aos profissionais, que nas mesmas circunstâncias, tenham sofrido lesões motora total ou parcial como perda de movimentos de braço ou perna, bem como como o traumatismo crânio-encefálico que pode comprometer a fala e o uso da linguagem.



\iffalse
\begin{center}
  \textbf{REFERÊNCIAS}
\end{center}


\fi



\end{document}

