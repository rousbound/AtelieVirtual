\documentclass[10pt]{article}
\usepackage[portuguese]{babel}
\usepackage[utf8]{inputenc}
\usepackage[pdftex]{graphicx}
\usepackage{venndiagram}
\usepackage{subcaption}
\usepackage{caption}
\usepackage[backend=biber,style=authoryear-ibid]{biblatex}
\usepackage[normalem]{ulem}
\usepackage[margin=0.8in]{geometry}
\addbibresource{psychoanalysis.bib}
\graphicspath{{Pictures/}}
\usepackage{tikz}
\usepackage{setspace}
\usepackage{enumitem}
\usepackage{textcomp}
\usepackage{hyperref}


\date{}

\newcommand{\quotebox}[3]{
  \begin{center}
\noindent\fbox{ 
  \parbox{#3\textwidth}{%
  {\itshape#1\itshape}

  \raggedleft {\textbf{#2}} 
    }%  
}
\end{center}
}

\newcommand{\spawnfig}[3]
{
  \begin{figure}[h]
  \centering
  \includegraphics[scale={#3}]{#1}
  \caption{#2}
  \end{figure}
}


\begin{document}
\maketitle
\begin{center}
  % Deputado DANNIEL LIBRELON
  \huge
  \vspace{-3cm}\href{http://alerjln1.alerj.rj.gov.br/scpro1923.nsf/f4b46b3cdbba990083256cc900746cf6/6cfcb47c29cfb7838325844900506f8c?OpenDocument}{PROJETO DE LEI Nº 946/2019}
\bigskip
\bigskip
\bigskip
  
\end{center}

\textbf{EMENTA:} 
ALTERA O ANEXO DA LEI Nº 5645, DE 6 DE JANEIRO     DE 2010, INCLUINDO,      NO CALENDÁRIO OFICIAL DO ESTADO DO RIO DE JANEIRO, A SEMANA ESTADUAL DO TRABALHO SOCIAL EVANGÉLICO.








\bigskip

\noindent
A ASSEMBLEIA LEGISLATIVA DO ESTADO DO RIO DE JANEIRO RESOLVE:

\begin{enumerate}[label=Art. \arabic*\textdegree]

\item - Fica alterado o anexo da Lei nº 5645, de 06 de janeiro de 2010, incluindo no calendário oficial do Estado do Rio de Janeiro a Semana Estadual do Trabalho Social Evangélico.

\item - A Semana Estadual do Trabalho Evangélico terá como objetivos:

\begin{enumerate}[label=\Roman*]
\item - apoiar, incentivar e valorizar a difusão dos serviços sociais realizados pelas comunidades evangélicas;
\end{enumerate}

\item - promover, aperfeiçoar e divulgar os trabalhos sociais realizados pelas igrejas e entidades evangélicas no Estado do Rio de Janeiro;

\item - estimular a exposição e produção de trabalhos sociais evangélicos no Estado do Rio de Janeiro.

\item - Durante a Semana Estadual do Trabalho Social Evangélico as entidades representativas do segmento, assim como a administração pública direta e indireta promoverão em parceria eventos públicos e ações sociais voltados para a toda a população.

\item - Para a divulgação desta Semana e conseqüente valorização do trabalho social evangélico o poder executivo poderá firmar convênios não 

onerosos com instituições públicas e privadas, para que sejam elaboradas campanhas publicitárias.

\item - As despesas decorrentes da aplicação desta Lei correrão por conta de dotações orçamentárias próprias para este fim, suplementadas se necessárias.Art. 6º - O Anexo da Lei nº 5.645, de 06 de janeiro de 2010, passa a vigorar com a seguinte redação:"ANEXOCALENDÁRIO DATAS COMEMORATIVAS DO ESTADO DO RIO DE JANEIRO

JANEIRO

(&#8230;)JUNHO
(&#8230;)

TERCEIRA SEMANA DE JUNHO - SEMANA ESTADUAL DO TRABALHO SOCIAL EVANGÉLICO.


                                                           (...)&#"



\item - Esta lei entra em vigor na data de sua publicação.




\end{enumerate}




\begin{center}
  Plenário Barbosa Lima Sobrinho, 01 de agosto de 2019.

   \bigskip

  \textbf{ DANNIEL LIBRELON}

  \bigskip

  \textbf{JUSTIFICATIVA}
  \bigskip

\end{center}

  É inegável a importância de trabalhos sociais evangélicos, através de organizações internacionais, nacionais e regionais, no auxílio a crianças, idosos, dependentes químicos (álcool e drogas),  missionários e outros.
Tanto as organizações não governamentais como as igrejas com seus programas de ação social têm se constituído como uma força fundamental contra a exclusão e as desigualdades enfrentadas por milhões de pessoas que sofrem constantemente com a falta de condições dignas de vida.
O censo demográfico realizado no ano de 2010, pelo IBGE, apontou a seguinte composição religiosa no Brasil: 64,6\% dos brasileiros declaram-se católicos; 22\% declaram-se protestantes (evangélicos tradicionais, pentecostais e neopentecostais); temos assim, um total de 86,8\% de cristãos no país, um número satisfatório que demonstra a  importância desse grupo na composição da sociedade em geral.
É importante também destacar o grande trabalho que os grupos evangélicos têm feito no resgate, evangelização e ressocialização de dependentes químicos no Estado do Rio de Janeiro.
Diante do exposto, considerando tudo o que já foi realizado, e continuará sendo feito em prol dos desfavorecidos socialmente é muito importante a inclusão no calendário oficial do Estado do Rio de Janeiro da Semana Estadual do Trabalho Social Evangélico. Será um momento de valorização das ações sociais do grupo, assim como de divulgação de todo o trabalho para outras pessoas que queiram participar. Sendo assim conto com o apoio de todos os meus pares para aprovação desta proposição.



\iffalse
\begin{center}
  \textbf{REFERÊNCIAS}
\end{center}


\fi



\end{document}

