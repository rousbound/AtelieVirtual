\documentclass[10pt]{article}
\usepackage[portuguese]{babel}
\usepackage[utf8]{inputenc}
\usepackage[pdftex]{graphicx}
\usepackage{venndiagram}
\usepackage{subcaption}
\usepackage{caption}
\usepackage[backend=biber,style=authoryear-ibid]{biblatex}
\usepackage[normalem]{ulem}
\usepackage[margin=0.8in]{geometry}
\addbibresource{psychoanalysis.bib}
\graphicspath{{Pictures/}}
\usepackage{tikz}
\usepackage{setspace}
\usepackage{enumitem}
\usepackage{textcomp}
\usepackage{hyperref}


\date{}

\newcommand{\quotebox}[3]{
  \begin{center}
\noindent\fbox{ 
  \parbox{#3\textwidth}{%
  {\itshape#1\itshape}

  \raggedleft {\textbf{#2}} 
    }%  
}
\end{center}
}

\newcommand{\spawnfig}[3]
{
  \begin{figure}[h]
  \centering
  \includegraphics[scale={#3}]{#1}
  \caption{#2}
  \end{figure}
}


\begin{document}
\maketitle
\begin{center}
  % Deputado ELIOMAR COELHO
  \huge
  \vspace{-3cm}\href{http://alerjln1.alerj.rj.gov.br/scpro1923.nsf/f4b46b3cdbba990083256cc900746cf6/e591c996eeafe95e03258504004f3c26?OpenDocument}{PROJETO DE LEI Nº 1801/2020}
\bigskip
\bigskip
\bigskip
  
\end{center}

\textbf{EMENTA:} 
MODIFICA O ANEXO DA LEI Nº 5.645/2010








\bigskip

\noindent
A ASSEMBLEIA LEGISLATIVA DO ESTADO DO RIO DE JANEIRO RESOLVE:

\begin{enumerate}[label=Art. \arabic*\textdegree]

\item - Fica incluído no anexo da Lei n°5.645, de 6 de janeiro de 2010, que consolida a legislação das datas comemorativas do Calendário Oficial do Estado do Rio de Janeiro, o &#``Dia do Esperanto&#", a ser comemorado, anualmente, no dia 15 de dezembro.

\item - O anexo da Lei n° 5.645, de 6 de janeiro de 2010, passa a ter a seguinte redação:
&#``ANEXO CALENDÁRIO DE DATAS COMEMORATIVAS DO ESTADO DO RIO DE JANEIRO: 
(&#8230;)
DEZEMBRO
(&#8230;)

15 de dezembro - &#``Dia do Esperanto&#".
(&#8230;) (NR)&#"

\item - Esta Lei entra em vigor na data de sua publicação.


\end{enumerate}




\begin{center}
  Plenário Barbosa Lima Sobrinho, 04 de fevereiro de 2020.

   \bigskip

  \textbf{ ELIOMAR COELHO}

  \bigskip

  \textbf{JUSTIFICATIVA}
  \bigskip

\end{center}

  A ideia base do ESPERANTO foi lançada por Dr. Lázaro Luiz Zamenhof, um médico polonês, na época com 28 anos, em 1887, há 130 anos. Desde aquela época o projeto da língua planejada vem se tornando uma língua viva, com uma cultura própria e internacional e em alguns lugares já com falantes nativos. O ESPERANTO é uma Língua Neutra Internacional conhecida em todo o mundo, se faz presente em todas as áreas do conhecimento humano, é planejada, de fácil aprendizado e que tem por finalidade servir de meio de comunicação entre pessoas que falam idiomas diferentes; ser a segunda língua de cada povo.
Comparado às outras línguas, o ESPERANTO é mais fácil de se aprender por causa da sua gramática regular e planejada, com pronúncia totalmente fonética (cada letra representa um único som e cada som é representado por uma única letra). Possui vocabulário internacional baseado nos principais idiomas modernos (inglês, francês e italiano) e no latim.
Apresenta um sistema regular de formação de novas palavras por acréscimos de afixos (prefixos e sufixos).
Com o uso do ESPERANTO surge uma nova forma de relacionamento entre povos de línguas diferentes, baseada no respeito mútuo, sem hegemonia de uma língua nacional imposta pela força econômica, política ou outra qualquer.
A diversidade linguística é patrimônio cultural da humanidade. O esperanto é uma ferramenta adequada para proteção e difusão da diversidade cultural porque promove intercâmbio igualitário entre os povos, preservando suas línguas e culturas.
Quem fala o esperanto fala com o mundo. O esperanto interage com todos os continentes, sendo um veículo de comunicações que atende as exigências do mundo moderno.
Falando esperanto tem-se facilidade de fazer amigos em todo mundo, participar de eventos internacionais sem uso de intérpretes, não precisa aprender várias línguas para visitar muitos países, pois neles encontrará também falantes de esperanto. Com pouco tempo de aprendizagem, o esperanto permite contatos diretos e intensos com pessoas de outros países. Como cidadão do mundo, quem fala esperanto supera fronteiras, promove intercâmbio com pessoas de línguas diferentes, o que possibilita viagens internacionais com vivências inesquecíveis.
\item - rias universidades brasileiras já ensinam o esperanto: Universidade de Brasília&#-UNB; Universidade Federal do Ceará; Universidade de Campinas; Universidade do Espirito Santo e mais recentemente a Universidade Federal Fluminense &#- UFF através de seu departamento PROJETO DE LÍNGUAS ESTRANGEIRAS MODERNAS &#- PROLEM, Niterói/RJ.
Já foi proposto um Projeto de lei (n° 6.162/2009), de autoria do Senado Federal (Cristovam Buarque), permitindo o ensino do esperanto nas escolas de nível médio.
Muitas instituições particulares ensinam o esperanto aos seus associados, como por exemplo a Cooperativa Cultural dos Esperantistas, na Avenida Treze de Maio, 23, S/L 108, Centro, Rio de Janeiro, onde existem cursos em vários níveis e horários, palestras e uma vasta biblioteca.
A rádio Rio de Janeiro, semanalmente às terças-feiras, das 15:00 horas às 16 horas, tem um programa dedicado ao esperanto; a rádio Vaticano tem duas transmissões semanais no idioma internacional, o esperanto.
Emissoras de rádio de Cuba, Polônia, China e de outros países também fazem regularmente transmissões em esperanto.
Os especialistas comemoram o esperanto no dia 15 de dezembro, data de nascimento do iniciador do Esperanto- Lázaro Luiz Zemenhof.
Assim, o dia 15 de dezembro será incluído no Calendário Oficial do Estado do Rio de Janeiro, como o &#``Dia do Esperanto&#".



\iffalse
\begin{center}
  \textbf{REFERÊNCIAS}
\end{center}


\fi



\end{document}

