\documentclass[10pt]{article}
\usepackage[portuguese]{babel}
\usepackage[utf8]{inputenc}
\usepackage[pdftex]{graphicx}
\usepackage{venndiagram}
\usepackage{subcaption}
\usepackage{caption}
\usepackage[backend=biber,style=authoryear-ibid]{biblatex}
\usepackage[normalem]{ulem}
\usepackage[margin=0.8in]{geometry}
\addbibresource{psychoanalysis.bib}
\graphicspath{{Pictures/}}
\usepackage{tikz}
\usepackage{setspace}
\usepackage{enumitem}
\usepackage{textcomp}
\usepackage{hyperref}


\date{}

\newcommand{\quotebox}[3]{
  \begin{center}
\noindent\fbox{ 
  \parbox{#3\textwidth}{%
  {\itshape#1\itshape}

  \raggedleft {\textbf{#2}} 
    }%  
}
\end{center}
}

\newcommand{\spawnfig}[3]
{
  \begin{figure}[h]
  \centering
  \includegraphics[scale={#3}]{#1}
  \caption{#2}
  \end{figure}
}


\begin{document}
\maketitle
\begin{center}
  %  PODER EXECUTIVO
  \huge
  \vspace{-3cm}\href{http://alerjln1.alerj.rj.gov.br/scpro1923.nsf/f4b46b3cdbba990083256cc900746cf6/953b6aff197410c1032585a6005ec982?OpenDocument}{PROJETO DE LEI Nº 2884/2020}
\bigskip
\bigskip
\bigskip
  
\end{center}

\textbf{EMENTA:} 
REGULAMENTA O INCISO II, ART. 24-I DO DECRETO-LEI Nº 667, DE 02 DE JULHO DE 1969, ACRESCENTADO PELA LEI 13.954 DE 16 DE DEZEMBRO DE 2019, DISPONDO SOBRE OS REQUISITOS PARA O INGRESSO DE MILITARES TEMPORÁRIOS VOLUNTÁRIOS NO CORPO DE BOMBEIROS MILITAR DO ESTADO DO RIO DE JANEIRO E DÁ OUTRAS PROVIDÊNCIAS








\bigskip

\noindent
A ASSEMBLEIA LEGISLATIVA DO ESTADO DO RIO DE JANEIRO RESOLVE:

\begin{enumerate}[label=Art. \arabic*\textdegree]
\item - Esta lei dispõe sobre o Serviço Militar Temporário Voluntário (SMTV), que consiste no exercício de atividades específicas, desempenhadas no Corpo de Bombeiros Militar do Estado do Rio de Janeiro (CBMERJ), por prazo determinado e destina-se a completar os Quadros de Oficiais e as diversas Qualificações de Bombeiros Militares Particulares de praças. 
§1º Os Militares Temporários Voluntários somente poderão exercer funções nas fileiras do CBMERJ e em atividade  de bombeiro militar.
§2º A complementação total de militares temporários não poderá ser superior a 50% (cinquenta porcento) do efetivo previsto.
§3º Para ingresso no Serviço Militar Temporário Voluntário (SMTV) será exigida a idade mínima de 18 (dezoito) e máxima de 25 (vinte e cinco) anos, para homens e mulheres.
\item - As condições de seleção, matrícula, incorporação, estágio, prorrogação e exclusão dos Oficiais Temporários Voluntários (OTV) e das Praças Temporárias Voluntárias (PTV) será regulamentada pelo Comando-Geral do Corpo de Bombeiros Militar de acordo com as necessidades da Instituição.
§1º O ingresso para o Serviço Militar Temporário Voluntário será mediante processo seletivo simplificado.
§2º Os requisitos mínimos necessários para ingresso em cada área de atuação do CBMERJ serão definidos no edital do respectivo processo seletivo simplificado.
\item - O Serviço Militar Temporário Voluntário terá a duração de 12 (doze) meses.
§1º Aos militares temporários que concluírem com aproveitamento o tempo de serviço estipulado no caput, poderão requerer a prorrogação deste tempo, uma ou mais vezes, desde que não ultrapasse a duração máxima de 08 (oito) anos no serviço ativo, incluído eventual serviço militar obrigatório, segundo critério e conveniência da corporação.
§2º A contagem do tempo de Serviço Militar Temporário terá início no dia da incorporação.
\item - Os Oficiais Temporários Voluntários (OTV) e as Praças Temporárias Voluntárias (PTV), tanto quanto possível e respeitado o interesse público, serão lotados em Organização de Bombeiro Militar (OBM) localizado no Município de sua residência, para cumprimento do tempo inicial, definido no caput do artigo 3º desta lei.
Parágrafo único. Nos casos de prorrogação do tempo de serviço militar temporário, a critério da conveniência e oportunidade da Instituição, os incorporados poderão servir em qualquer Organização de Bombeiro Militar, indistintamente do Município de sua residência.
\item - Durante o período inicial do Serviço Militar Temporário Voluntário, as PraçasTemporárias Voluntárias terão direito a remuneração, conforme previsto na lei de remunaração dos militares do Estado, aplicando a estes o escalonamento de 125 ao soldo.
§1º Poderá ser utilizado até o percentual limite de 15% (quinze por cento) do montante dos recursos financeiros constituintes da receita do FUNESBOM, para pagamento de despesas de pessoal referentes ao Serviço Militar Temporário Voluntário. 
§2º Na hipótese de prorrogação do Serviço Militar Temporário Voluntário, as Praças passarão a ter direito a remuneração escalonada, não superiores a de um Bombeiro Militar de carreira de mesma classe ou nível e escala hierárquica.
\item - Os Oficiais Temporários terão direito à remuneração não superiores a de um Bombeiro Militar de carreira de mesma classe ou nível e escala hierárquica.
\item - O art. 1º da Lei nº 622, de 02 de dezembro de 1982, alterado pelo art. 1º da Lei 5996, de 29 de junho de 2011, passa a vigorar com a seguinte redação:



&#``Art. 1º Fica criado o Fundo Especial do Corpo de Bombeiros do Estado do Rio de Janeiro - FUNESBOM destinado à aplicação de recursos financeiros para reequipamento material, realizações ou serviços, inclusive programas de ensino, de assistência médico-hospitalar e de assistência social, do Corpo de Bombeiros Militar do Estado do Rio de Janeiro, investimentos em equipamentos e projetos de prevenção e combate de incêndios nas cidades e reservas ecológicas, incluindo as áreas da mata atlântica, e manutenção dos órgãos e serviços da Secretaria de Estado de Defesa Civil, voltados prioritariamente para atividades de capacitação e atualização de recursos humanos, desenvolvimento de programas de valorização e motivação profissional, bem como para pagamento de despesas de pessoal referentes a gratificações e ao serviço militar temporário.
Parágrafo único. Fica assegurado exclusivamente para a manutenção, reequipamento e o custeio da Secretaria de Estado de Defesa Civil e do Corpo de Bombeiros Militar do Estado do Rio de Janeiro o percentual de 60% (sessenta por cento) do montante dos recursos financeiros constituintes da receita do FUNESBOM.&#"



\item - O militar temporário, licenciado ex offício por término de prorrogação de tempo de serviço, fará jus à compensação pecuniária equivalente a 01 (uma) remuneração mensal por ano de efetivo serviço militar prestado, tomando-se como base de cálculo o valor da remuneração correspondente ao posto ou graduação, na data de pagamento da referida compensação.
\item - Os militares temporários não adquirem estabilidade e passam a compor a reserva não remunerada do CBMERJ após serem desligados do serviço ativo.
\item - O Poder Executivo Estadual editará normas complementares necessários ao cumprimento desta lei.
\item - As despesas decorrentes da aplicação desta Lei correrão à conta do orçamento do Poder Executivo do Estado do Rio de Janeiro.
\item - Esta Lei entra em vigor na data de sua publicação.


WILSON WITZEL
Governador


\end{enumerate}




\begin{center}
  

   \bigskip

  \textbf{  PODER EXECUTIVO}

  \bigskip

  \textbf{JUSTIFICATIVA}
  \bigskip

\end{center}

  MENSAGEM Nº  28 / 2020           


EXCELENTÍSSIMOS SENHORES PRESIDENTE E DEMAIS MEMBROS DA ASSEMBLEIA LEGISLATIVA DO ESTADO DO RIO DE JANEIRO
Tenho a honra de submeter à deliberação de Vossas Excelências o incluso Projeto de Lei que &#``REGULAMENTA O INCISO II DO ART. 24-I DO DECRETO-LEI Nº 667, DE 02 DE JULHO DE 1969, ACRESCENTADO PELA LEI 13.954, DE 16 DE DEZEMBRO DE 2019, DISPONDO SOBRE OS REQUISITOS PARA O INGRESSO DE MILITARES TEMPORÁRIOS VOLUNTÁRIOS NO CORPO DE BOMBEIROS MILITAR DO ESTADO DO RIO DE JANEIRO E DÁ OUTRAS PROVIDÊNCIAS&#".
O presente Projeto de Lei é fruto de estudos realizados pelo Corpo de Bombeiros do Estado do Rio de Janeiro, e tem por objetivo regulamentar o inciso II do art. 24-I do Decreto-Lei nº 667, de 02 de julho de 1969, acrescentado pela Lei no 13.954, de 16 de dezembro de 2019, no que tange aos requisitos para o ingresso de militares temporários nos Estados.
O Corpo de Bombeiros Militar do Estado do Rio de Janeiro está num processo constante de redução do seu efetivo, proveniente da passagem para a reserva remunerada de seus militares, porém sem a devida reposição de seus quadros, pela limitação de realização de novos concursos públicos, em virtude da crise econômico-financeira que assolou o nosso Estado fluminense.
Busca-se, portanto, suprir as necessidades atuais de melhor gerir os quadros de pessoal, paralelamente com a responsabilidade de se evitar a incorporação de grandes volumes de efetivos militares com estabilidade, assim diminuindo o impacto previdenciário futuro, obtendo também a constante renovação da tropa, oportunizando a mais cidadãos, especialmente os mais jovens, a obterem uma qualificação profissional que muito lhes será útil em sua formação profissional, facilitando-se assim uma futura empregabilidade no setor privado. 
Vislumbra-se, outrossim, a aplicação de uma política de valorização profissional, por meio de regras para a continuidade do militar no serviço ativo, evitando-se que os postos e/ou graduações do topo da carreira fiquem inchados e menos eficientes.  
Por isso, este projeto de lei torna-se fundamental para que o Corpo de Bombeiros continue atuando com qualidade e eficiência nas diversas atribuições que possui, além de permitir que o cidadão adquira uma oportunidade de trabalho, na busca de uma requalificação e/ou reinclusão no mercado profissional, diante do grande índice de desemprego no Estado.
Importante consignar que a proposta, tal como apresentada, visa tão somente introduzir um novo modelo de gestão de pessoal, na medida em que não cria ou mesmo transforma cargos, e sim, a utiliza-se do número de cargos já existentes, sem aumento de efetivo e sem aumento de despesas.
Dessa forma, considerando o relevante interesse público da matéria, esperamos contar, mais uma vez, com o apoio e o respaldo dessa Egrégia Casa e solicitando que seja atribuído ao processo o regime de urgência, nos termos do artigo 114 da Constituição do Estado, reitero a vossas Excelências o protesto de elevada estima e consideração.  

WILSON WITZEL
Governador



\iffalse
\begin{center}
  \textbf{REFERÊNCIAS}
\end{center}


\fi



\end{document}

