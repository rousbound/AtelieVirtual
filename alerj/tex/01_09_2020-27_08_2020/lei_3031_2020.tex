\documentclass[10pt]{article}
\usepackage[portuguese]{babel}
\usepackage[utf8]{inputenc}
\usepackage[pdftex]{graphicx}
\usepackage{venndiagram}
\usepackage{subcaption}
\usepackage{caption}
\usepackage[backend=biber,style=authoryear-ibid]{biblatex}
\usepackage[normalem]{ulem}
\usepackage[margin=0.8in]{geometry}
\addbibresource{psychoanalysis.bib}
\graphicspath{{Pictures/}}
\usepackage{tikz}
\usepackage{setspace}
\usepackage{enumitem}
\usepackage{textcomp}
\usepackage{hyperref}


\date{}

\newcommand{\quotebox}[3]{
  \begin{center}
\noindent\fbox{ 
  \parbox{#3\textwidth}{%
  {\itshape#1\itshape}

  \raggedleft {\textbf{#2}} 
    }%  
}
\end{center}
}

\newcommand{\spawnfig}[3]
{
  \begin{figure}[h]
  \centering
  \includegraphics[scale={#3}]{#1}
  \caption{#2}
  \end{figure}
}


\begin{document}
\maketitle
\begin{center}
  %  PODER EXECUTIVO
  \huge
  \vspace{-3cm}\href{http://alerjln1.alerj.rj.gov.br/scpro1923.nsf/f4b46b3cdbba990083256cc900746cf6/9d2ef1aac02a86a5032585cb005dda58?OpenDocument}{PROJETO DE LEI Nº 3031/2020}
\bigskip
\bigskip
\bigskip
  
\end{center}

\textbf{EMENTA:} 
ALTERA A LEI Nº 5.240, DE 14 DE MAIO DE 2008, QUE &#``INSTITUI O CONSELHO ESTADUAL DE TRABALHO, EMPREGO E RENDA NO ESTADO DO RIO DE JANEIRO&#"








\bigskip

\noindent
A ASSEMBLEIA LEGISLATIVA DO ESTADO DO RIO DE JANEIRO RESOLVE:

\begin{enumerate}[label=Art. \arabic*\textdegree]
\item - O art. 1° da Lei n° 5.240, de 14 de maio de 2008, passa a vigorar com a seguinte redação:

&#``Art. 1º Fica instituído o Conselho Estadual do Trabalho, Emprego e Renda - CETER/RJ, órgão colegiado que deliberará, em caráter permanente, sobre as políticas públicas de fomento e apoio à geração de trabalho, emprego e renda e à qualificação profissional no Estado do Rio de Janeiro.
Parágrafo único. O CETER/RJ ficará vinculado à Secretaria de Estado de Trabalho e Renda - SETRAB e, em caso de alteração de estrutura do Poder Executivo, à Secretaria de Estado responsável pelas políticas públicas referentes ao fomento e apoio à geração de trabalho, emprego e renda e à qualificação profissional. &#``(NR)

\item - O §3º do art. 2º da Lei n° 5.240, de 14 de maio de 2008, passa a vigorar com a seguinte redação:

&#``Art. 2° (...)
(...)
§3º O Conselho poderá criar Grupos Técnicos para assessoramento dos Conselheiros nos assuntos de sua competência, na forma da Resolução CODEFAT em vigor.&#" (NR)

\item - O art. 3° da Lei n° 5.240, de 14 de maio de 2008, passa a vigorar com a seguinte redação:

&#``Art. 3º O Conselho Estadual do Trabalho, Emprego e Renda terá as seguintes atribuições:
\begin{enumerate}[label=\Roman*]
\item - deliberar e definir acerca da Política de Trabalho, Emprego e Renda, no âmbito do Estado do Rio de Janeiro, em consonância com a Política Nacional de Trabalho, Emprego e Renda, possibilitando ações coordenadas entre as esferas administrativas;
\item - apreciar e aprovar o plano de ações e serviços do SINE, na forma estabelecida pelo CODEFAT, bem como a proposta orçamentária da Política de Trabalho, Emprego e Renda, e suas alterações, a ser encaminhada pelo órgão da Administração Pública Estadual, responsável pela coordenação da Política de Trabalho, Emprego e Renda;
\item - acompanhar, controlar e fiscalizar a execução da Política de Trabalho, Emprego e Renda, conforme normas e regulamentos estabelecidos pelo CODEFAT e pelo Ministério da Economia;
\item - orientar e controlar o Fundo do Trabalho, incluindo sua gestão patrimonial, inclusive a recuperação de créditos e a alienação de bens e direitos;
\item - aprovar seu Regimento Interno, observando-se os critérios definidos pelo CODEFAT;
\item - exercer a fiscalização dos recursos financeiros destinados ao SINE, depositados em conta especial de titularidade do Fundo do Trabalho;
\item - apreciar e aprovar relatório de gestão anual que comprove a execução das ações do SINE, quanto à utilização dos recursos federais descentralizados para os fundos do trabalho das esferas de governo que a ele aderirem;
\item - aprovar a prestação de contas anual do Fundo do Trabalho;
\item - baixar normas complementares necessárias à gestão do Fundo do Trabalho;
\item - deliberar sobre outros assuntos de interesse do Fundo do Trabalho;
\end{enumerate}

\item - propor aos órgãos públicos e entidades não governamentais programas, projetos e medidas efetivas que visem a minimizar os impactos negativos do desemprego conjuntural e estrutural no Estado;

\end{enumerate}
\item - elaborar e apreciar projetos de geração de trabalho, emprego e renda e de qualificação profissional no Estado;

\item - incentivar a instituição de Conselhos Municipais de Trabalho pelas Câmaras de Vereadores, homologá-los e assessorá-los;
\item - propor programas, projetos e medidas que incentivem o associativismo, o cooperativismo e a auto-organização como forma de enfrentar o impacto do desemprego nas áreas urbana e rural do Estado.
\item - acompanhar e fiscalizar a aplicação dos recursos públicos utilizados na geração de trabalho, emprego e renda e na qualificação profissional no Estado, priorizando os oriundos do Fundo de Amparo ao Trabalhador - FAT;
 XVI- propor os objetivos, as regras, os critérios e as metas para planos de qualificação profissional no Estado e acompanhar sua execução, garantindo sua interiorização e transparência por meio dos Conselhos e Comissões Municipais de Emprego;
\item - formular as propostas relacionadas com as políticas públicas de geração de trabalho, emprego e renda e de qualificação profissional;
\item - formular a proposta de piso regional de salários;
\item - elaborar projetos que gerem empregos, desenvolvam habilidades e qualifiquem profissionalmente os cidadãos do Estado do Rio de Janeiro;
\item - fomentar ações de qualificação social e profissional ao trabalhador, sem ônus para o mesmo;
\item - apresentar propostas de fiscalização quanto ao correto recolhimento das contribuições previdenciárias ao INSS e em relação ao recolhimento do FGTS;
\item - propor ações de microcrédito produtivo e outras medidas que beneficiem os micro e pequenos empreendimentos, inclusive os informais;
\item - garantir que os recursos do Fundo Estadual do Trabalho sejam aplicados no: 
a)	financiamento do SINE;
b)	financiamento do total ou parcial de programas, ações e atividades previstos no Plano Estadual de Ações e Serviços pactuado no âmbito do SINE;
c)	fomento ao trabalho, emprego e renda, nas ações previstas no art. 9º da Lei Federal 13.667/18, nos termos do art. 8º, sem prejuízo de outras atribuídas pelo CODEFAT;
d)	pagamento das despesas com o funcionamento do Conselho do Trabalho, Emprego Renda, envolvendo custeio, manutenção e pagamento das despesas conexas aos objetivos do Fundo, exceto as de pessoal;
e)	pagamento pela prestação de serviços às entidades conveniadas, públicas ou privadas, para a execução de programas e projetos específicos na área do trabalho;
f)	pagamento de subsídio à pessoa física beneficiária de programa ou projeto da política pública de trabalho, emprego e renda;
g)	aquisição de material permanente e de consumo e de outros insumos e serviços necessários ao desenvolvimento dos programas e projetos relacionados à Política Estadual de Trabalho, Emprego Renda;
h)	reforma, ampliação, de imóvel público, aquisição ou locação de imóveis para prestação de serviços de atendimento ao trabalhador;
i)	desenvolvimento e aperfeiçoamento dos instrumentos de gestão, planejamento, administração e controle das ações e serviços no âmbito da política estadual de trabalho, emprego e renda;
j)	custeio, manutenção e pagamento das despesas conexas aos objetivos do Fundo, no desenvolvimento de ações, serviços, programas afetos ao SINE;
k)	financiamento de ações, programas e projetos previstos nos Planos Municipais de Ações e Serviços da área trabalho;
l)	prestar assistência para fins de garantia de empregabilidade para pessoas em vulnerabilidade social;
m)	estímulo aos Municípios e aos consórcios que eles venham a constituir, fornecendo-lhes suporte técnico e financeiro, para viabilização das ações e serviços do SINE;
n)	financiamento total ou parcial de programas, ações e projetos de qualificação e educação profissional; e 
o)	demais ações previstas na Resolução n° 831, de 21 de maio de 2019 e suas posteriores alterações.
Parágrafo único. A aplicação dos recursos do FT/RJ depende de prévia aprovação do Conselho do Trabalho, Emprego e Renda - CETER/RJ.&#" (NR)





\item - O art. 4° da Lei n° 5.240, de 14 de maio de 2008, passa a vigorar com a seguinte redação:

&#``Art. 4º O Conselho Estadual do Trabalho, Emprego e Geração de Renda será composto por, no mínimo, 09 (nove) membros e, no máximo 18 (dezoito) membros, que representarão paritariamente os trabalhadores, os empregadores e o Poder Executivo, da seguinte forma:
\begin{enumerate}[label=\Roman*]
\item - pelos trabalhadores, os seis membros e seus respectivos suplentes serão indicados pelas centrais sindicais que atenderem aos requisitos de representatividade de que trata o art. 2° da Lei n° 11.648, de 31 de março de 2008, observado o disposto no art. 3° da referida Lei, representada da seguinte forma:
a)  Central Única dos Trabalhadores - CUT ;
b) União Geral dos Trabalhadores - UGT;
c) Central dos Trabalhadores e Trabalhadoras do Brasil - CTB; 
d) Força Sindical - FS;
e)  Nova Central Sindical dos Trabalhadores - NCST; e
f) Central dos Sindicatos Brasileiros - CSB.
\item - pelos empregadores, por um representante de cada uma das seguintes entidades:
a) Federação da Agricultura do Estado do Rio de Janeiro - FAERJ; 
b) Federação das Indústrias do Estado do Rio de Janeiro - FIRJAN;
c) Federação do Comércio do Estado do Rio de Janeiro - FECOMERCIO; 
d) Federação dos Hospitais e Estabelecimentos de Serviços de Saúde do Estado do Rio de Janeiro - FEHERJ;
e) Federação das Empresas de Transportes de Passageiros do Estado do Rio de Janeiro - FETRANSPOR; e
f) Associação Comercial do Estado do Rio de Janeiro - ACRJ. 
\item - pelo Poder Público, por um representante de cada um dos seguintes órgãos:
a) Superintendência Regional do Trabalho no Rio de Janeiro - SRTb/RJ;
b) Secretaria de Estado da Casa Civi;
c) Secretaria de Estado de Desenvolvimento Econômico, Energia e Relações Internacionais;
d) Secretaria de Estado de Agricultura, Pecuária, Pesca e Abastecimento;
e) Secretaria de Estado de Ciência, Tecnologia e Inovação; e
f) Secretaria de Estado de Trabalho e Renda.

§1º (...)
§2º (...)
§3º Cada representante efetivo terá um suplente e seus mandatos seguirão a periodicidade determinada pela Resolução CODEFAT em vigor. 
§4º Os membros do Conselho não são remunerados e serão nomeados pelo Secretário Estadual responsável pelas políticas públicas relacionadas ao Trabalho, Emprego e Renda, observados obrigatoriamente os nomes dos titulares e suplentes enviados pelos órgãos e pelas respectivas entidades representantes dos trabalhadores e empregadores.
§5º A Presidência e Vice-Presidência do CETER-RJ, eleitas a cada dois anos por maioria absoluta dos seus representantes, serão alternadas entre as representações dos trabalhadores, dos empregadores e do governo, e exercidas pelos representantes da Secretaria Estadual responsável pelo tema de Trabalho, Emprego e Renda ou pela Superintendência Regional do Trabalho no Rio de Janeiro, quando couber a representação ao Governo, vedada a recondução do presidente do vice-presidente para período consecutivo de mandato. (NR)
§6° No caso de vacância da presidência caberá ao Colegiado realizar eleição de um novo presidente para completar o mandato do antecessor, dentre os membros da mesma bancada, garantindo o sistema de rodízio, assegurando a continuidade da atuação do vice-presidente até o final de seu mandato.&#" 

\item - Acrescenta o §7° ao art. 4° da Lei n° 5.240, de 14 de maio de 2008, com a seguinte redação:

&#``§7º A substituição e reposição das entidades que integram o Conselho, bem como eventuais formas de votação em casos extraordinários, observará o dispostos nas regras previstas no Regimento Interno, observando a legislação vigente.&#" (NR)

\item - O art. 5° da Lei n° 5.240, de 14 de maio de 2008, passa a vigorar com a seguinte redação:

&#``Art. 5° O Conselho Estadual do Trabalho, Emprego e Renda promoverá uma conferência, a realizar-se preferencialmente no mês de maio, na qual serão empossados o Presidente e o Vice-Presidente, e para a qual são convocadas as entidades envolvidas no processo de geração de emprego e renda.&#" (NR)

\item - O art. 6° da Lei n° 5.240 de 14 de maio de 2008, passa a vigorar com a seguinte redação:

&#``Art. 6º O Conselho Estadual do Trabalho, Emprego e Renda terá uma Secretaria Executiva, à qual competirão as ações de cunho operacional e o suporte administrativo.
Parágrafo único. A Secretaria Executiva do Conselho será exercida pela equipe designada pelo Secretário de Estado titular da Pasta que trata das políticas públicas relacionadas ao trabalho, emprego e renda.&#" (NR)

\item - Com o objetivo de evitar a interrupção das atividades do Conselho, o mandato dos seus membros se encerrará em maio de 2023, resguardadas as normas previstas na Resolução CODEFAT n° 831/2019 e suas posteriores alterações. 
\item - O Conselho promoverá a adequação de seu regimento interno no prazo de noventa dias, a contar da publicação desta Lei.
\item - Esta Lei entrará em vigor na data de sua publicação.


WILSON WITZEL
Governador


\end{enumerate}




\begin{center}
  

   \bigskip

  \textbf{  PODER EXECUTIVO}

  \bigskip

  \textbf{JUSTIFICATIVA}
  \bigskip

\end{center}

  MENSAGEM Nº 32  / 2020          


EXCELENTÍSSIMOS SENHORES PRESIDENTE E DEMAIS MEMBROS DA ASSEMBLEIA LEGISLATIVA DO ESTADO DO RIO DE JANEIRO
Honra-me submeter à elevada deliberação dessa Egrégia Casa o incluso Projeto de Lei que "ALTERA A LEI Nº 5.240, DE 14 DE MAIO DE 2008, QUE &#``INSTITUI O CONSELHO ESTADUAL DE TRABALHO, EMPREGO E RENDA NO ESTADO DO RIO DE JANEIRO&#".
Inicialmente, cumpre ressaltar que a aprovação do presente Projeto de Lei esta alicerçada na necessidade de aperfeiçoamento das garantias, direitos e deveres dos trabalhadores previstos na Lei Federal n° 7.998, de 11 de janeiro de 1999, que regula o Programa do Seguro-Desemprego e o abono de que tratam o inciso II do art. 7º, o inciso IV do art. 201 e o art. 239, da Carta Magna, instituindo o Fundo de Amparo ao Trabalhador - FAT. 
Dentro desta perspectiva, a implementação da medida não ocasionará impactos financeiros, sendo certo que eventual não aprovação, certamente acarretará ineficiência e ineficácia decorrentes da não utilização do FAT de aproximadamente três milhões de reais, o que impedirá o crescimento econômico do Estado do Rio de Janeiro por conta da não mitigação de 1.4 milhões de desempregados segundo o IBGE em 2019.
Cumpre repisar, que as alterações propostas contribuirão significativamente para o fomento de uma lei justa e abrangente, o que viabilizará que o Estado do Rio de Janeiro acolha todas as pluralidades do ecossistema trabalhista.
Por fim, cabe ressaltar que tais alterações, possibilitarão que o Fundo do Trabalhador do Estado do Rio de Janeiro possa receber receitas federais provenientes do FAT, fundamentais para a gestão da política de emprego e geração de renda fluminense.
Assim, considerando o relevante interesse público da matéria, esperamos contar, mais uma vez, com o apoio e o respaldo dessa Egrégia Casa e solicitando que seja atribuído ao processo o regime de urgência, nos termos do artigo 114 da Constituição do Estado, reitero a vossas Excelências o protesto de elevada estima e consideração.


WILSON WITZEL
Governador



\iffalse
\begin{center}
  \textbf{REFERÊNCIAS}
\end{center}


\fi



\end{document}

