\documentclass[10pt]{article}
\usepackage[portuguese]{babel}
\usepackage[utf8]{inputenc}
\usepackage[pdftex]{graphicx}
\usepackage{venndiagram}
\usepackage{subcaption}
\usepackage{caption}
\usepackage[backend=biber,style=authoryear-ibid]{biblatex}
\usepackage[normalem]{ulem}
\usepackage[margin=0.8in]{geometry}
\addbibresource{psychoanalysis.bib}
\graphicspath{{Pictures/}}
\usepackage{tikz}
\usepackage{setspace}
\usepackage{enumitem}
\usepackage{textcomp}
\usepackage{hyperref}


\date{}

\newcommand{\quotebox}[3]{
  \begin{center}
\noindent\fbox{ 
  \parbox{#3\textwidth}{%
  {\itshape#1\itshape}

  \raggedleft {\textbf{#2}} 
    }%  
}
\end{center}
}

\newcommand{\spawnfig}[3]
{
  \begin{figure}[h]
  \centering
  \includegraphics[scale={#3}]{#1}
  \caption{#2}
  \end{figure}
}


\begin{document}
\maketitle
\begin{center}
  % Deputado DELEGADO CARLOS AUGUSTO
  \huge
  \vspace{-3cm}\href{http://alerjln1.alerj.rj.gov.br/scpro1923.nsf/f4b46b3cdbba990083256cc900746cf6/c5093f8c3ef3b9620325859200535c17?OpenDocument}{PROJETO DE LEI Nº 2799/2020}
\bigskip
\bigskip
\bigskip
  
\end{center}

\textbf{EMENTA:} 
DISPÕE SOBRE A PRIORIDADE PARA O RECEBIMENTO DE FUTURA VACINA CONTRA O VÍRUS COVID-19








\bigskip

\noindent
A ASSEMBLEIA LEGISLATIVA DO ESTADO DO RIO DE JANEIRO RESOLVE:

\begin{enumerate}[label=Art. \arabic*\textdegree]

\item - Fica garantida a prioridade aos Profissionais de Saúde,  Profissionais de Segurança Pública e pessoas vulneráveis para o recebimento de futura vacina contra o vírus da Covid-19 (Novo Corornavírus).

§1º - Os Profissionais de Saúde, mencionados no caput deste artigo, são os médicos, enfermeiros, técnicos de saúde e demais profissionais determinados pela Secretaria de Saúde do Estado do Rio de Janeiro.

§2º - Consideram-se como Profissionais de Segurança Pública, mencionados no caput deste artigo, os seguintes servidores públicos:
\begin{enumerate}[label=\Roman*]
\item - Da Secretaria de Estado de Polícia Civil;
\item - Da Secretaria de Estado de Polícia Militar;
\item - Da Polícia Penitenciária;
\item - Do Corpo de Bombeiro Militar;
\item - Da Defesa Civil;
\item - Do Departamento Geral de Ações Socioeducativas (DEGASE);
\item - Profissionais do Segurança Presente.
\end{enumerate}

§3º - Consideram-se pessoas vulneráveis, mencionadas no caput deste artigo, as seguintes:
\begin{enumerate}[label=\Roman*]
\item - Pessoas Idosas;
\item - Com condições médicas pré-existentes (como pressão alta, doenças cardíacas, doenças pulmonares, câncer ou diabetes);
\item - Pessoas que trabalham ou moram em locais de alta transmissão, como prisões e casas de repouso;
\item - Demais pessoas vulneráveis determinadas pela Secretária de Saúde do Estado do Rio de Janeiro;
\end{enumerate}

\item - O Poder executivo deverá regulamentar a presente lei.

\item - Esta Lei entra em vigor na data de sua publicação .


\end{enumerate}




\begin{center}
  Plenário Barbosa Lima Sobrinho, 25 de junho de 2020.DEPUTADO CARLOS AUGUSTO

   \bigskip

  \textbf{ DELEGADO CARLOS AUGUSTO}

  \bigskip

  \textbf{JUSTIFICATIVA}
  \bigskip

\end{center}

  O presente Projeto de Lei tem o objetivo de garantir a prioridade de recebimento de uma futura vacina contra o vírus da Covid-19 (Novo Corornavírus) aos Profissionais de Saúde, Profissionais de Segurança Pública e pessoas vulneráveis.

A Organização Mundial da Saúde (OMS) espera a produção de milhões de doses da vacina este ano, conforme reportagem abaixo:


 A Organização Mundial da Saúde (OMS) espera que centenas de milhões de doses de uma vacina contra a Covid-19 possam ser produzidas neste ano e dois bilhões de doses até o final de 2021, disse a cientista-chefe Soumya Swaminathan, nesta quinta-feira (18).
A OMS está elaborando planos para ajudar a decidir quem deveria receber as primeiras doses uma vez que uma vacina seja aprovada, afirmou a cientista.
A prioridade seria dada a profissionais da linha de frente, como médicos, pessoas vulneráveis por causa da idade ou outra doença e a quem trabalha ou mora em locais de alta transmissão, como prisões e casas de repouso.


Fonte: https://agenciabrasil.ebc.com.br/internacional/noticia/2020-06/covid-19-oms-espera-producao-de-milhoes-de-doses-de-vacina-neste-ano

Desta forma, é necessário estabelecer que, além das pessoas vulneráveis, os profissionais da linha de frente terão prioridade no recebimento dessas vacinas contra o novo coronavírus, pois os mesmos exercem atividades de alto risco, ininterruptas e de caráter essencial.

No aspecto da constitucionalidade o projeto ora apresentado encontra fundamento no art. 24, XII da Magna Carta que preceitua que os Estados possuem competência concorrente com a União para legislar sobre a temática da saúde.

Além disso, o Supremo Tribunal Federal, em recente decisão, confirmou a competência concorrente dos Estados, Distrito Federal, Municípios e União em ações para combater o COVID-19, conforme abaixo:







MEDIDA CAUTELAR NA AÇÃO DIRETA DE INCONSTITUCIONALIDADE 6.341 DISTRITO FEDERAL
SAÚDE &#- CRISE &#- CORONAVÍRUS &#- MEDIDA PROVISÓRIA &#- PROVIDÊNCIAS &#- LEGITIMAÇÃO CONCORRENTE. Surgem atendidos os requisitos de urgência e necessidade, no que medida provisória dispõe sobre providências no campo da saúde pública nacional, sem prejuízo da legitimação concorrente dos Estados, do Distrito Federal e dos Municípios.







Portanto, não existem óbices jurídicos à implementação das medidas previstas neste projeto de lei de iniciativa parlamentar.

Trata-se de medida necessária que, além de ser socialmente adequada é também constitucional em todos os aspectos formal e material, encontrando respaldo, inclusive, na própria jurisprudência do Pretório Excelso.

Por fim, sendo o tema de extrema relevância e urgência, contamos com a ajuda de nossos pares para a aprovação do presente Projeto de Lei.



\iffalse
\begin{center}
  \textbf{REFERÊNCIAS}
\end{center}


\fi



\end{document}

