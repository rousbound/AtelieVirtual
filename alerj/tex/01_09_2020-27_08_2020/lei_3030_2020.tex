\documentclass[10pt]{article}
\usepackage[portuguese]{babel}
\usepackage[utf8]{inputenc}
\usepackage[pdftex]{graphicx}
\usepackage{venndiagram}
\usepackage{subcaption}
\usepackage{caption}
\usepackage[backend=biber,style=authoryear-ibid]{biblatex}
\usepackage[normalem]{ulem}
\usepackage[margin=0.8in]{geometry}
\addbibresource{psychoanalysis.bib}
\graphicspath{{Pictures/}}
\usepackage{tikz}
\usepackage{setspace}
\usepackage{enumitem}
\usepackage{textcomp}
\usepackage{hyperref}


\date{}

\newcommand{\quotebox}[3]{
  \begin{center}
\noindent\fbox{ 
  \parbox{#3\textwidth}{%
  {\itshape#1\itshape}

  \raggedleft {\textbf{#2}} 
    }%  
}
\end{center}
}

\newcommand{\spawnfig}[3]
{
  \begin{figure}[h]
  \centering
  \includegraphics[scale={#3}]{#1}
  \caption{#2}
  \end{figure}
}


\begin{document}
\maketitle
\begin{center}
  %  PODER EXECUTIVO
  \huge
  \vspace{-3cm}\href{http://alerjln1.alerj.rj.gov.br/scpro1923.nsf/f4b46b3cdbba990083256cc900746cf6/94de1155ed78b055032585cb005cf4ab?OpenDocument}{PROJETO DE LEI Nº 3030/2020}
\bigskip
\bigskip
\bigskip
  
\end{center}

\textbf{EMENTA:} 
DISPÕE SOBRE A COMPOSIÇÃO DE CLASSE DE DOCENTE 1, PREVISTA NA LEI N° 1.614, DE 24 DE JANEIRO DE 1990, E DÁ OUTRAS PROVIDÉNCIAS








\bigskip

\noindent
A ASSEMBLEIA LEGISLATIVA DO ESTADO DO RIO DE JANEIRO RESOLVE:

\begin{enumerate}[label=Art. \arabic*\textdegree]
\item - A classe de Docente I, a que se refere a Lei nº 1.614, de 24 de janeiro de 1990, passa a ser composta de cargos de provimento efetivo de Professor Docente I.
Parágrafo único.  A jornada de trabalho dos servidores ocupantes do cargo de que trata o caput deste artigo poderá ser de 16 (dezesseis) ou 30 (trinta) horas semanais, a critério da Administração e consideradas as necessidades do serviço público.

\item - Não haverá alteração no regime de trabalho dos servidores integrantes da classe de Docente I, que manterão as respectivas jornadas semanais vigentes na data de publicação desta Lei.

\item - Durante a vigência do Regime de Recuperação Fiscal (Lei Complementar 159, de 19 de maio de 2017) sob o qual o Estado do Rio de Janeiro está submetido, poderá ser autorizada a alteração da jornada de trabalho do Professor Docente I submetido ao regime de 16 horas semanais, para 30 horas semanais, desde que sejam oferecidas medidas de compensação com efeitos financeiros nos termos estabelecidos pelo Decreto Federal n.º 9.109, de 27 de julho de 2017.
§1º A alteração de jornada de trabalho de que trata o caput deste artigo deverá ocorrer de forma gradativa, a critério da Administração, considerando-se estritamente o interesse público e a necessidade do serviço.
§2º Será assegurada a proporcionalidade da remuneração aos professores que tiverem a jornada de trabalho ampliada na forma deste artigo. 
§3º A efetivação da medida prevista neste artigo está condicionada à existência de respectiva autorização e de prévia dotação orçamentária, bem como ao integral atendimento do disposto no § 1º do art, 169 da Constituição Federal, na Lei Complementar nº 101, de 04 de maio de 2000, e nas demais normas pertinentes às questões orçamentárias e financeiras e ao controle de gastos com pessoal na Administração Pública Estadual.

\item - O Estado do Rio de Janeiro, por meio da Secretaria de Estado de Educação, regulamentará os procedimentos e critérios para que haja a autorização da jornada de trabalho do Professor Docente I de 16 horas para 30 horas semanais, devendo, necessariamente, observar o seguinte:
\begin{enumerate}[label=\Roman*]
\item - Identificação da necessidade da alteração, considerando-se o interesse público, mediante a identificação de carência de professores nas unidades escolares da Rede SEEDUC;
\item - Priorização das disciplinas que possuam matriz curricular compatível com a carga horária ampliada;
\item - Manifestação de vontade do servidor na alteração da jornada de trabalho;
\end{enumerate}
§1º Considerando que a alteração da jornada de trabalho dos Professores Docentes I ocorrerá de forma gradativa, a SEEDUC deverá estabelecer os critérios para a escolha dos servidores, garantindo a observância do disposto nos incisos I e II deste artigo e no art. 37, caput, da Constituição Federal.  
§2º O disposto no §2º do art. 3º desta Lei deve ser observado em todas as oportunidades em que for possibilitada a alteração da jornada de trabalho dos servidores ocupantes do cargo de Professor Docente I.  

\item - O vencimento-base do Professor Docente I se encontra previsto no Anexo I desta Lei.

\item - O inciso I, do art. 6º, da Lei nº 6027, de 29 de agosto de 2011, passa a vigorar com a seguinte redação:




&#``Art.6º (...)
\begin{enumerate}[label=\Roman*]
\item - Quadro Permanente, composto pelos cargos de Professor Docente I e Professor Inspetor Escolar. (NR)
\item - 
\end{enumerate}





\item - Os quantitativos dos cargos de Professor Docente I e de Professor Inspetor Escolar ficam definidos no Anexo II desta Lei.

\item - Aplica-se aos servidores ocupantes do cargo de Professores Docente I o plano de carreira previsto na Lei nº 1.614, de 24 de janeiro de 1990.

\item - Esta Lei entra em vigor na data de sua publicação, revogadas as disposições em contrário, em especial, os arts. 1º, 2º e 3º, e o Anexo I da Lei nº 6.027, de 29 de agosto de 2011 e a Lei nº 6.794, de 04 de junho de 2014.


WILSON WITZEL
Governador

ANEXO I
 
TABELA DE VENCIMENTO BASE DO CARGO DE PROFESSOR DOCENTE I
 

CARGOJORNADA DE TRABALHOREFVENCIMENTO BASE

PROFESSOR DOCENTE I16 HORAS SEMANAIS31.179,35

41.320,85

51.479,35

61.656,51

71.855,71

82.078,39

92.327,79

30 HORAS SEMANAIS32.211,25

42.476,60

52.773,79

63.105,94

73.479,45

83.896,99

94.364,62

 
 

 
ANEXO II
 

TABELA COM O QUADRO PERMANENTE&#8203;

PROFESSOR DOCENTE I59.350

PROFESSOR INSPETOR ESCOLAR624

 


\end{enumerate}




\begin{center}
  

   \bigskip

  \textbf{  PODER EXECUTIVO}

  \bigskip

  \textbf{JUSTIFICATIVA}
  \bigskip

\end{center}

  MENSAGEM Nº 31 / 2020           


EXCELENTÍSSIMOS SENHORES PRESIDENTE E DEMAIS MEMBROS DA ASSEMBLEIA LEGISLATIVA DO ESTADO DO RIO DE JANEIRO
Tenho a honra de submeter à deliberação de Vossas Excelências o incluso Projeto de Lei que &#``DISPÕE SOBRE A COMPOSIÇÃO DE CLASSE DE DOCENTE I, PREVISTA NA LEI Nº 1.614, DE 24 DE JANEIRO DE 1990, E DÁ OUTRAS PROVIDÊNCIAS&#".
A premência que se reveste o presente Projeto de Lei esta alicerçada na necessidade de adequação da composição do cargo de Professor Docente I, para que passe a ser composta por apenas um cargo público de provimento efetivo (Professor Docente I), cujos ocupantes poderão ser submetidos à jornada semanal de trabalho de 16 (dezesseis) ou 30 (trinta) horas.
Atualmente, esta sendo adotada uma equivocada subdivisão que permite a existência de dois cargos públicos efetivos distintos de Professor Docente I, o que não é razoável uma vez que os servidores ocupantes de tais cargos desempenham exatamente as mesmas atribuições - função de docência, e cumprem exatamente os mesmos requisitos para ingresso no cargo, ao prestarem concurso público. 
Além de objetivar a unificação do cargo de Professor Docente I, a presente iniciativa visa viabilizar que aos servidores ocupantes do cargo de Professor Docente I - 16 horas a possibilidade de, futuramente, após encerrada a vigência do Regime de Recuperação Fiscal, ampliarem a sua jornada de trabalho de 16 (dezesseis) para 30 (trinta) horas semanais, e que  esta ampliação da jornada de trabalho ocorra de forma gradativa, em observância ao interesse da Administração e com o necessário atendimento às normas orçamentárias.
Cumpre repisar, que a possibilidade de ampliação da jornada de trabalho dos Professores Docentes I, só será possível aos servidores que manifestem interesse na denominada "migração", que ocorrerá somente após a vigência do Regime de Recuperação Fiscal, de forma gradativa, considerando-se a "necessidade da Administração" e as "restrições orçamentárias vigentes".
Sendo assim, é forçoso concluir que a presente medida é salutar e necessária para a implementação justa da unificação do cargo de Professor Docente I.
Portanto, considerando o relevante interesse público da matéria, esperamos contar, mais uma vez, com o apoio e o respaldo dessa Egrégia Casa e solicitando que seja atribuído ao processo o regime de urgência, nos termos do artigo 114 da Constituição do Estado, reitero a vossas Excelências o protesto de elevada estima e consideração.



WILSON WITZEL
Governador



\iffalse
\begin{center}
  \textbf{REFERÊNCIAS}
\end{center}


\fi



\end{document}

