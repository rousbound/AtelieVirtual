\documentclass[10pt]{article}
\usepackage[portuguese]{babel}
\usepackage[utf8]{inputenc}
\usepackage[pdftex]{graphicx}
\usepackage{venndiagram}
\usepackage{subcaption}
\usepackage{caption}
\usepackage[backend=biber,style=authoryear-ibid]{biblatex}
\usepackage[normalem]{ulem}
\usepackage[margin=0.8in]{geometry}
\addbibresource{psychoanalysis.bib}
\graphicspath{{Pictures/}}
\usepackage{tikz}
\usepackage{setspace}
\usepackage{enumitem}
\usepackage{textcomp}
\usepackage{hyperref}


\date{}

\newcommand{\quotebox}[3]{
  \begin{center}
\noindent\fbox{ 
  \parbox{#3\textwidth}{%
  {\itshape#1\itshape}

  \raggedleft {\textbf{#2}} 
    }%  
}
\end{center}
}

\newcommand{\spawnfig}[3]
{
  \begin{figure}[h]
  \centering
  \includegraphics[scale={#3}]{#1}
  \caption{#2}
  \end{figure}
}


\begin{document}
\maketitle
\begin{center}
  % Deputados CARLO CAIADO, MÁRCIO PACHECO
  \huge
  \vspace{-3cm}\href{http://alerjln1.alerj.rj.gov.br/scpro1923.nsf/9665df2600e114f703256caa00231316/a2f84c94d7f081bd032584d200686e0b?OpenDocument}{PROJETO DE LEI Nº 368/2019}
\bigskip
\bigskip
\bigskip
  
\end{center}

\textbf{EMENTA:} 
CONCEDE O TÍTULO DE CIDADÃO DO ESTADO DO RIO DE JANEIRO AO EXCELENTÍSSIMO SENHOR MANOEL CARDOSO LINHARES, ENGENHEIRO E EMPRESÁRIO.








\bigskip

\noindent
A ASSEMBLEIA LEGISLATIVA DO ESTADO DO RIO DE JANEIRO RESOLVE:

\begin{enumerate}[label=Art. \arabic*\textdegree]
\item - Fica concedido o Título de Cidadão do Estado do Rio de Janeiro ao Senhor Manoel Cardoso Linhares, engenheiro e empresário.

\item - Esta Resolução entrará em vigor na data de sua publicação.

\end{enumerate}




\begin{center}
  Plenário Barbosa Lima Sobrinho, 16 de dezembro de 2019.

   \bigskip

  \textbf{ CARLO CAIADO, MÁRCIO PACHECO}

  \bigskip

  \textbf{JUSTIFICATIVA}
  \bigskip

\end{center}

  
	A concessão do Título de Cidadão do Estado do Rio de Janeiro é um ato de reconhecimento a quem fez e faz em prol do bem comum, por seu trabalho e dedicação ao Estado do Rio de Janeiro.

	Manoel Cardoso Linhares é um destes cidadãos. Natural de Crateús-CE, filho de Luís Maria Arruda Linhares e Alda Cardoso Linhares, casado com Morgana Maria Frota Ximenes Cardoso Linhares e pai de três filhos: Manoel Cardoso Linhares Filho, Rodrigo Frota Linhares e Manoella Frota Linhares.

	Engenheiro civil formado pela Universidade de Fortaleza (UNIFOR), é proprietário do Fortaleza Mar Hotel, com mais de duas décadas de atuação no mercado hoteleiro cearense.
Exerceu o cargo de vice-prefeito da Cidade do Eusébio/CE entre os anos de 1993-1996 e 2005-2008.

	Foi Diretor da ABIH/Ceará, da qual já foi presidente e à frente da presidência da Entidade, reivindicou juntos aos candidatos ao Governo do Estado na época, Cid Gomes e Lúcio Alcântara, a construção de um novo Centro de Eventos no Estado. Ambos se comprometeram, e Cid Gomes, eleito, construiu o Centro de Eventos do Ceará, que hoje é um marco para o turismo no Ceará e um dos mais modernos da América Latina.

	É ex-Presidente do Rotary Club de Fortaleza, na gestão 2017-2018.

	Foi Presidente da ABIH Nacional, na gestão 2018/2019 e reeleito para gestão 2020/2021.

	É Presidente do Sindicato Intermunicipal de Hotéis e Meios de Hospedagem - SindihotéisCE (gestão 2014-2018 e reeleito para a gestão 2018-2022), onde conseguiu, após 70 anos, a alteração estatutária de base do Sindicato de Municipal para Intermunicipal.

	Diretor do SKAL Nacional e SKAL Fortaleza, passou por diversos cargos na Diretoria durante os últimos 20 anos.

	Vice-presidente de Hotéis e Meios de Hospedagem da Federação Brasileira de Hospedagem e Alimentação - FBHA, é ainda membro do Conselho Ministerial do Turismo, da Academia Cearense de Turismo- ACTR,  do Conselho Empresarial de Turismo da CNC, do Conselho Empresarial de Turismo da Fecomércio/CE, do Conselho Regional do SESC e SENAC Ceará e Conselheiro do Fortur/Comtur Fortaleza, da Câmara Setorial de Turismo e Eventos no Ceará e do Conselho Municipal do Trabalho de Fortaleza.

	Por seu importante curriculum é que proponho esta homenagem ao Engenheiro e Empresário Manoel Cardoso Linhares, contando com apoio dos nobres colegas para que a proposta logre êxito.



\iffalse
\begin{center}
  \textbf{REFERÊNCIAS}
\end{center}


\fi



\end{document}

