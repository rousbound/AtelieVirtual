\documentclass[10pt]{article}
\usepackage[portuguese]{babel}
\usepackage[utf8]{inputenc}
\usepackage[pdftex]{graphicx}
\usepackage{venndiagram}
\usepackage{subcaption}
\usepackage{caption}
\usepackage[backend=biber,style=authoryear-ibid]{biblatex}
\usepackage[normalem]{ulem}
\usepackage[margin=0.8in]{geometry}
\addbibresource{psychoanalysis.bib}
\graphicspath{{Pictures/}}
\usepackage{tikz}
\usepackage{setspace}
\usepackage{enumitem}
\usepackage{textcomp}
\usepackage{hyperref}


\date{}

\newcommand{\quotebox}[3]{
  \begin{center}
\noindent\fbox{ 
  \parbox{#3\textwidth}{%
  {\itshape#1\itshape}

  \raggedleft {\textbf{#2}} 
    }%  
}
\end{center}
}

\newcommand{\spawnfig}[3]
{
  \begin{figure}[h]
  \centering
  \includegraphics[scale={#3}]{#1}
  \caption{#2}
  \end{figure}
}


\begin{document}
\maketitle
\begin{center}
  % Deputado DR DEODALTO
  \huge
  \vspace{-3cm}\href{http://alerjln1.alerj.rj.gov.br/scpro1923.nsf/f4b46b3cdbba990083256cc900746cf6/514543173675750c8325841000777017?OpenDocument}{PROJETO DE LEI Nº 712/2019}
\bigskip
\bigskip
\bigskip
  
\end{center}

\textbf{EMENTA:} 
ALTERA O ANEXO DA LEI Nº 5.645, DE 06 DE JANEIRO DE 2010, INCLUINDO NO CALENDÁRIO OFICIAL DE EVENTOS DO ESTADO A SEMANA DE CONSCIENTIZAÇÃO ACERCA DO CONSUMO EXCESSIVO DE AÇUCAR.








\bigskip

\noindent
A ASSEMBLEIA LEGISLATIVA DO ESTADO DO RIO DE JANEIRO RESOLVE:

\begin{enumerate}[label=Art. \arabic*\textdegree]

\item - Fica instituída a &#``SEMANA DE CONSCIENTIZAÇÃO ACERCA DO CONSUMO EXCESSIVO DE AÇUCAR&#" no âmbito do Estado do Rio de Janeiro.
Parágrafo único - A semana aqui referida ocorrerá, anualmente, na segunda semana do mês de novembro.
\item - Ficará a cargo do Poder Executivo implementar políticas públicas com o objetivo de:
| - Levar informações pertinentes por meio de campanhas publicitárias e educativas;
|| - Veicular campanhas nas unidades de saúde, rede pública de ensino e demais órgãos;
||| - Efetuar parcerias com municípios e entidades privadas interessados em aderir à campanha de conscientização.
\item - Esta Lei entra em vigor na data de sua publicação.





\end{enumerate}




\begin{center}
  Plenário Barbosa Lima Sobrinho, 06 de junho de 2019.

   \bigskip

  \textbf{ DR DEODALTO}

  \bigskip

  \textbf{JUSTIFICATIVA}
  \bigskip

\end{center}

  O açúcar está relacionado com mais de 70 doenças que dificultam e oneram o dia a dia da população. Também está diretamente ligado à doenças mais graves como a diabetes, que é a terceira doença que mais causa mortes no mundo. 
A maior parte do açúcar ingerido por nós consumidores está presente na composição de diferentes alimentos industrializados não só pela capacidade de melhorar o sabor como também por ser um conservante que aumenta o tempo de prateleira. 
Observemos que nem sempre a presença do açúcar fica claro nas embalagens. Muitas vezes, sua presença, na composição de alimentos, vem disfarçada como xarope de milho, dextrose, glucose, sacarose, maltodextrina e diversos outros nomes, os quais nos leva a ingerir este tipo de carboidrato sem a consciência que ele se encontra no alimento, dificultando o controle de seu consumo.
Esta proposição tem por ideal esclarecer os malefícios do consumo inconsciente do açúcar e doenças relacionadas; incentivar a leitura de rótulos e embalagens; incentivar o preparo de alimentos com menos açúcar; assim como apresentar os benefícios de uma dieta saudável com menos açúcar.
A ideia para elaboração desta proposição surgiu com base em orientações do &#"Guia com recomendações de consumo de açúcar para adultos e crianças&#" (https://www.who.int/nmh/publications/ncd-action-plan/en/) da Organização Mundial da Saúde (OMS), lançado em 4 de março de 2015, bem como da diretriz &#``Ingestão de açúcares por adultos e crianças&#" extraído de https://www.paho.org/bra/index.php?option=com_content&amp;view=article&amp;id=4783:oms-recomenda-que-os-paises-reduzam-o-consumo-de-acucar-entre-adultos-e-criancas&amp;Itemid=820 da Organização Pan-Americana de Saúde (OPAS).
A data escolhida é por ocasião do dia 14 de novembro, Dia Mundial do Combate ao Diabetes. Sendo assim, objetivamos com a proposta não só conscientizar como também  buscar melhorar a saúde e a qualidade de vida dos cidadãos fluminenses. Diante do exposto, conto com o apoio de meus pares na aprovação deste projeto de lei.



\iffalse
\begin{center}
  \textbf{REFERÊNCIAS}
\end{center}


\fi



\end{document}

