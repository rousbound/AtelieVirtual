\documentclass[10pt]{article}
\usepackage[portuguese]{babel}
\usepackage[utf8]{inputenc}
\usepackage[pdftex]{graphicx}
\usepackage{venndiagram}
\usepackage{subcaption}
\usepackage{caption}
\usepackage[backend=biber,style=authoryear-ibid]{biblatex}
\usepackage[normalem]{ulem}
\usepackage[margin=0.8in]{geometry}
\addbibresource{psychoanalysis.bib}
\graphicspath{{Pictures/}}
\usepackage{tikz}
\usepackage{setspace}
\usepackage{enumitem}
\usepackage{textcomp}
\usepackage{hyperref}


\date{}

\newcommand{\quotebox}[3]{
  \begin{center}
\noindent\fbox{ 
  \parbox{#3\textwidth}{%
  {\itshape#1\itshape}

  \raggedleft {\textbf{#2}} 
    }%  
}
\end{center}
}

\newcommand{\spawnfig}[3]
{
  \begin{figure}[h]
  \centering
  \includegraphics[scale={#3}]{#1}
  \caption{#2}
  \end{figure}
}


\begin{document}
\maketitle
\begin{center}
  % Deputados LUIZ PAULO, LUCINHA
  \huge
  \vspace{-3cm}\href{http://alerjln1.alerj.rj.gov.br/scpro1923.nsf/f4b46b3cdbba990083256cc900746cf6/e0d68ac5331848790325859000612065?OpenDocument}{PROJETO DE LEI Nº 2779/2020}
\bigskip
\bigskip
\bigskip
  
\end{center}

\textbf{EMENTA:} 
DISPÕE SOBRE A REALIZAÇÃO DE AUTOVISTORIAS ENQUANTO PERDURAR O ESTADO DE CALAMIDADE PÚBLICA, ESTABELECIDO PELO DECRETO Nº 46.973, DE 16 DE MARÇO DE 2020 E RECONHECIDO PELA LEI Nº 8.794 DE 17 DE ABRIL DE 2020








\bigskip

\noindent
A ASSEMBLEIA LEGISLATIVA DO ESTADO DO RIO DE JANEIRO RESOLVE:

\begin{enumerate}[label=Art. \arabic*\textdegree]
\item - Ficam os condomínios residenciais e comerciais dispensados da obrigatoriedade da realização de autovistoria enquanto perdurar o estado de calamidade pública, estabelecido pelo Decreto nº 46.973, de 16 de março de 2020 e reconhecido pela Lei nº 8.794 de 17 de abril de 2020. 

\item - Os efeitos da suspensão a que se refere o artigo anterior não são aplicados às obras de natureza emergenciais. 

\item - Esta Lei entrará em vigor na data de sua publicação. 

\end{enumerate}




\begin{center}
  Plenário Barbosa Lima Sobrinho, 23 de junho de 2020.

   \bigskip

  \textbf{ LUIZ PAULO, LUCINHA}

  \bigskip

  \textbf{JUSTIFICATIVA}
  \bigskip

\end{center}

  Considerando a publicação do Decreto nº 46.973 de 16 de março de 2020 que &#``Reconhece a situação de emergência na saúde pública do Estado do Rio de Janeiro em razão do contágio e adota medidas de enfrentamento da propagação decorrente do Novo Coronavírus (COVID-19) e dá outras providências&#", reconhecido pela Lei nº 8.794 de 17 de abril de 2020, bem como os graves impactos econômicos gerados pela pandemia no Novo Coronavírus (COVID-19) aos condomínios residenciais e comerciais que vêm sofrendo com o aumento da inadimplência, é que submetemos, para a apreciação desta Casa de Leis, esse relevante projeto de lei. 



\iffalse
\begin{center}
  \textbf{REFERÊNCIAS}
\end{center}


\fi



\end{document}

