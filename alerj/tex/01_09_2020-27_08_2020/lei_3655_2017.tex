\documentclass[10pt]{article}
\usepackage[portuguese]{babel}
\usepackage[utf8]{inputenc}
\usepackage[pdftex]{graphicx}
\usepackage{venndiagram}
\usepackage{subcaption}
\usepackage{caption}
\usepackage[backend=biber,style=authoryear-ibid]{biblatex}
\usepackage[normalem]{ulem}
\usepackage[margin=0.8in]{geometry}
\addbibresource{psychoanalysis.bib}
\graphicspath{{Pictures/}}
\usepackage{tikz}
\usepackage{setspace}
\usepackage{enumitem}
\usepackage{textcomp}
\usepackage{hyperref}


\date{}

\newcommand{\quotebox}[3]{
  \begin{center}
\noindent\fbox{ 
  \parbox{#3\textwidth}{%
  {\itshape#1\itshape}

  \raggedleft {\textbf{#2}} 
    }%  
}
\end{center}
}

\newcommand{\spawnfig}[3]
{
  \begin{figure}[h]
  \centering
  \includegraphics[scale={#3}]{#1}
  \caption{#2}
  \end{figure}
}


\begin{document}
\maketitle
\begin{center}
  % Deputado ÁTILA NUNES
  \huge
  \vspace{-3cm}\href{http://alerjln1.alerj.rj.gov.br/scpro1519.nsf/f4b46b3cdbba990083256cc900746cf6/04bc8bb81e187805832581d8005de236?OpenDocument}{PROJETO DE LEI Nº 3655/2017}
\bigskip
\bigskip
\bigskip
  
\end{center}

\textbf{EMENTA:} 
ALTERA A LEI Nº 7.108, DE 19 DE NOVEMBRO DE 2015, PARA INSTITUIR NO CALENDÁRIO OFICIAL DO RIO DE JANEIRO O DIA  DE LUTA CONTRA O HIV/AIDS, BEM COMO IMPLEMENTAR AÇÕES DE PREVENÇÃO ÀS DOENÇAS SEXUALMENTE TRANSMISSÍVEIS  - DST E AO HIV/AIDS NA CAMPANHA DENOMINADA &#``DEZEMBRO VERMELHO&#".








\bigskip

\noindent
A ASSEMBLEIA LEGISLATIVA DO ESTADO DO RIO DE JANEIRO RESOLVE:

\begin{enumerate}[label=Art. \arabic*\textdegree]
\item - Modifique-se a Ementa da Lei nº 7.108, de 19 de novembro de 2015, que passa a vigorar com a seguinte redação:



"INSTITUI NO CALENDÁRIO OFICIAL DO ESTADO DO RIO DE JANEIRO O DIA DE LUTA CONTRA O HIV/AIDS E A CAMPANHA DE PREVENÇÃO ÀS DOENÇAS SEXUALMENTE TRANSMISSÍVEIS  - DST E AO HIV/AIDS DENOMINADA &#``DEZEMBRO VERMELHO&#".



\item - Modifique-se o artigo 1º da Lei nº 7.108, de 19 de novembro de 2015, que passa a vigorar com a seguinte redação:



\item - Fica instituído no Estado do Rio de Janeiro o dia 01 (um) de dezembro como o Dia Estadual de Luta contra o HIV/AIDS, bem como a Campanha de Prevenção às Doenças Sexualmente Transmissíveis - DST e ao HIV/AIDS denominada de &#``Dezembro Vermelho&#", a ser comemorada anualmente durante todo o mês de dezembro, com o objetivo de sensibilizar a população quanto à importância da prevenção e enfrentamento do HIV/AIDS e outras DST´s, com foco na conscientização, prevenção, assistência e combate destas doenças, bem como proteção, tratamento  e promoção dos direitos de seus portadores.



\item - Acrescente-se o § 1º ao artigo 1º da Lei nº 7.108, de 19 de novembro de 2015, com a seguinte redação:



§ 1º - O símbolo da Campanha prevista no caput deste artigo será &#``um laço&#"  na cor vermelha, podendo as Instituições Públicas Estaduais participarem da divulgação da Campanha mediante a utilização de iluminação e decorações em suas sedes, monumentos e logradouros públicos na mesma cor vermelha durante a realização da Campanha, em especial os de relevante importância e grande fluxo de pessoas.







\item - Acrescente-se o § 2º ao artigo 1º da Lei nº 7.108, de 19 de novembro de 2015, com a seguinte redação:



§ 2º - No decorrer do mês serão desenvolvidas ações educativas tais como palestras e seminários nos diversos segmentos da sociedade, principalmente em estabelecimentos do ensino médio e fundamental, podendo o Poder Público firmar convênios com os municípios e associações sem fins lucrativos para realização destes atos.



\item - Acrescente-se o § 3º ao artigo 1º da Lei nº 7.108, de 19 de novembro de 2015, com a seguinte redação:



§ 3º -  O Poder Executivo Estadual deverá ampliar e facilitar o acesso à realização do exame preventivo, inclusive com disponibilização de laboratórios móveis com os equipamentos e pessoal necessários para a realização de exames junto às  comunidades em datas pré-determinadas e amplamente divulgadas durante todo o decorrer da campanha.




\item - Acrescente-se o § 4º ao artigo 1º da Lei nº 7.108, de 19 de novembro de 2015, com a seguinte redação:



§ 4º -  A campanha deverá ser desenvolvida em todas as esferas do poder, em ações unificadas do Poder Executivo Estadual e respectivos municípios, com participação dos profissionais da saúde e enfermagem necessários para a intensificação das ações preventivas e realização dos correspondentes exames.




\item - Modifique-se o artigo 2º da Lei nº 7.108, de 19 de novembro de 2015, que passa a vigorar com a seguinte redação:



\item - A campanha ora instituída passa a integrar o Calendário Oficial de Datas e Eventos do Estado do Rio de Janeiro, passando o Anexo da Lei nº 5.645, de 06 de janeiro de 2010, a vigorar com a seguinte redação:




CALENDÁRIO DATAS COMEMORATIVAS DO ESTADO DO RIO DE JANEIRO:

(&#8230;)

DEZEMBRO

(&#8230;)

MÊS DE DEZEMBRO - Mês da Campanha de Prevenção às Doenças Sexualmente Transmissíveis - DST e ao HIV/AIDS denominada &#``DEZEMBRO VERMELHO&#".

DIA 01 - Dia  Estadual de Luta contra o HIV/AIDS.

(...)


\item - Esta Lei entra em vigor na data de sua publicação.




\end{enumerate}




\begin{center}
  Plenário Barbosa Lima Sobrinho, 27 de novembro de 2017.

   \bigskip

  \textbf{ ÁTILA NUNES}

  \bigskip

  \textbf{JUSTIFICATIVA}
  \bigskip

\end{center}

  A presente proposição tem por finalidade aperfeiçoar a Lei Estadual nº 7.108/15, que instituiu a Campanha denominada "Dezembro Vermelho", corrigindo o seu texto para determinar o seu real objetivo de buscar uma maior conscientização quanto às doenças sexualmente transmissíveis, em especial em relação ao HIV/AIDS, bem como incluir a respectiva Campanha no Calendário Oficial do Estado do Rio de Janeiro.   O fato é que, não obstante a existência da Lei ora alterada, o aumento dos casos de tais doenças, principalmente da AIDS vem sendo observado principalmente entre os jovens com faixa etária entre 16 e 24 anos.  A melhor forma de combater este doença é fortalecer as estratégias de campanhas preventivas, especialmente nessa faixa etária em que muitos jovens desconhecem os riscos da doença, pleo que o objetivo principal da presente proposição é viabilizar a ampla divulgação da temática, tornando-a um sucesso no combate a tais doenças, à exemplo da Campanha Outubro Rosa. 
Em razão do exposto, conto com o apoio dos meus nobres pares para a aprovação da presente proposição.



\iffalse
\begin{center}
  \textbf{REFERÊNCIAS}
\end{center}


\fi



\end{document}

