\documentclass[10pt]{article}
\usepackage[portuguese]{babel}
\usepackage[utf8]{inputenc}
\usepackage[pdftex]{graphicx}
\usepackage{venndiagram}
\usepackage{subcaption}
\usepackage{caption}
\usepackage[backend=biber,style=authoryear-ibid]{biblatex}
\usepackage[normalem]{ulem}
\usepackage[margin=0.8in]{geometry}
\addbibresource{psychoanalysis.bib}
\graphicspath{{Pictures/}}
\usepackage{tikz}
\usepackage{setspace}
\usepackage{enumitem}
\usepackage{textcomp}
\usepackage{hyperref}


\date{}

\newcommand{\quotebox}[3]{
  \begin{center}
\noindent\fbox{ 
  \parbox{#3\textwidth}{%
  {\itshape#1\itshape}

  \raggedleft {\textbf{#2}} 
    }%  
}
\end{center}
}

\newcommand{\spawnfig}[3]
{
  \begin{figure}[h]
  \centering
  \includegraphics[scale={#3}]{#1}
  \caption{#2}
  \end{figure}
}


\begin{document}
\maketitle
\begin{center}
  % Deputada ENFERMEIRA REJANE
  \huge
  \vspace{-3cm}\href{http://alerjln1.alerj.rj.gov.br/scpro1923.nsf/f4b46b3cdbba990083256cc900746cf6/e468d6c07dc6dbc603258554005eb245?OpenDocument}{PROJETO DE LEI Nº 2425/2020}
\bigskip
\bigskip
\bigskip
  
\end{center}

\textbf{EMENTA:} 
CRIA PROGRAMA ESPECIAL DE CONTRATAÇÃO DE ESTAGIÁRIOS PELO PERÍODO EM QUE DURAR O ESTADO DE CALAMIDADE EM DECORRÊNCIA DA CODIV-19 E DÁ OUTRAS PROVIDENCIAS.    








\bigskip

\noindent
A ASSEMBLEIA LEGISLATIVA DO ESTADO DO RIO DE JANEIRO RESOLVE:

\begin{enumerate}[label=Art. \arabic*\textdegree]
\item - O poder executivo deverá estabelecer programa especial de contratação de estagiários pelo período em que durar o estado de calamidade em decorrência da codiv-19.

2º A administração deverá contratar estudantes do último ano de enfermagem e medicina para o programa especial descrito no artigo primeiro, que atuarão em atividades complementares no sistema público de saúde estadual, devendo ser observados os seguintes critérios:

\begin{enumerate}[label=\Roman*]
\item - Pagamento de bolsa em valor não inferior ao salário mínimo federal. 
\end{enumerate}
 
\item - A realização de atividades não relacionadas com o enfrentamento direto ou indireto com a codiv-19, podendo atuar:

a)	Nas campanhas de vacinação;
b)	No programa de saúde da família;
c)	Nos hospitais não referência para covid-19
d)	Serviços remotos de orientação ao público sobre a covid-19
e)	Outras atividades em sua área de formação, sem contato com pacientes com suspeita ou confirmação de Covid-19.

\item - Em todas as atividades realizadas pelos estagiários, exceto a alínea d do inciso II, deverá ser fornecido equipamento de proteção individual adequado, em especial, luvas de látex, marcaras cirúrgicas e capote descartável.

\item - A contratação, por parte da administração, em favor do estagiário, de seguro contra acidentes pessoais, nos termos do inciso IV do art. 9º da Lei nº 11.788, de  25 de setembro de 2008 V - Indicação de um profissional, com formação ou experiência profissional na área de conhecimento desenvolvida no curso do estagiário, para orientar e supervisionar por grupo de 10 (dez) estagiários simultaneamente;

\item - Jornada de atividade em estágio não superior a 6 (seis) horas diárias e 30 (trinta) horas semanais.

\item - A administração deverá ofertar o número mínimo de vagas para o programa criado pela presente lei na proporção de uma vaga para cada 10 (dez) profissionais, servidores ou contratados, atuando na rede de saúde Estadual.

\item - A previsão contida no inciso II do art. 2° deverá ser aplicada inclusive em relação aos estagiários já contratados pela administração.

\item - Ficam suspensas as contratações de estagiários da área da saúda distintas da presente lei.

\item - As despesas resultantes da aplicação da presente Lei correrão à conta dos recursos próprios da secretária de saúde,  ficando o Poder Executivo autorizado a abrir créditos suplementares, se necessário.

\item - Esta lei entra em vigor na data de sua publicação.


\end{enumerate}




\begin{center}
  Plenário Barbosa Lima Sobrinho, 24 de abril de 2020

   \bigskip

  \textbf{ ENFERMEIRA REJANE}

  \bigskip

  \textbf{JUSTIFICATIVA}
  \bigskip

\end{center}

  A declaração de pandemia pela Organização Mundial da Saúde  (OMS)  provocada  pelo novo   coronavírus, com gravíssimas implicações  principalmente em relação aos  profissionais de saúde que atuam diretamente com a população brasileira nas unidades de saúde de todo o país.
No Estado do Rio de Janeiro, com  a aparição no Brasil do COVID-19, popularmente chamado de Coronavírus, foi reconhecido o estado de   calamidade pública,  por meio do   Decreto   N° 46.984  de  20  de  março  de  2020. A partir deste momento, foram identificados os  efeitos  práticos  desta  decisão,  ao   mesmo   tempo   em  que é  detectada  a  necessidade de implantar, imprescindivelmente, outras soluções  para serem aplicadas à nova situação. 

O presente projeto visa solucionar duas questões que surgiram na atual crise. A primeira é a falta de profissionais para atuar diretamente no combate a pandemia, a segunda questão é a temeridade de se alocar estudantes ainda em formação no combate direto ao vírus.

O programa estabelecido visa a manutenção dos serviços de saúde essenciais que não possui ligação direta com a pandemia e ainda possibilita o remanejamento de profissionais experientes para os hospitais de referência.

Assim, conto com a colaboração dos meus pares para a aprovação da presente proposta.



\iffalse
\begin{center}
  \textbf{REFERÊNCIAS}
\end{center}


\fi



\end{document}

