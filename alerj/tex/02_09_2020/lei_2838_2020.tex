\documentclass[10pt]{article}
\usepackage[portuguese]{babel}
\usepackage[utf8]{inputenc}
\usepackage[pdftex]{graphicx}
\usepackage{venndiagram}
\usepackage{subcaption}
\usepackage{caption}
\usepackage[backend=biber,style=authoryear-ibid]{biblatex}
\usepackage[normalem]{ulem}
\usepackage[margin=0.8in]{geometry}
\addbibresource{psychoanalysis.bib}
\graphicspath{{Pictures/}}
\usepackage{tikz}
\usepackage{setspace}
\usepackage{enumitem}
\usepackage{textcomp}
\usepackage{hyperref}


\date{}

\newcommand{\quotebox}[3]{
  \begin{center}
\noindent\fbox{ 
  \parbox{#3\textwidth}{%
  {\itshape#1\itshape}

  \raggedleft {\textbf{#2}} 
    }%  
}
\end{center}
}

\newcommand{\spawnfig}[3]
{
  \begin{figure}[h]
  \centering
  \includegraphics[scale={#3}]{#1}
  \caption{#2}
  \end{figure}
}


\begin{document}
\maketitle
\begin{center}
  % Deputado DANNIEL LIBRELON
  \huge
  \vspace{-3cm}\href{http://alerjln1.alerj.rj.gov.br/scpro1923.nsf/f4b46b3cdbba990083256cc900746cf6/22a9b2ea9f05d3d40325859e00557696?OpenDocument}{PROJETO DE LEI Nº 2838/2020}
\bigskip
\bigskip
\bigskip
  
\end{center}

\textbf{EMENTA:} 
DISPÕE SOBRE A INFORMAÇÃO, APOIO E ACOLHIMENTO QUALIFICADO ÀS  GESTANTES E PARTURIENTES DURANTE ENDEMIAS, EPIDEMIAS OU PANDEMIAS.








\bigskip

\noindent
A ASSEMBLEIA LEGISLATIVA DO ESTADO DO RIO DE JANEIRO RESOLVE:

\begin{enumerate}[label=Art. \arabic*\textdegree]
\item - Autoriza o poder executivo a prestar serviço virtual de informação, apoio e acolhimento qualificado às gestantes e parturientes, durante endemias, epidemias ou pandemias, com informações referentes ao pré-natal, puerpério e pós-parto.

\item - Os procedimentos para o atendimento ao serviços indicado no artigo 1º poderão ser coordenados pela Secretaria de Estado de Saúde.

\item - Esta lei entra em vigor na data de sua publicação.

\end{enumerate}




\begin{center}
  Plenário Barbosa Lima Sobrinho, 06 de julho de 2020.DANNIEL LIBRELON

   \bigskip

  \textbf{ DANNIEL LIBRELON}

  \bigskip

  \textbf{JUSTIFICATIVA}
  \bigskip

\end{center}

  A gravidez é um momento especial, cheio de emoção e antecipação, mas, para as gestantes que enfrentam o surto da doença do novo coronavírus (Covid-19), o medo, a ansiedade e a incerteza podem afetar esse momento feliz.
Diante desse contexto, é de fundamental importância a criação de atendimentos especializados para proteger as mulheres gestantes e puérperas em todos os setores e unidades de saúde do Estado do Rio de Janeiro, durante todo o período de atenção pré-natal, parto e pós-parto.
O serviço virtual de informação, apoio e acolhimento qualificado às gestantes e parturientes seria mais uma alternativa para as gestantes e puérperas neste momento tão delicado, e uma forma de amenizar a sobrecarga das unidades e dos profissionais de saúde nesse tempo de pandemia.
Assim, o projeto aborda matéria de contemporânea preocupação social, sendo sua aprovação um avanço significativo para a proteção da saúde das gestantes e puérperas.
Diante do exposto, após apreciação, conto com a aprovação desta propositura pelos nobres Pares.



\iffalse
\begin{center}
  \textbf{REFERÊNCIAS}
\end{center}


\fi



\end{document}

