\documentclass[10pt]{article}
\usepackage[portuguese]{babel}
\usepackage[utf8]{inputenc}
\usepackage[pdftex]{graphicx}
\usepackage{venndiagram}
\usepackage{subcaption}
\usepackage{caption}
\usepackage[backend=biber,style=authoryear-ibid]{biblatex}
\usepackage[normalem]{ulem}
\usepackage[margin=0.8in]{geometry}
\addbibresource{psychoanalysis.bib}
\graphicspath{{Pictures/}}
\usepackage{tikz}
\usepackage{setspace}
\usepackage{enumitem}
\usepackage{textcomp}
\usepackage{hyperref}


\date{}

\newcommand{\quotebox}[3]{
  \begin{center}
\noindent\fbox{ 
  \parbox{#3\textwidth}{%
  {\itshape#1\itshape}

  \raggedleft {\textbf{#2}} 
    }%  
}
\end{center}
}

\newcommand{\spawnfig}[3]
{
  \begin{figure}[h]
  \centering
  \includegraphics[scale={#3}]{#1}
  \caption{#2}
  \end{figure}
}


\begin{document}
\maketitle
\begin{center}
  % Deputada MARTHA ROCHA
  \huge
  \vspace{-3cm}\href{http://alerjln1.alerj.rj.gov.br/scpro1519.nsf/f4b46b3cdbba990083256cc900746cf6/d1cb1b5e1deb66f583258145005c710d?OpenDocument}{PROJETO DE LEI Nº 2984/2017}
\bigskip
\bigskip
\bigskip
  
\end{center}

\textbf{EMENTA:} 
DISPÕE SOBRE A OBRIGATORIEDADE DAS EMPRESAS DE CARTÕES DE CRÉDITO OU DÉBITO AVISAREM AOS CONSUMIDORES/CLIENTES SOBRE A OCORRÊNCIA DE BLOQUEIO DO CARTÃO DE CRÉDITO OU DÉBITO.








\bigskip

\noindent
A ASSEMBLEIA LEGISLATIVA DO ESTADO DO RIO DE JANEIRO RESOLVE:

\begin{enumerate}[label=Art. \arabic*\textdegree]

\item - Ficam obrigadas as empresas de cartões de crédito ou débito a informar acerca do bloqueio do cartão de crédito dos clientes do Estado Rio de Janeiro. 
§ 1º. Considera-se obrigatório o serviço sempre que aquele bloqueio não tiver sido solicitado pelo próprio cliente. 
§ 2º. As empresas terão o prazo de 24 (vinte e quatro) horas para comunicar ao cliente o bloqueio. 
§ 3º. A forma sob a qual será realizado o aviso deverá ser escolhida dentre as opções elencadas pela operadora do cartão de crédito ou débito e oferecidas ao cliente.

\item - As empresas de cartões de crédito ou débito deverão informar o motivo do bloqueio.
 
\item - O descumprimento das disposições desta lei sujeitará o infrator às penalidades previstas no Código de Defesa do Consumidor, devendo a multa ser revertida para o Fundo Especial de Apoio a Programas de Proteção e Defesa do Consumidor - FEPROCON.
  
\item - Caberá ao Poder Executivo regulamentar e fiscalizar a presente Lei em todos os aspectos necessários para a sua efetiva aplicação.
 
\item - Esta Lei entra em vigor após 90 dias da sua publicação.

\end{enumerate}




\begin{center}
  Plenário Barbosa Lima Sobrinho, 20 de junho de 2017.

   \bigskip

  \textbf{ MARTHA ROCHA}

  \bigskip

  \textbf{JUSTIFICATIVA}
  \bigskip

\end{center}

  	Trata-se de Projeto de Lei que &#``DISPÕE SOBRE A OBRIGATORIEDADE DE AS EMPRESAS DE CARTÕES DE CRÉDITO OU DÉBITO AVISAREM AOS CONSUMIDORES/CLIENTES SOBRE A OCORRÊNCIA DE BLOQUEIO DO CARTÃO DE CRÉDITO OU DÉBITO&#".
	O presente projeto de Lei tem como objetivo informar o consumidor acerca do bloqueio dos serviços de cartão de débito e crédito pelo contratado, assim como notificar o contratante sobre o motivo desse bloqueio.
	De acordo com dados de pesquisa do DATAFOLHA, elaborada em 2013, 76\% da população brasileira possui algum meio eletrônico de pagamento, sendo o Brasil o terceiro maior pais emissor de cartões do mundo. Só em 2015 foram emitidos no Brasil cerca de 909 milhões de cartões de crédito ou débito.
	Atualmente, as operadoras podem cancelar ou bloquear cartões sem avisar previamente o consumidor, indo contra a Lei 8.078 de 1990, o Código de Defesa do Consumidor, ação que este projeto pretende corrigir.
	Sendo assim, visando assegurar ao consumidor o seu direito à informação, entendo ser de grande importância e pertinência a presente propositura, razão pela qual conto com a aprovação de meus nobres pares.



\iffalse
\begin{center}
  \textbf{REFERÊNCIAS}
\end{center}


\fi



\end{document}

