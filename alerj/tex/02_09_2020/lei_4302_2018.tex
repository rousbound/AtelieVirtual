\documentclass[10pt]{article}
\usepackage[portuguese]{babel}
\usepackage[utf8]{inputenc}
\usepackage[pdftex]{graphicx}
\usepackage{venndiagram}
\usepackage{subcaption}
\usepackage{caption}
\usepackage[backend=biber,style=authoryear-ibid]{biblatex}
\usepackage[normalem]{ulem}
\usepackage[margin=0.8in]{geometry}
\addbibresource{psychoanalysis.bib}
\graphicspath{{Pictures/}}
\usepackage{tikz}
\usepackage{setspace}
\usepackage{enumitem}
\usepackage{textcomp}
\usepackage{hyperref}


\date{}

\newcommand{\quotebox}[3]{
  \begin{center}
\noindent\fbox{ 
  \parbox{#3\textwidth}{%
  {\itshape#1\itshape}

  \raggedleft {\textbf{#2}} 
    }%  
}
\end{center}
}

\newcommand{\spawnfig}[3]
{
  \begin{figure}[h]
  \centering
  \includegraphics[scale={#3}]{#1}
  \caption{#2}
  \end{figure}
}


\begin{document}
\maketitle
\begin{center}
  % Deputada ZEIDAN LULA
  \huge
  \vspace{-3cm}\href{http://alerjln1.alerj.rj.gov.br/scpro1519.nsf/f4b46b3cdbba990083256cc900746cf6/0b3199ecaeb73e69832582e4006c71c6?OpenDocument}{PROJETO DE LEI Nº 4302/2018}
\bigskip
\bigskip
\bigskip
  
\end{center}

\textbf{EMENTA:} 
AUTORIZA O PODER EXECUTIVO A INSTALAR UMA SUBSEDE DA UNIVERSIDADE DO ESTADO DO RIO DE JANEIRO - UERJ NO MUNICÍPIO DE MARICÁ, BEM COMO POLOS OU CAMPI AVANÇADOS.








\bigskip

\noindent
A ASSEMBLEIA LEGISLATIVA DO ESTADO DO RIO DE JANEIRO RESOLVE:

\begin{enumerate}[label=Art. \arabic*\textdegree]

\item - Fica o Poder Executivo autorizado a instalar uma subsede ou filial da Universidade do Estado do Rio de Janeiro - UERJ no município de Maricá, onde deverá ser instalado um campus universitário completo com cursos de graduação e pós-graduação, de pesquisa e de extensão, em todos os níveis, voltado para o atendimento de toda a região Metropolitana e da Baixada Litorânea.

\item - A partir deste campus universitário, poderá o Poder Executivo implantar polos ou campi avançados da Universidade do Estado do Rio de Janeiro - UERJ nos municípios circunvizinhos, de forma a estender a oferta de cursos superiores em todos os municípios citados, com amplo alcance da referida Região.

Parágrafo Único - O Ato que instituir cada Polo ou campus deverá determinar os cursos superiores que inicialmente serão ministrados no mesmo, tanto em grau de bacharelado, quanto em licenciatura e tecnologia, dando ênfase, preferencialmente, ao desenvolvimento cultural da população local e à vocação sócio econômico da região, podendo os cursos disponibilizados serem ampliados a qualquer tempo, conforme a necessidade e conveniência do Executivo.

\item - As despesas decorrentes da aplicação desta Lei correrão a conta de Programa de Trabalho próprio do orçamento da UERJ, ficando o Poder Executivo Estadual, bem como Municipal autorizados a suplementar as dotações orçamentárias que se mostrarem insuficientes para o alcance dos objetivos da presente Lei.

\item - O Poder Executivo baixará os Atos que se fizerem necessários à regulamentação da presente Lei visando à sua fiel execução.

\item - Esta lei entrará em vigor na data de sua publicação.


\end{enumerate}




\begin{center}
  Plenário Barbosa Lima Sobrinho, 09 de Agosto de 2018

   \bigskip

  \textbf{ ZEIDAN LULA}

  \bigskip

  \textbf{JUSTIFICATIVA}
  \bigskip

\end{center}

  A educação é uma das áreas que mais pode afetar o futuro de uma nação, preparando-a e capacitando-a para viver o progresso que tanto almeja.   Neste contexto, o Poder Executivo deve sempre buscar novos mecanismos de aperfeiçoamento dos ensino fundamental e médio, bem como ampliar a oferta de vagas nas universidades públicas nas áreas menos assistidas pelo Governo.   Por certo que muitos estudantes se vêm impedidos de ingressar em uma universidade pública devido à distância de sua residência ao campus universitário, gerando uma despesa incompatível com o seu apertado orçamento doméstico.   A abertura de uma subsede da UERJ no município de Maricá visa encurtar essa distância que tem retirado tantos estudantes da sala de aula de uma universidade pública, cabendo ressaltar que o Município de Maricá conta hoje com mais de 40.000 habitantes, tendo uma posição estratégica para o alcance da região metropolitana, bem como da Baixada Litorânea.   A partir desta subsede, a instalação de  polos ou campi avançados facilitará ainda mais o ingresso de alunos carentes no ensino superior, reduzindo a distância física e social da universidade pública, além de proporcionar a esta região do Estado uma real melhoria do seu desenvolvimento econômico e social.     O Poder Público tem interesse em formar mão de obra qualificada para o mercado de trabalho, além de levar o desenvolvimento a esta tão importante região de nosso Estado, o que mostra a viabilidade do presente Projeto de Lei, levando à Faculdade a quem mais precisa.




\iffalse
\begin{center}
  \textbf{REFERÊNCIAS}
\end{center}


\fi



\end{document}

