\documentclass[10pt]{article}
\usepackage[portuguese]{babel}
\usepackage[utf8]{inputenc}
\usepackage[pdftex]{graphicx}
\usepackage{venndiagram}
\usepackage{subcaption}
\usepackage{caption}
\usepackage[backend=biber,style=authoryear-ibid]{biblatex}
\usepackage[normalem]{ulem}
\usepackage[margin=0.8in]{geometry}
\addbibresource{psychoanalysis.bib}
\graphicspath{{Pictures/}}
\usepackage{tikz}
\usepackage{setspace}
\usepackage{enumitem}
\usepackage{textcomp}
\usepackage{hyperref}


\date{}

\newcommand{\quotebox}[3]{
  \begin{center}
\noindent\fbox{ 
  \parbox{#3\textwidth}{%
  {\itshape#1\itshape}

  \raggedleft {\textbf{#2}} 
    }%  
}
\end{center}
}

\newcommand{\spawnfig}[3]
{
  \begin{figure}[h]
  \centering
  \includegraphics[scale={#3}]{#1}
  \caption{#2}
  \end{figure}
}


\begin{document}
\maketitle
\begin{center}
  % Deputado MÁRCIO CANELLA
  \huge
  \vspace{-3cm}\href{http://alerjln1.alerj.rj.gov.br/scpro1923.nsf/f4b46b3cdbba990083256cc900746cf6/1c0416d676b0b0ee032584c60050a7e2?OpenDocument}{PROJETO DE LEI Nº 1733/2019}
\bigskip
\bigskip
\bigskip
  
\end{center}

\textbf{EMENTA:} 

ALTERA A LEI Nº 5.645, DE 06 DE JANEIRO DE 2010, PARA INSTITUIR NO CALENDÁRIO OFICIAL DO ESTADO DO RIO DE JANEIRO A SEMANA ESTADUAL DA CONSCIENTIZAÇÃO SOBRE O HERPES ZOSTER, NO ÂMBITO DO ESTADO DO RIO DE JANEIRO.








\bigskip

\noindent
A ASSEMBLEIA LEGISLATIVA DO ESTADO DO RIO DE JANEIRO RESOLVE:

\begin{enumerate}[label=Art. \arabic*\textdegree]



\item - Fica instituído no Estado do Rio de Janeiro a Semana Estadual da Conscientização sobre o Herpes Zoster em suas diversas formas de manifestação, que se realizará anualmente, na primeira semana do mês de fevereiro, com o objetivo de dar ampla divulgação das características desta doença, suas causas e tratamentos dos sintomas, bem como a indicação das medidas preventivas a serem adotadas.

\item - Esta campanha deverá ser desenvolvida por meio da vinculação de anúncios nos meios de comunicação, fixação de cartazes e distribuição de cartilhas nos estabelecimentos de saúde públicos e privados, bem como por meio da realização de palestras e simpósios na rede pública de saúde e de  ensino, realizadas em horários separados para os estudantes e para os demais moradores da comunidade local, podendo abranger outros temas correlatos pertinentes. 

\item - Fica o Poder Executivo autorizado a firmar convênios não onerosos com instituições públicas e particulares, para que sejam elaboradas campanhas publicitárias de divulgação, esclarecimentos e difusão sobre o Herpes Zoster.

\item - As despesas decorrentes da aplicação desta lei correrão por conta de dotações orçamentárias próprias para este fim, suplementadas se necessárias. 

\item - O Anexo da Lei nº 5645, de 06 de Janeiro de 2010 passa a vigorar com a seguinte redação: 


CALENDÁRIO DE DATAS COMEMORATIVAS DO ESTADO DO RIO DE JANEIRO 

(&#8230;)

FEVEREIRO

(&#8230;)

PRIMEIRA SEMANA - SEMANA ESTADUAL DA CONSCIENTIZAÇÃO SOBRE O HERPES ZOSTER.

(...)


\item - Esta Lei entra em vigor na data de sua publicação.


\end{enumerate}




\begin{center}
  Plenário Barbosa Lima Sobrinho, 04 de dezembro de 2019.

   \bigskip

  \textbf{ MÁRCIO CANELLA}

  \bigskip

  \textbf{JUSTIFICATIVA}
  \bigskip

\end{center}

  

O herpes-zoster, é popularmente conhecido como &#``cobreiro&#" e se traduz numa inflamação aguda causada pelo mesmo vírus da catapora (Vírus Varicela-Zoster - VVZ).  Após desenvolver a catapora, o que normalmente acontece na infância, o indivíduo fica com o vírus adormecido no sistema nervoso, podendo ocorrer a reativação desse vírus pela ocorrência de alguns fatores, dentre eles o estresse do dia a dia, a grande exposição ao sol e a baixa imunidade, que pode ser provocada por uma simples gripe por exemplo.

Seu principal sintoma é a dor intensa na extensão do nervo da medula espinhal até a pele, o que pode se manter mesmo após a cura das lesões - É a chamada &#``neuralgia pós-herpética&#".  Na maioria dos casos, tal neuralgia se resolve nos primeiros três meses, mas em alguns casos pode persistir por anos.  No Brasil, a cada ano, registram-se cerca de 10.000 hospitalizações no sistema público por varicela (catapora) e zoster, sendo que a taxa de mortalidade por complicações em adultos aumenta a partir dos 50 anos de idade.

A dor associada ao zoster pode perturbar o sono, o humor, o trabalho e as atividades cotidianas, impactando negativamente a qualidade de vida e levando ao distanciamento social e à depressão. O zoster na região dos olhos costuma ter complicações frequentes e pode afetar a visão de forma permanente.  Para o tratamento do zoster são utilizados, em geral, medicamentos antivirais, na tentativa de diminuir o tempo, o nível de gravidade e as complicações; analgésicos para reduzir a dor e corticosteróides para reduzir o processo inflamatório.  Há também a disponibilidade de vacina que é recomendada pelas autoridades da saúde para pessoas com mais de 50 anos.

Todavia, estas informações são desconhecidas de grande parte da população, o que potencializa os danos decorrentes da doença em decorrência da gravidade das consequências de um não tratamento.  Diante do exposto, proponho uma campanha anual que venha a esclarecer a população fluminense, em especial no período de verão,onde a exposição solar se intensifica, pelo que conto com o apoio dos meus nobres pares para a provação da presente proposição



\iffalse
\begin{center}
  \textbf{REFERÊNCIAS}
\end{center}


\fi



\end{document}

