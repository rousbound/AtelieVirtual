\documentclass[10pt]{article}
\usepackage[portuguese]{babel}
\usepackage[utf8]{inputenc}
\usepackage[pdftex]{graphicx}
\usepackage{venndiagram}
\usepackage{subcaption}
\usepackage{caption}
\usepackage[backend=biber,style=authoryear-ibid]{biblatex}
\usepackage[normalem]{ulem}
\usepackage[margin=0.8in]{geometry}
\addbibresource{psychoanalysis.bib}
\graphicspath{{Pictures/}}
\usepackage{tikz}
\usepackage{setspace}
\usepackage{enumitem}
\usepackage{textcomp}
\usepackage{hyperref}


\date{}

\newcommand{\quotebox}[3]{
  \begin{center}
\noindent\fbox{ 
  \parbox{#3\textwidth}{%
  {\itshape#1\itshape}

  \raggedleft {\textbf{#2}} 
    }%  
}
\end{center}
}

\newcommand{\spawnfig}[3]
{
  \begin{figure}[h]
  \centering
  \includegraphics[scale={#3}]{#1}
  \caption{#2}
  \end{figure}
}


\begin{document}
\maketitle
\begin{center}
  % Deputado ALANA PASSOS, ALEXANDRE KNOPLOCH, ANDERSON  MORAES, ANDRÉ L. CECILIANO, BRUNO DAUAIRE, CORONEL SALEMA, DR. SERGINHO, FILIPPE POUBEL, GIL VIANNA, GUSTAVO SCHMIDT, MARCELO DO SEU DINO, MÁRCIO GUALBERTO, MARCOS MULLER, RENATO ZACA, RODRIGO AMORIM, ROSENVERG REIS, SUBTENENTE BERNARDO, FILIPE SOARES, VANDRO FAMÍLIA, CARLOS MACEDO, GIOVANI RATINHO
  \huge
  \vspace{-3cm}\href{http://alerjln1.alerj.rj.gov.br/scpro1923.nsf/f4b46b3cdbba990083256cc900746cf6/1f8c0a4d545d66688325847a00664775?OpenDocument}{PROJETO DE LEI Nº 1326/2019}
\bigskip
\bigskip
\bigskip
  
\end{center}

\textbf{EMENTA:} 
CONCEDE ANISTIA AOS POLICIAIS MILITARES E BOMBEIROS MILITARES EXCLUIDOS DOS QUADROS EM DECORRENCIA DE ATOS ADMISTRATIVOS-DISCIPLINARES PUNITIVOS








\bigskip

\noindent
A ASSEMBLEIA LEGISLATIVA DO ESTADO DO RIO DE JANEIRO RESOLVE:

\begin{enumerate}[label=Art. \arabic*\textdegree]

\item - Ficam anistiados os policais militares e bombeiros militares excluídos dos quadros em decorrência de atos admistrativos-disciplinares punitivos, editados pelo Secretário de Estado de Segurança, no período de 1º de janeiro de 2007 a 31 de dezembro de 2018, que tiveram sentença penal absolvitória.

Parágrafo único - Compete ao Poder Executivo detalhar os atos que se enquadram no disposto pelo Caput e promover a reintegração dos policiais em até 90 (noventa) dias contados da publicação do presente Decreto.


\item - Esta Lei entra em vigor na data de sua publicação


\end{enumerate}




\begin{center}
  Plenário Barbosa Lima Sobrinho, 19 de setembro de 2019

   \bigskip

  \textbf{ ALANA PASSOS, ALEXANDRE KNOPLOCH, ANDERSON  MORAES, ANDRÉ L. CECILIANO, BRUNO DAUAIRE, CORONEL SALEMA, DR. SERGINHO, FILIPPE POUBEL, GIL VIANNA, GUSTAVO SCHMIDT, MARCELO DO SEU DINO, MÁRCIO GUALBERTO, MARCOS MULLER, RENATO ZACA, RODRIGO AMORIM, ROSENVERG REIS, SUBTENENTE BERNARDO, FILIPE SOARES, VANDRO FAMÍLIA, CARLOS MACEDO, GIOVANI RATINHO}

  \bigskip

  \textbf{JUSTIFICATIVA}
  \bigskip

\end{center}

  A presente iniciativa objetiva corrigir uma indescritível injustiça praticada contra a categoria de servidores públicos do Estado que mais foram expostos, usados e cobrados na administração dos ex-governadores e atuais presidiários Sérgio Cabral e Luiz Fernando Pezão, qual seja, nossos valorosos policiais.

Durante mais de uma década os policiais civis e militares foram submetidos à condições desumanas de serviço, jogados em operações suicidas e zonas deflagradas de conflitos, desprovidos da devida segurança e estrutura básica para o exercício regular de suas funções.

No ano de 2017 o Rio de janeiro registrou 134 policiais mortos, número equivalente a mais de 1/3 dos policiais mortos no país, segundo dados do Fórum Brasileiro de Segurança Pública, resultado de uma política mascarada e fracassada de segurança pública, que longe de resolver as mazelas do Estado, submeteu valentes homens e mulheres a claro risco de morte e de suas integridades físicas, objetivando, unicamente, maquiar o problema de segurança pública junto a população, enquanto se debruçavam em seus devaneios políticos e empreitadas criminosas para saquear os cofres públicos às custas do sacrifício de nossos policiais.

Em que pese o excelente corpo técnico disciplinar das corregedorias de nossas polícias, as circunstâncias políticas que envolveram a gestão da segurança pública do Estado no período escaparam ao teor do processo administrativo disciplinar, aplicando-se a letra fria da Lei sobre  policiais jogados no fronte de batalha, que engrossaram estatísticas eleitoreiras de banimento de servidores, muitos deles com anos de bons serviços prestados à sociedade e que derramaram seu sangue, inúmeras vezes, em favor da população do Estado.

Soma-se ainda questões típicas do regramento militar, tais como o pundonor, previsto no Decreto nº6.579/83 - Regulamento Disciplinar da Polícia Militar do Estado do Rio de Janeiro, responsável por inúmeras exclusões, cuja aplicação desconexa as circunstâncias em que os policiais foram expostos, sobretudo aqueles que atuaram no fracassado Programa das Unidades de Polícias Pacificadoras &#- UPPs, causaram enormes injustiças que devem ser revistas diante do advento de uma sentença penal absolutória. 

Exigências subjetivas de comportamento emocional irrepreensível num contexto de guerra, sem apoio do Estado e da população de localidades dominadas por criminosos, devem ser sopesadas, sobretudo quando evidenciada a interferência de políticos criminosos nas decisões de segurança pública do Estado.

Ressalta-se que o que se traz à baila com a presente proposição não é a confrontação da independência das instâncias civil, penal e administrativa, conforme previsto no artigo 125 da Lei Federal nº 8.112/90, acompanhado pelo art.291 do Decreto nº 2.479/79, estatutos do funcionalismo público federal e estadual, respectivamente, mas a correção da Administração Pública pela não aplicação do Princípio da Autotutela, consagrado no artigo 53 da Lei Federal nº 9.784/99, que tem o condão de revogar seus próprios atos, por conveniência ou oportunidade ou anulá-los quando eivados de vícios (Neste sentido: Súmula STF nº 356: A administração pública pode declarar a nulidade dos seus próprios atos.&#"Súmula STF nº 473: A Administração pode anular seus próprios atos, quando eivados de vícios que os tornam ilegais, porque deles não se originam direitos; ou revogá-los, por motivo de conveniência ou oportunidade, respeitados os direitos adquiridos, e ressalvados, em todos os casos, a apreciação judicial.), circunstâncias essas que se evidenciaram com a prisão dos ex-governadores e diversas autoridades daqueles Governos, descortinando interesses escusos nas demissões de policiais civis e exclusão de policiais militares, não obstante cada caso revisto na esfera penal do Poder Judiciário.

Tal revisão, repita-se, no singular caso do Rio de Janeiro que tem dois governadores presos, tornar-se absolutamente oportuna, vindo o presente Decreto tratar, exclusivamente, aqueles fatos que ensejaram a exclusão do policial militar na instância administrativa, mas que posteriormente, pelo mesmo fato, obtiveram sentença absolutória negando a causa que o excluiu, liame necessário à comunicabilidade entre a instância penal com a administrativa, conforme previsto no art. 126 da Lei Federal nº 8.112/90. 

Portanto, tornou-se claro que a apreciação das exclusões arbitrárias pelo Poder Judiciário,  pela via da ação ordinária que resultou em sentença penal absolutória, neste caso concreto, deve trazer efeitos à esfera administrativa, diante de todo o cenário político que aprisionou os agentes de segurança pública do Estado em benefício de projetos pessoais de agentes políticos, contexto repaginado com a absolvição do policial no Poder Judiciário, que em última análise, desfez a injustiça que vitimou aquele servidor na esfera penal, devendo, contudo, ser estendida à esfera administrativa e materializada com sua reintegração, razão da presente proposição. 

Ademais, o presente ato, além de atuar no resgate da dignidade humana dos agentes de segurança injustiçados e de suas famílias, irá irradiar um poder motivador às atuais forças policiais, pois se depararão com uma concreta medida de apoio e reparação das injustiças praticadas neste nefasto período, vindo a contribuir, diretamente, no fortalecimento da categoria e consequente valorização da classe e aperfeiçoamento das políticas de segurança do Estado.

Visitando matérias jornalísticas da época, vale destacar uma delas contida no site da CBN, datada de 14/02/2017 (Disponível no site: https://cbn.globoradio.globo.com/rio-de-janeiro/2017/02/14/NOS-ULTIMOS-CINCO-ANOS-850-AGENTES-FORAM-EXPULSOS-DAS-POLICIAS-DO-RJ.htm) , em que foi levantando o quantitativo de cerca de 730 policiais militares expulsos anualmente no período de 2012 a 2016. Tal número representa, aproximadamente, o efetivo de um batalhão de grande porte, enfraquecendo o contingente de policiais militares frente a índices alarmantes de criminalidade.

Além das mortes, a exposição desumana de nossos policiais militares tem causado enormes efeitos a sua saúde psicológica. No ano de 2016, foram 1.937 afastados das ruas por problemas psiquiátricos, em quanto em 2017 foram 1.659 casos (Disponível no site: https://www1.folha.uol.com.br/cotidiano/2018/09/afastados-das-ruas-policiais-cariocas-sofrem-efeitos-psiquiatricos.shtml.) Esses dados evidenciam o desequilíbrio entre o rigor das punições, pautadas em legislações arcaicas, e a situação real dos policiais nas ruas do Rio de Janeiro. 

Todo este contexto influenciou diretamente no caos na Segurança Pública do Estado que ocasionou, no ano seguinte, a intervenção federal sob o comando do Exército Brasileiro, reforçando o fracasso das medidas adotadas, entre elas as exclusões arbitrárias em tela.

Por fim, não se deve utilizar do argumento de aumento de despesa de pessoal, item vedado pelo Regime de Recuperação Fiscal do Estado - RRF, visto que o caso configura-se preenchimento de vacância, na forma do artigo 33 da Lei Federal nº 8.112/90 e artigo 61 do Decreto nº 2479/79, exceção prevista no RRF.

Certos de que a proposição vem ao encontro dos anseios da população e da política de valorização dos policiais do Estado empreendida pelo Excelentíssimo Governador e pelos  nobres parlamentares, encaminhamos à consideração de Vossas Excelências, rogando a aprovação.




\iffalse
\begin{center}
  \textbf{REFERÊNCIAS}
\end{center}


\fi



\end{document}

