\documentclass[10pt]{article}
\usepackage[portuguese]{babel}
\usepackage[utf8]{inputenc}
\usepackage[pdftex]{graphicx}
\usepackage{venndiagram}
\usepackage{subcaption}
\usepackage{caption}
\usepackage[backend=biber,style=authoryear-ibid]{biblatex}
\usepackage[normalem]{ulem}
\usepackage[margin=0.8in]{geometry}
\addbibresource{psychoanalysis.bib}
\graphicspath{{Pictures/}}
\usepackage{tikz}
\usepackage{setspace}
\usepackage{enumitem}
\usepackage{textcomp}
\usepackage{hyperref}


\date{}

\newcommand{\quotebox}[3]{
  \begin{center}
\noindent\fbox{ 
  \parbox{#3\textwidth}{%
  {\itshape#1\itshape}

  \raggedleft {\textbf{#2}} 
    }%  
}
\end{center}
}

\newcommand{\spawnfig}[3]
{
  \begin{figure}[h]
  \centering
  \includegraphics[scale={#3}]{#1}
  \caption{#2}
  \end{figure}
}


\begin{document}
\maketitle
\begin{center}
  % Deputado BEBETO, LÉO VIEIRA
  \huge
  \vspace{-3cm}\href{http://alerjln1.alerj.rj.gov.br/scpro1923.nsf/f4b46b3cdbba990083256cc900746cf6/7c79328e377226f80325856d003f1393?OpenDocument}{PROJETO DE LEI Nº 2744/2020}
\bigskip
\bigskip
\bigskip
  
\end{center}

\textbf{EMENTA:} 
DETERMINA A OBRIGAÇÃO DE AFERIÇÃO DE TEMPERATURA CORPORAL NOS COMÉRCIOS E AGÊNCIAS BANCÁRIAS, AUTORIZADOS A FUNCIONAR POR SEREM SERVIÇOS ESSENCIAIS, NA FORMA QUE MENCIONA








\bigskip

\noindent
A ASSEMBLEIA LEGISLATIVA DO ESTADO DO RIO DE JANEIRO RESOLVE:

\begin{enumerate}[label=Art. \arabic*\textdegree]
\item - Ficam OS COMÉRCIOS E AGÊNCIAS BANCÁRIAS, AUTORIZADOS A FUNCIONAR, localizados no âmbito do Estado do Rio de Janeiro, obrigados a utilizar termômetros digitais para medição da temperatura de clientes e funcionários como medida de prevenção a disseminação da COVID-19, enquanto durarem os efeitos do Estado de Calamidade em Saúde.
\item - O Aparelho a ser utilizado será o termômetro infravermelho
Parágrafo Único. Havendo ocorrências de identificação de temperatura fora dos parâmetros acima (37,5º), determina-se:
A - No Caso de Funcionário, o mesmo não poderá exercer suas atividades e será instruído a procurar ou será encaminhado ao serviço médico;
B - No Caso de Cliente, o mesmo não poderá a entrar no estabelecimento, também sendo aconselhado a procurar o serviço médico.
\item - Os Estabelecimentos abrangidos por esta lei deverão colocar em local visível cartazes contendo a referida Lei.
\item - A inobservância das disposições contidas na presente lei sujeitará os infratores às seguintes penalidades:
\begin{enumerate}[label=\Roman*]
\item - Advertência;
\item - Suspensão temporária dos serviços;
\item - interdição do estabelecimento;
\item - Multa diária de 1.000 Ufir.
\end{enumerate}
\item - Esta lei entra em vigor na data de sua publicação.

\end{enumerate}




\begin{center}
  Plenário Barbosa Lima Sobrinho, 08 de Junho de 2020.

   \bigskip

  \textbf{ BEBETO, LÉO VIEIRA}

  \bigskip

  \textbf{JUSTIFICATIVA}
  \bigskip

\end{center}

  O bom seria que não houvesse a necessidade de adotar medidas que possam ser consideradas antipáticas, mas estamos vivendo um momento em que temos que, a cada dia, adotar medidas de maior controle. Os chamados assintomáticos não sabem que carregam o vírus e, com este desconhecimento, sem mesmo querer produzir qualquer prejuízo para a sociedade, este cidadão pode infectar um número considerável de pessoas, assim sendo, não identificamos nesta proposta qualquer atitude que não seja o objeto principal: oferecer proteção a todos que, por absoluta necessidade, são obrigados a trabalhar nessas unidades, como aqueles que as buscam para resolver suas necessidades de compra ou financeira. 



\iffalse
\begin{center}
  \textbf{REFERÊNCIAS}
\end{center}


\fi



\end{document}

