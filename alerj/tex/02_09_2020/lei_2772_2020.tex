\documentclass[10pt]{article}
\usepackage[portuguese]{babel}
\usepackage[utf8]{inputenc}
\usepackage[pdftex]{graphicx}
\usepackage{venndiagram}
\usepackage{subcaption}
\usepackage{caption}
\usepackage[backend=biber,style=authoryear-ibid]{biblatex}
\usepackage[normalem]{ulem}
\usepackage[margin=0.8in]{geometry}
\addbibresource{psychoanalysis.bib}
\graphicspath{{Pictures/}}
\usepackage{tikz}
\usepackage{setspace}
\usepackage{enumitem}
\usepackage{textcomp}
\usepackage{hyperref}


\date{}

\newcommand{\quotebox}[3]{
  \begin{center}
\noindent\fbox{ 
  \parbox{#3\textwidth}{%
  {\itshape#1\itshape}

  \raggedleft {\textbf{#2}} 
    }%  
}
\end{center}
}

\newcommand{\spawnfig}[3]
{
  \begin{figure}[h]
  \centering
  \includegraphics[scale={#3}]{#1}
  \caption{#2}
  \end{figure}
}


\begin{document}
\maketitle
\begin{center}
  %  PODER EXECUTIVO
  \huge
  \vspace{-3cm}\href{http://alerjln1.alerj.rj.gov.br/scpro1923.nsf/f4b46b3cdbba990083256cc900746cf6/2faf228451d0fdbf0325858b005252b7?OpenDocument}{PROJETO DE LEI Nº 2772/2020}
\bigskip
\bigskip
\bigskip
  
\end{center}

\textbf{EMENTA:} 
DISPÕE SOBRE INSTITUIÇÃO DE UM REGIME DIFERENCIADO DE TRIBUTAÇÃO PARA O SETOR ATACADISTA, COM BASE NO § 8º DO ART. 3º DA LEI COMPLEMENTAR Nº 160, DE 7 DE AGOSTO DE 2017, E NA CLÁUSULA DÉCIMA TERCEIRA DO CONVÊNIO ICMS N° 190/2017, NOS TERMOS EM QUE ESPECIFICA








\bigskip

\noindent
A ASSEMBLEIA LEGISLATIVA DO ESTADO DO RIO DE JANEIRO RESOLVE:

\begin{enumerate}[label=Art. \arabic*\textdegree]

\item - Fica instituído, com base no § 8º do art. 3º da Lei Complementar nº 160, de 7 de agosto de 2017, e na cláusula décima terceira do Convênio ICMS n° 190/2017, o regime diferenciado de tributação para o setor atacadista, lastreado nos art. 16 da Lei n° 10.568/2016 e art. 5-A, inc. VII, da Lei n° 7.000/2001, do Estado do Espírito Santo, nos termos previstos nesta Lei.

\item - O tratamento tributário de que trata esta Lei implica a concessão dos seguintes incentivos fiscais:
\begin{enumerate}[label=\Roman*]
\item - crédito presumido nas operações de saídas interestaduais, de modo que a carga tributária efetiva seja equivalente a 1,1 % (um inteiro e dez centésimos por cento), vedado o aproveitamento de outros créditos relacionados a tais operações;
\item - diferimento do ICMS nas operações de importação de mercadorias para o momento da saída, realizada pela diretamente empresa, por conta e ordem ou por encomenda, devendo o referido imposto ser pago englobadamente com o devido pela saída, conforme alíquota de destino, não se aplicando o disposto no artigo 39 do Livro I do Regulamento do ICMS aprovado pelo Decreto nº 27.427, de 17 de novembro de 2000.
\end{enumerate}

\item - Poderão aderir ao regime tributário de que trata esta Lei os estabelecimentos atacadistas que realizem operações com máquinas e equipamentos para contribuintes do ICMS, mesmo quando realizarem ajustes técnicos nas mercadorias para fins de atendimento de exigências constantes de leis e/ou atos administrativos ou simples substituição de embalagem.

\item - O regime de tributação de que trata esta Lei não se aplica ao estabelecimento atacadista que seja filial de indústria localizada em outra unidade da Federação, ressalvado o disposto no art. 11° desta Lei.

\item - As alíquotas de ICMS que envolvam operações internas realizadas por estabelecimentos atacadistas ficam fixadas em:
\begin{enumerate}[label=\Roman*]
\item - sete por cento) nos produtos que compõem a cesta básica;
\item - doze por cento), sendo 2% (dois por cento) destinado ao Fundo Estadual de Combate à Pobreza e às Desigualdades Sociais - FECP, nos demais casos.
\end{enumerate}
Parágrafo único. O crédito do ICMS relativo às aquisições de mercadorias destinadas a comercialização no mercado interno fica limitado a:
\begin{enumerate}[label=\Roman*]
\item - sete) por cento, nos produtos que compõem a cesta básica;
\item - doze) por cento, nos demais casos.
\end{enumerate}

\item - O estabelecimento comercial atacadista enquadrado no regime de tributação de que trata está Lei será responsável pelo recolhimento do ICMS devido nas operações subsequentes, no caso de comercialização de mercadorias sujeitas ao regime de substituição tributária, indicadas em ato normativo expedido pelo Poder Executivo, não se aplicando o disposto no art. 23, inciso IV, item 2, da Lei nº 2.657 de 26 de dezembro de 1996.
§1º Na saída interna para contribuinte, a base de cálculo do ICMS retido por substituição tributária será obtida pela opção, efetuada por meio de ato normativo expedido pelo Poder Executivo, dentre as seguintes técnicas:
\begin{enumerate}[label=\Roman*]
\item - adicionando-se ao valor de partida os valores correspondentes a frete e carreto, seguro, imposto e outros encargos transferíveis ao destinatário e a margem de valor agregado (MVA) indicada em ato normativo expedido pelo Poder Executivo;
\item - pelo preço médio ponderado ao consumidor final (PMPF); ou
\item - pelo preço máximo ao consumidor (PMC).
\end{enumerate}
§2º Considera-se como valor de partida a que se refere o inciso I do § 1º deste artigo, o valor da operação de saída constante da Nota Fiscal do estabelecimento atacadista.
§3º O imposto devido por substituição tributária pelo contribuinte comercial atacadista será calculado mediante a aplicação da alíquota prevista no art. 3º e será recolhido em separado, deduzindo-se do valor obtido o ICMS próprio destacado na Nota Fiscal de saída.

\item - Para fazer jus ao regime tributário de que trata esta Lei, a empresa beneficiária deverá:
\begin{enumerate}[label=\Roman*]
\item - assegurar o recolhimento mensal mínimo equivalente à média aritmética de recolhimento de ICMS nos últimos 12 meses anteriores à adesão ao regime;
\item - ter como objeto social exclusivo o comércio atacadista de mercadoria;
\item - estar em situação de regularidade fiscal e cadastral junto à Secretaria de Estado de Fazenda; 
\item - estar em situação de regularidade junto à Dívida Ativa do Estado do Rio de Janeiro;
\item - não efetuar vendas para contribuintes localizados no Estado do Rio de Janeiro por meio de estabelecimentos localizados em outros Estados da Federação.
\end{enumerate}
§1° O valor complementar recolhido para fins de observância do disposto no inciso I do caput deste artigo poderá ser utilizado como crédito nos períodos subsequentes em que houver ICMS a recolher em montante superior ao valor mínimo, desde que a compensação ocorra no prazo máximo de 12 meses, a contar do período subsequente em que houver o recolhimento complementar, observado o recolhimento mínimo previsto no I do caput deste artigo.
§2º Para os estabelecimentos que, na data de solicitação de enquadramento, ainda não tenham efetuado doze recolhimentos, para fins de apuração da média a que se refere o inciso I do caput deste artigo, o valor mínimo de ICMS a recolher deverá ser equivalente a 1,1% (um inteiro de dez centésimos por cento) do valor faturado no respectivo período de apuração.
§ 3º Empresas que tenham sido criadas a partir de reorganização societária, deverão obedecer ao limite de recolhimento mínimo que seria aplicável à empresa sucedida, nos termos previstos no inciso I do caput deste artigo.
§ 4º As regras de recolhimento mínimo previstas neste artigo poderão ser flexibilizadas, excepcionalmente, em caso de recessão econômica ou de ocorrência de motivo de força maior que impossibilite o seu cumprimento, mediante decisão fundamentada em critérios técnicos, proferida por órgão competente definido em ato normativo expedido pelo Chefe do Poder Executivo.

\item - Para fins do disposto nesta Lei, será considerado estabelecimento atacadista apenas aquele que atender, cumulativamente, aos seguintes requisitos:
\begin{enumerate}[label=\Roman*]
\item - possuir área de armazenagem e estoque de produtos localizados no Estado do Rio de Janeiro de, no mínimo, 1.000 m² (mil metros quadrados);
\item - comprovar que, no trimestre imediatamente anterior à protocolização do pedido de enquadramento, comercializou mercadorias com pelo menos 1.000 (mil) estabelecimentos distintos e não interdependentes do beneficiário, inscritos no Cadastro do RJ - CAD ICMS. 
\item - apresentar movimentação de carga no local; e
\item - gerar empregos diretos ou indiretos no Estado do Rio de Janeiro.
\end{enumerate}
§1° Nas hipóteses envolvendo a comercialização de mercadorias por atacadistas para lojas de conveniência, estabelecidas em postos de serviços e abastecimento de combustíveis, a exigência prevista no inciso II será reduzida para, no mínimo, 100 (cem) outros estabelecimentos não interdependentes.
§2º Para efeitos do inciso III do caput deste artigo, não se considera movimentação de carga o transbordo de mercadorias.
§3º Para cumprimento do disposto no inciso IV do caput deste artigo, será exigida a contratação de profissionais das seguintes especializações:
\begin{enumerate}[label=\Roman*]
\item - vendedores externos;
\item - encarregado de logística;
\item - conferente;
\item - separador;
\item - motorista;
\item - ajudante de caminhão.
\end{enumerate}
§ 4º Os profissionais mencionados no § 3º deste artigo podem ser terceirizados, desde que sejam contratados por empresas localizadas no Estado do Rio de Janeiro ou sejam profissionais autônomos residentes no Estado do Rio de Janeiro.

\item - Perderá o direito a fruição do regime tributário previsto nesta Lei, com a consequente restauração da sistemática convencional de apuração do ICMS, o estabelecimento beneficiário que deixar de cumprir os requisitos ou condições previstos nos artigos 7º e 8º.
Parágrafo único. O desenquadramento do regime tributário de que trata esta Lei retroagirá à data em que for identificado o descumprimento dos requisitos ou condições previstos nos artigos 5° e 6º desta Lei.

\item - Fica vedada a utilização do regime de tributação de que trata esta Lei para as operações com as seguintes mercadorias:
\begin{enumerate}[label=\Roman*]
\item - com café, energia elétrica, lubrificantes, combustíveis líquidos e gasosos, derivados ou não de petróleo e produtos fármacos;
\item - que destinem mercadorias a consumidor final;
\item - com cacau e pimenta-do-reino in natura e couro bovino;
\item - de venda, ou remessa a qualquer título, de mercadoria ou bem, nos casos em que o adquirente, ou destinatário, localizado em outra unidade da Federação, determine que o estabelecimento alienante, ou remetente, localizado neste Estado, promova a sua entrega a destinatário localizado neste Estado, inclusive na hipótese de venda à ordem;
\item - nas transferências de mercadorias ou bens importados sujeitos aos efeitos da Resolução nº 13, de 2012, do Senado Federal;
\item - nas operações internas, com os produtos abaixo relacionados:
\end{enumerate}
a) fio-máquina de ferro ou aços não ligados - código NCM 72.13;
b) barras de ferro ou aços não ligados, simplesmente forjadas, laminadas, estiradas ou extrudadas, a quente, incluídas as que tenham sido submetidas a torção após laminagem - código NCM 72.14;
c) outras barras de ferro ou aços não ligados - código NCM 72.15;
d) perfis de ferro ou aços não ligados - código NCM 72.16;
e) fios de ferro ou aços não ligados - código NCM 72.17;
f) cordas, cabos, tranças (entrançados*), lingas e artefatos semelhantes, de ferro ou aço, não isolados para usos elétricos - código NCM 73.12;
g) arame farpado, de ferro ou aço; arames ou tiras, retorcidos, mesmo farpados, de ferro ou aço, dos tipos dos utilizados em cercas - código NCM 73.13;
h) telas metálicas (incluídas as telas contínuas ou sem fim), grades e redes, de fios de ferro ou aço; chapas e tiras, distendidas, de ferro ou aço - código NCM 73.14;
i) tachas, pregos, percevejos, escápulas, grampos ondulados ou biselados e artefatos semelhantes, de ferro fundido, ferro ou aço, mesmo com cabeça de outra matéria, exceto cobre - código NCM 73.17; e
j) parafusos, pinos ou pernos, roscados, porcas, tira-fundos, ganchos roscados, rebites, chavetas, cavilhas, contrapinos ou troços, arruelas (anilhas*) (incluídas as de pressão) e artefatos semelhantes, de ferro fundido, ferro ou aço - código NCM 73.18.
Parágrafo único. Fica o Poder Executivo autorizado a ampliar o rol das vedações previstas no caput, tendo em vista o interesse público.

\item - Os estabelecimentos atacadistas que sejam filiais de empresas localizadas em outros Estados da Federação poderão requerer o enquadramento no regime tributário de que trata esta Lei para fins de realização exclusiva de operações interestaduais, não se aplicando a tais estabelecimentos o disposto no inciso II do art. 6° desta Lei. 
Parágrafo único. No caso de enquadramento previsto no caput deste artigo, a realização de operações de saídas internas será tributada de acordo com as regras de tributação previstas na Lei 2.657, de 26 de dezembro de 1996, não se aplicando o disposto nesta Lei.

\item - O disposto no art. 11 aplica-se às empresas de comércio exterior atacadistas que promovam importação de mercadorias pelos portos ou aeroportos localizados em território fluminense, ficando dispensadas do cumprimento do disposto no inciso II do art. 7° e no art. 8° desta Lei.
Parágrafo único. A adesão ao regime previsto nessa Lei, para os estabelecimentos de que trata o caput deste artigo, fica condicionada a comprovação de habilitação para a prática de atos no Sistema Integrado de Comércio Exterior (Siscomex) na modalidade ilimitada (Radar), conforme requisitos estabelecidos pela Receita Federal do Brasil.

\item - A adesão ao regime de tributação de que trata esta Lei deverá ser requerida nos termos previstos em ato normativo expedido pelo Poder Executivo.
Parágrafo único. Os requerimentos de adesão ao regime tributário de que trata esta Lei deverão ser apreciados pelo órgão competente, obedecendo-se a ordem cronológica de solicitação.

\item - A adesão ao regime tributário de que trata esta Lei implica a renúncia a qualquer outro regime diferenciado de tributação.
Parágrafo único.  Os contribuintes beneficiários de qualquer outro regime diferenciado de tributação poderão aderir ao regime de tributação de que trata esta Lei, sendo-lhes assegurado o direito de usufruir do regime antigo até que advenha decisão administrativa favorável à adesão.

\item - Fica revogada a Lei nº 4.173, de 29 de setembro de 2003, preservando-se os seus efeitos para os contribuintes que firmaram termos de acordos até prazo final neles previstos, observado o disposto no parágrafo único do art. 14.

\item - Esta Lei entra em vigor no primeiro dia do segundo mês subsequente ao de sua publicação e produzirá seus efeitos até a data prevista no art. 3°, § 2°, inc. III, da Lei Complementar n° 160, de 07 de agosto de 2017.
WILSON WITZEL
Governador


\end{enumerate}




\begin{center}
  

   \bigskip

  \textbf{  PODER EXECUTIVO}

  \bigskip

  \textbf{JUSTIFICATIVA}
  \bigskip

\end{center}

  MENSAGEM Nº 25/2020EXCELENTÍSSIMOS SENHORES PRESIDENTE E DEMAIS MEMBROS DA ASSEMBLEIA LEGISLATIVA DO ESTADO DO RIO DE JANEIRO
Tenho a honra de submeter à deliberação de Vossas Excelências o incluso Projeto de Lei que &#``DISPÕE SOBRE INSTITUIÇÃO DE UM REGIME DIFERENCIADO DE TRIBUTAÇÃO PARA O SETOR ATACADISTA, COM BASE NO § 8º DO ART. 3º DA LEI COMPLEMENTAR Nº 160, DE 7 DE AGOSTO DE 2017, E NA CLÁUSULA DÉCIMA TERCEIRA DO CONVÊNIO ICMS N° 190/2017, NOS TERMOS EM QUE ESPECIFICA&#". 
Inicialmente, cumpre ressaltar que o Setor Atacadista tem um papel muito relevante para a economia fluminense. Todavia, os contribuintes fluminenses que atuam nesse setor estão perdendo mercado para estabelecimentos localizados em outros Estados da Federação que ofertam uma tributação menos onerosa. Isso tem implicado um crescimento das aquisições interestaduais por comerciantes varejistas, em evidente prejuízo ao erário fluminense.
Atualmente, o Estado do Rio de Janeiro dispõe de um regime diferenciado de tributação para o Setor atacadista. Tal regime decorre da combinação de dois diplomas normativos: a Lei nº 4.173/2003 e o Decreto 44.498/2013. No entanto, o regramento atual é bastante complexo e marcado por um processo burocrático que privilegia alguns contribuintes em detrimento de outros. 
Nesse contexto, esta proposta visa a aderir ao regime diferenciado de tributação para o setor atacadista previsto na legislação do Estado do Espírito Santo (art. 16 da Lei nº 10.568/2016 e inc. VII do art. 5-A da Lei nº 7.000/2001), consoante autorizam o § 8º do art. 3º da Lei Complementar nº 160/2017, e a cláusula décima terceira do convênio ICMS n° 190/2017.
De modo geral, a presente proposta está estruturada da seguinte forma:
O art. 1º da proposta faz alusão ao objeto do projeto que é a criação de um regime diferenciado de tributação, com fulcro no direito de adesão previsto no § 8º do art. 3º da LC nº 160/2017 e na cláusula décima terceira do Convênio ICMS nº 190/2017.
O art. 2º da proposta estabelece os incentivos ficais que ficam concedidos, sendo certo que seu inciso I concede crédito presumido nas operações interestaduais, de modo que a tributação efetiva seja equivalente a 1,1\% (um inteiro e dez centésimos por cento), reproduzindo, com simplificação da técnica, a regra prevista no caput do art. 16 da Lei nº 10.568/2016 do Estado do Espírito Santo. 
No inciso II do art. 2º, concede-se diferimento do ICMS na operação de importação de mercadorias para o momento da saída, realizada diretamente pela empresa, por conta e ordem ou por encomenda, devendo o imposto ser pago englobadamente com o devido na saída. 
Trata-se de uma regra não prevista na legislação que serviu de paradigma. Não obstante, não se trata de uma ampliação do regime diferenciado de tributação que serviu de referência, pois o diferimento não é considerado um incentivo fiscal, mas, sim, uma mera técnica de tributação. Assim, a concessão de diferimento do ICMS não demanda autorização prévia do CONFAZ, segundo a jurisprudência consolidada do STF.
Ademais, esse diferimento já vem sendo praticado no Estado do Rio de Janeiro, com base no inciso II do art. 1° do Decreto n° 44.498 de 29 de novembro de 2013. A única diferença é que se deixa claro que o diferimento também pode ocorrer no caso de importações por conta e ordem e por encomenda. 
O art. 3º deixa claro algo que já está implícito, no sentido de que os estabelecimentos atacadistas que revendam máquinas e equipamentos para contribuintes de ICMS poderão aderir ao regime, mesmo que realizem ajustes técnicos nas mercadorias para fins de atendimento de exigências constantes de leis e/ou atos administrativos ou simples substituição de embalagem. Não se trata de uma ampliação do incentivo fiscal, uma vez que a legislação que serve de paradigma não contém qualquer vedação à fruição do regime por tais contribuintes.  Trata-se apenas de um esclarecimento com finalidade de deixar claro que o estabelecimento atacadista que realizar pequenos ajustes nas mercadorias, que, em tese, poderiam configurar industrialização, poderão aderir ao regime.
O art. 4º estabelece uma regra, não prevista na legislação que serviu de paradigma, que veda a adesão ao regime ao estabelecimento atacadista que seja filial de indústria localizada em outro Estado da Federação. 
Esta regra visa a prestigiar o contribuinte sediado no Estado do Rio de Janeiro, dificultando sua mera substituição por outros estabelecimentos sem um correspondente incremento de arrecadação. Trata-se de uma regra restritiva que não encontra óbice no direito de adesão, previsto no § 8º do art. 3º da LC n° 160/2017. Ademais, é uma regra que já vem sendo praticada no Estado do Rio de Janeiro, nos termos do Decreto n° 44.498/2013. 
O art. 5º estabelece que as alíquotas de ICMS nas operações internas realizadas por estabelecimentos atacadistas serão de: i) 7\% (sete por cento) nos produtos que compõem a cesta básica; e ii) 12\% (doze por cento), sendo 2\% (dois por cento) destinado ao Fundo Estadual de Combate à Pobreza e às Desigualdades Sociais &#- FECP, nos demais casos. 
Nesse ponto específico, optou-se por uma alteração da técnica de tributação prevista no ato normativo que serviu de paradigma (art. 5-A, inc. VII, da Lei nº 7.000/2001, do Estado do Espírito Santo). O art. 5º da minuta estabelece as alíquotas de 7\% (sete por cento) ou de 12\% (doze por cento) nas operações que especifica, enquanto que a legislação que serve de paradigma prevê a redução da base de calculo, de modo que a carga tributária efetiva resulte em 7\% (sete por cento). 
Ora, a base de cálculo e a alíquota são elementos que compõem o aspecto quantitativo da regra matriz de incidência. Com efeito, é possível alterar a tributação por meio da alteração da base de cálculo ou da alíquota. Mas o que importa, de fato, é a carga tributária efetiva, independentemente da técnica utilizada. 
Nesse contexto, a proposta inserida no art. 5º da proposta estabelece a mesma carga tributária prevista na legislação capixaba (inc. VII do art. 5-A da Lei nº 7.000/2001), 7\%, no caso dos produtos que compõem a cesta básica, e de 12\% nas operações com outras mercadorias. 
Desse modo, a alteração da técnica de tributação não implica uma ampliação do incentivo fiscal que serviu de paradigma, mas, sim, uma redução, o que é permitido. 
Por outro lado, mesmo que não se tratasse do exercício do direito de adesão, previsto no § 8º do art. 3º da LC nº 160/2017 e na cláusula décima terceira no Convênio ICMS nº 190/2017, a fixação das alíquotas, nos termos do art. 5º da minuta, encontra respaldo no ordenamento jurídico pátrio.
Isso porque o Convênio ICMS nº 128/1994 autorizou os Estados a tributarem a cesta básica em 7\% (sete por cento). Já a fixação da alíquota de ICMS em 12\% (doze por cento) é uma decisão que cabe aos Estados membros, no âmbito da sua discricionariedade política. 
O art. 155, § 2º, inc. V, alínea &#``a&#", da CF/88, preceitua que é facultado ao Senado Federal estabelecer alíquotas mínimas nas operações internas, mediante resolução de iniciativa de um terço e aprovada pela maioria absoluta de seus membros. Adicionalmente, o inciso VI do § 2º do art. 155 da CF/88 enuncia que, salvo deliberação em contrário dos Estados e do Distrito Federal, nos termos do disposto no inciso XII, "g", as alíquotas internas, nas operações relativas à circulação de mercadorias e nas prestações de serviços, não poderão ser inferiores às previstas para as operações interestaduais.
O Senado ainda não exerceu a faculdade de fixar a alíquota mínima de ICMS nas operações internas, mas o fez com relação às operações interestaduais, designadamente no art. 1º da Resolução n° 22/1989, fixando-a em 12\% (doze por cento).
Ora, da interpretação sistemática dessas regras, extrai-se a autonomia legislativa dos Estados para fixar a alíquota do ICMS em 12\% nas operações internas que desejar, sem a necessidade de autorização prévia do CONFAZ, nos termos da LC nº 24/1975.  
Sendo assim, o art. 5º está em perfeita consonância com o ordenamento jurídico pátrio, mesmo que não se tratasse do exercício de adesão, uma vez que: i) a fixação da alíquota do ICMS em 7\% (sete por cento) no caso de operações com produtos da cesta básica está autorizada pelo Convênio ICMS nº 128/1994; e ii) a fixação da alíquota em 12\% (doze por cento) é uma decisão que se encontra dentro do âmbito da autonomia constitucional concedida aos Estados da Federação. 
O Parágrafo único do art. 5º estabelece uma regra de limitação de crédito, tal como ocorre no § 2° do art. 16 da Lei nº 10.568/2016. Não obstante, utilizou-se, na minuta, uma regra diferenciada de limitação no que tange à apropriação de créditos de ICMS. Tal apropriação será limitada: i) a 7\% (sete por cento) nos produtos que compõem a cesta básica; e ii) a 12\% (doze por cento) nos demais casos. 
Ressalte-se que tal alteração não torna o incentivo fiscal proposto na minuta mais benéfico do que aquele que serviu de paradigma. Isso porque, nos produtos que compõem a cesta básica, a regra é idêntica a do ato normativo que serviu de referência (§ 2 do art. 16 da Lei nº 10.568/2016). Nos demais casos, vislumbra-se uma elevação da carga tributária efetiva de 7\% (sete por cento) para 12\% (doze por cento). Logo, é justificável a elevação do percentual de tomada de crédito de ICMS decorrente das aquisições.
O art. 6º estampa uma regra não prevista na legislação que serviu de paradigma que atribui aos estabelecimentos atacadistas aderentes ao regime diferenciado de tributação proposto a condição de responsável tributário pelas operações subsequentes. 
Trata-se de uma norma que reproduz, com meras alterações redacionais, as disposições contidas no art. 8º da LC nº 87/1996 e que materializam apenas uma opção por uma técnica de arrecadação já tradicional no ICMS. 
Ademais, o inc. III do § 6º do art. 5-A da Lei nº 7.000/20001 (que serviu de paradigma) estabelece que o Secretário de Estado de Fazenda poderá conceder a condição de substituto tributário aos estabelecimentos atacadistas. Na minuta, a condição de substituto tributário decorre da própria Lei, em sintonia com o que determina o art. 6º da LC nº 87/1996. 
O art. 7º da proposta estabelece alguns requisitos e condições para a fruição do regime tributário proposto. Tais requisitos tornam o incentivo fiscal proposto na minuta mais restritivo do que o previsto na legislação capixaba (§ 6º do art. 16 da Lei nº 10.568/2016).
O inciso I do art. 7º da minuta estabelece uma regra não prevista na legislação que serviu de paradigma, no sentido de que para fazer jus ao regime tributário diferenciado, o estabelecimento beneficiário deverá assegurar o recolhimento mensal mínimo equivalente à média aritmética de recolhimento de ICMS nos últimos 12 meses anteriores à adesão ao regime. 
Essa regra visa a evitar perda inicial de arrecadação, decorrente da diminuição do ônus tributário. Logo, para usufruir efetivamente de uma diminuição da tributação, o estabelecimento beneficiário terá que ampliar suas operações. Dessa forma, evita-se perda de arrecadação. 
O inciso II do art. 7º estabelece como requisito para adesão ao regime diferenciado de tributação a exploração econômica exclusiva de comércio atacadista, de modo a não subverter a finalidade precípua do projeto. Trata-se de uma norma mais restritiva que a prevista na legislação capixaba que não impõe a exclusividade, mas apenas que a atividade atacadista seja a principal (art. 16, § 6º, inc. I, da Lei nº 10.568/2016). Essa regra busca evitar fraudes na utilização do regime que foi concebido tendo em vista a atividade atacadista e não a varejista.
Os requisitos previstos nos incisos III e IV do art. 7º são requisitos de praxe relacionados à regularidade fiscal e cadastral junto à SEFAZ.
Por fim, o inc. V do art. 7º estabelece uma restrição complementar no sentido de que as empresas que aderirem ao regime diferenciado de tributação não poderão efetuar vendas para contribuintes estabelecidos no Estado do Rio de Janeiro por meio de estabelecimentos próprios localizados em outro Estado da Federação. 
O § 1º do art. 7º estabelece que o eventual recolhimento complementar, previsto no inc. I do art. 7°, poderá ser utilizado como crédito em períodos subsequentes em que houver apuração de ICMS em valor maior do que o mínimo. Essa regra busca evitar que esse recolhimento complementar se torne um custo, designadamente em razão do efeito da sazonalidade, cujo seja a queda nas vendas. Todavia, restringe-se a possibilidade de compensação ao período de 12 meses. 
O § 2º do art. 7º estabelece uma forma diferenciada para o recolhimento mínimo para as empresas novas que ainda não tenham completado 12 (doze) meses de existência na data de solicitação de enquadramento. 
O § 3º do art. 7º estabelece que empresas que tenham sido criadas a partir de reorganização societária, deverão obedecer ao limite de recolhimento mínimo que seria aplicável à empresa sucedida. Tal regra procura inibir a criação artificial de novas empresas apenas para não atender a regra de recolhimento mínimo previsto no inc. I do art. 7º da minuta.
O § 4º do art. 7° estabelece uma regra que permite a flexibilização das regras que estabelecem o recolhimento mínimo no caso de ocorrência de recessão econômica ou motivo de força maior que afete a capacidade econômica do contribuinte beneficiado. Essa possibilidade de flexibilização está prevista no § 2º do art. 2º da Lei n° 8.445/2019. É Importante destacar que a decisão de flexibilização deverá ser respaldada em critérios técnicos.
É importante destacar que as regras relacionadas ao recolhimento mínimo não encontram paralelo na legislação que serviu de paradigma. Todavia, tal inovação restringe o âmbito do regime diferenciado de tributação proposto, o que pode ocorrer no que concerne ao direito de adesão previsto no § 8º do art. 3º da LC nº 160/2017 e na cláusula décima terceira do Convênio ICMS nº 190/2017. Por outro lado, garante uma previsibilidade na arrecadação, de modo a não comprometer o caixa do Estado.
O art. 8º da proposta estabelece um rol de requisitos mínimos para fins de enquadramento no regime diferenciado de tributação.  Tais requisitos transcendem em larga medida os requisitos previstos na legislação capixaba, o que implica o reconhecimento de uma regra mais restritiva do que aquela que serviu de paradigma. Contudo, são restrições necessárias para evitar o desvirtuamento do regime.
O art. 9º estabelece as situações em que o estabelecimento beneficiário perderá o direito de gozar do regime diferenciado de tributação.
Já o art. 10 reproduz as vedações à utilização do regime previstas no § 3º do art. 16 da Lei nº 10.568/2016 e no § 6º do art. 5-A da Lei nº 7.000/2001, subtraindo, apenas, a restrição relacionada às mercadorias sujeitas à substituição tributária no caso de mercadorias já adquiridas com o ICMS retido. A subtração ocorreu porque tal situação é incompatível com a sistemática adotada na minuta, em que o estabelecimento atacadista sempre será o substituto tributário, o que implicará a impossibilidade de aquisição de mercadorias com o ICMS já retido. 
O parágrafo único do art. 10 estabelece a possibilidade de o Poder Executivo ampliar o rol das vedações, tendo em vista o interesse público. Com efeito, será possível tornar o regime diferenciado de tributação, ora proposto, mais restritivo que o previsto na legislação que serviu de paradigma.
O art. 11 do projeto de lei flexibiliza a restrição prevista no inc. II do art. 8º que impõe a necessidade de comprovação de realização de operações com, no mínimo, 1.000 estabelecimento distintos e não interdependentes. Esta regra faz sentido no que tange às operações internas, mas não faz no que concerne às operações interestaduais, por isso foi afastada. 
O parágrafo único do art. 11 estipula que as operações internas realizadas por estabelecimentos enquadrados apenas para fins de realização de operações interestaduais deverão ser tributadas, em separado, de acordo com a sistemática convencional.
O art. 12 da proposta preconiza que não se aplica os requisitos previstos no inciso II do art. 7º e no art. 8º da minuta às empresas de comércio exterior que promovam importação de mercadorias pelos portos ou aeroportos localizados em território fluminense. 
Importa ressaltar que os art. 11 e 12 flexibilizam requisitos que não estão previstos na legislação que serviu de paradigma. Por essa razão, tais flexibilizações não implicam a concessão de um regime tributário mais generoso do que aquele que serviu de referência.
O art. 13 estabelece que o pedido de enquadramento deverá ser feito nos termos estabelecidos em ato normativo expedido pelo Poder Executivo, o qual definirá as questões operacionais relacionadas ao enquadramento.
O art. 14 estabelece uma regra não prevista na legislação que serviu de paradigma, no sentido de que a adesão ao regime diferenciado de tributação implica a renúncia a qualquer outro regime diferenciado de tributação concedido anteriormente ao aderente. Trata-se do acréscimo de uma restrição que visa a evitar a sobreposição de incentivos fiscais. 
O art. 15 revoga a Lei nº 4.173, de 29 de setembro de 2003 (lei que atualmente regula o setor atacadista), preservando os seus efeitos para os contribuintes que firmaram termos de acordos até prazo final neles previstos. Trata-se de uma medida que visa a garantir segurança jurídica no que tange aos acordos já firmados entre o Estado e as empresas beneficiárias do referido regime.
Por fim, o artigo 16 do projeto de lei estabelece a vigência do regime diferenciado de tributação, atento aos limites previstos no art. 3°, § 2°, inc. III, da Lei Complementar nº 160/2017 para os incentivos fiscais concedidos ao setor comercial.
Postos estes esclarecimentos, é imperioso reconhecer que a minuta proposta replica todas as condições previstas na legislação que serviu de paradigma e, ainda, acrescenta outros requisitos e condicionantes que implicam uma redução do incentivo fiscal globalmente considerado, mas que são importantes para adequação à realidade fluminense.
Embora tenha havido alterações relevantes no regime diferenciado de tributação proposto na minuta, em comparação com o regime previsto na legislação que serviu de paradigma (art. 16 da Lei n° 10.568/2016 e inciso VII do art. 5-A da Lei nº 7.000/2001, do Estado do Espírito Santo), não se vislumbra qualquer alteração que implique a caracterização de um regime diferenciado de tributação menos oneroso do que aquele que serviu de referência. 
Por fim, é importante destacar que não há qualquer violação ao regime de recuperação fiscal do Estado do Rio de Janeiro. Isso porque o art. 8º da Lei Complementar nº 159/2017 prescreve que:
&#``Art. 8º São vedados ao Estado durante a vigência do Regime de Recuperação Fiscal:
\item - a concessão ou a ampliação de incentivo ou benefício de natureza tributária da qual decorra renúncia de receita, ressalvados os concedidos nos termos da alínea &#``g&#" do inciso XII do § 2º do art. 155 da Constituição Federal;&#"
Ora, consoante se extrai desse preceito legal, a vedação à concessão ou ampliação de incentivo ou benefício fiscal aos Estados em Recuperação Fiscal só se aplica aos benefícios que não seguiram o rito previsto em Lei Complementar que versa sobre a concessão de incentivos fiscais por parte dos Estados membros. Ou seja, àqueles incentivos fiscais que não seguiram as disposições da Lei Complementar nº 24/1975 ou da Lei Complementar nº 160/2017, diplomas normativos que versam sobre a concessão de incentivos fiscais em matéria de ICMS.
Nesse contexto, a proposição de um regime diferenciado de tributação para o setor atacadista, tomando como parâmetro o regime previsto no art. 16 da Lei nº 10.568/2016 e no inc. VII do art. 5º-A da Lei nº 7.000/2001, ambas do Estado do Espírito Santo, com as alterações propostas, está em perfeita sintonia com as regras postas no ordenamento jurídico pátrio, mormente com as disposições contidas na Lei Complementar nº 160/2017 e no Convênio ICMS nº 190/2017. 
Por fim, a alteração é urgente porque os regimes de tributação como o aqui proposto só poderão produzir efeitos até 31/12/2022. Este é, portanto, o momento ideal para se atrair novas empresas. 
Portanto, considerando o relevante interesse público da matéria, esperamos contar, mais uma vez, com o apoio e o respaldo dessa Egrégia Casa e solicitando que seja atribuído ao processo o regime de urgência, nos termos do artigo 114 da Constituição do Estado, reitero a vossas Excelências o protesto de elevada estima e consideração.
WILSON WITZEL
Governador



\iffalse
\begin{center}
  \textbf{REFERÊNCIAS}
\end{center}


\fi



\end{document}

