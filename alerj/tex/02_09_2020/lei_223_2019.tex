\documentclass[10pt]{article}
\usepackage[portuguese]{babel}
\usepackage[utf8]{inputenc}
\usepackage[pdftex]{graphicx}
\usepackage{venndiagram}
\usepackage{subcaption}
\usepackage{caption}
\usepackage[backend=biber,style=authoryear-ibid]{biblatex}
\usepackage[normalem]{ulem}
\usepackage[margin=0.8in]{geometry}
\addbibresource{psychoanalysis.bib}
\graphicspath{{Pictures/}}
\usepackage{tikz}
\usepackage{setspace}
\usepackage{enumitem}
\usepackage{textcomp}
\usepackage{hyperref}


\date{}

\newcommand{\quotebox}[3]{
  \begin{center}
\noindent\fbox{ 
  \parbox{#3\textwidth}{%
  {\itshape#1\itshape}

  \raggedleft {\textbf{#2}} 
    }%  
}
\end{center}
}

\newcommand{\spawnfig}[3]
{
  \begin{figure}[h]
  \centering
  \includegraphics[scale={#3}]{#1}
  \caption{#2}
  \end{figure}
}


\begin{document}
\maketitle
\begin{center}
  % Deputado ALEXANDRE KNOPLOCH
  \huge
  \vspace{-3cm}\href{http://alerjln1.alerj.rj.gov.br/scpro1923.nsf/f4b46b3cdbba990083256cc900746cf6/e5091e99cf9900e6832583c10061537a?OpenDocument}{PROJETO DE LEI Nº 223/2019}
\bigskip
\bigskip
\bigskip
  
\end{center}

\textbf{EMENTA:} 
INSTITUI NO ÂMBITO DO ESTADO DO RIO DE JANEIRO, O FERIADO DO YOM KIPUR - DIA DO PERDÃO, PARA TODOS QUE SE AUTODECLARAREM PRATICANTES DA RELIGIÃO JUDAICA.








\bigskip

\noindent
A ASSEMBLEIA LEGISLATIVA DO ESTADO DO RIO DE JANEIRO RESOLVE:

\begin{enumerate}[label=Art. \arabic*\textdegree]
\item - Fica instituído o feriado do "Yom Kipur - Dia do Perdão", a todos que se autodeclararem praticantes da religião judaica, no âmbito do Estado do Rio de Janeiro.

Parágrafo único - A data do feriado acompanha o dia estabelecido no calendário judaico anual.

\item - A autodeclaração de que trata o caput do Art. 1º deverá ser feita através de termo de responsabilidade, apresentado no local de trabalho ao setor correspondente.

Parágrafo único - O funcionário, ou servidor, que apresentar falso testemunho para gozar do benefício estará sujeito às sanções penais estabelecidas pela legislação vigente.

\item - O não cumprimento da presente lei acarretará multa de 100 (cem) UFIR-RJ por funcionário.

Ar. 4º - Esta Lei entra em vigor na data da sua publicação.

\end{enumerate}




\begin{center}
  Plenário Barbosa Lima Sobrinho, 18 de março de 2019.

   \bigskip

  \textbf{ ALEXANDRE KNOPLOCH}

  \bigskip

  \textbf{JUSTIFICATIVA}
  \bigskip

\end{center}

  	O feriado de Yom Kipur é considerado o evento mais importante do calendário judaico. É um momento de reflexão, no qual o praticante da religião judaica pede perdão a Deus e aos seus semelhantes por todos os equívocos cometidos no ano anterior, prometendo não repeti-los no próximo. Um dos costumes que caracteriza o feriado é passar o dia todo em jejum o que, muitas vezes, impossibilita um judeu de concluir de forma saudável um dia de trabalho. O Estado do Rio de Janeiro, miscigenado como é, sempre deu exemplo de acolhimento às mais variadas culturas e credos. Caracteriza o nosso calendário os feriados cristãos e a lembrança de datas importantes para todas as culturas e religiões. Faz-se necessário, portanto, o reconhecimento desta Casa Legislativa, em forma de Lei, a um povo que ajudou a construir o nosso Estado e integra o cenário populacional fluminense. Nesse sentido é que apresento a presente proposição. 



\iffalse
\begin{center}
  \textbf{REFERÊNCIAS}
\end{center}


\fi



\end{document}

